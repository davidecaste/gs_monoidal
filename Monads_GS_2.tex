\documentclass{article}
\usepackage[english]{babel}
\usepackage{natbib}
% Useful packages
\usepackage{authblk}
\usepackage[all,2cell]{xy}

%\usepackage{natbib}
\usepackage{graphicx}%
\usepackage{multirow}%
\usepackage{amsmath,amssymb,amsfonts}%
\usepackage{amsthm}%
\usepackage{mathrsfs}%
\usepackage[title]{appendix}%
\usepackage{xcolor}%
\usepackage{manyfoot}%
\usepackage{booktabs}%
\usepackage{algorithm}%
\usepackage{algorithmicx}%
\usepackage{algpseudocode}%
\usepackage{listings}%


\usepackage{quiver}
\usepackage{mathtools}
\usepackage{todonotes}
\newcommand{\fab}[1]{\todo[color=red!30,inline,caption={}]{\textbf{F:} #1}}
\newcommand{\dav}[1]{\todo[color=blue!30,inline,caption={}]{\textbf{D:} #1}}
\newcommand{\cipr}[1]{\todo[color=orange!30,inline,caption={}]{\textbf{C:} #1}}


\newcommand{\comment}[1]{ }
\usepackage{hyperref}
\usepackage{xcolor}
\input{macro_davide}
%%%%%%%%%%%%%%%%%%%%%%%%%%%%our packages%%%%%%%%%%%%%%%%
\usepackage{quiver}
\usepackage{mathtools}

\usepackage{tikz}
\usepackage{tikzit}

\usetikzlibrary{shapes}

%\input{sample.tikzstyle}
\usepackage{freetikz}
\usetikzlibrary{decorations.markings,positioning,patterns}
\usetikzlibrary{shadows}
\usepackage{array}
\usepackage{tikz-cd}

%\usetikzlibrary{automata, positioning, arrows}
\input{stringdiagrams.tikzstyles}

\usepackage[all,2cell]{xy}
\UseAllTwocells
\xyoption{v2}

%%%%%%%%%%%%%%%%%%%%%%%%%%%%%%%%%%%%%




%%%%%%%%%%%%%%%%%%%%%%%%%%%%%%%%%%%%%%%55
\title{A taxonomy of categories for relations\thanks{This research was partly funded by the Advanced Research + Invention Agency (ARIA) Safeguarded AI Programme. 
The authors are indebted to Bart Jacobs, Tobias Fritz and Paolo Perrone for their insightful
comments on a preliminary draft of this paper.}}
\author[1]{Cipriano Junior Cioffo}
\author[1]{Fabio Gadducci} 
\author[2]{Davide Trotta}

\affil[1]{Department of Computer Science, University of Pisa, Pisa, Italy.}
\affil[2]{Department of Mathematics, University of Padova, Padova, Italy.}

\date{}

%%%%%%%%%%%%%%%%%%%%%%%%%%%%%%%%

\theoremstyle{plain} %italico
\newtheorem{mytheorem}{Theorem}[section]
\newtheorem{mycorollary}[mytheorem]{Corollary}
\newtheorem{mylemma}[mytheorem]{Lemma}
\newtheorem{myproposition}[mytheorem]{Proposition}
\newtheorem{mydefinition}[mytheorem]{Definition}


\theoremstyle{definition} %stampatello
\newtheorem{myproblem}[mytheorem]{Problem}
\newtheorem{myremark}[mytheorem]{Remark}
\newtheorem{myexample}[mytheorem]{Example}

\begin{document}
\maketitle

\begin{abstract}
The study of categories abstracting the structural properties of relations has been extensively developed over the years, resulting in a rich and diverse body of work. 
This paper strives to provide a modern and comprehensive presentation of these ``categories for relations'', including their enriched version, further showing how they 
arise as Kleisli categories of suitable symmetric monoidal monads. The resulting taxonomy aims at bringing clarity and 
organisation to the numerous related concepts and frameworks occurring in the literature.
\end{abstract}    

\section{Introduction}
%Storia teoria delle categorie per relazioni: 
%1) richiame che originariamente (in Categories for the Working Mathematician), la nozione di categoria astraeva il concetto di insieme e FUNZIONE 
Category theory, from its very beginnings, was conceived as an abstraction of the notions of set and function. This intuition is clearly expressed in the first sentence of the introduction 
of the well-known book by Mac Lane~\cite{MacLane-1998}

\begin{quote}
Category theory starts with the observation that many properties of 
mathematical systems can be unified and simplified by a presentation 
with diagrams of arrows. Each arrow $f : X \to Y$ represents a function; 
that is, a set $X$, a set $Y$, and a rule $x \mapsto f(x)$ which assigns 
to each element $x \in X$ an element $f(x) \in Y$.
\end{quote}

%\hfill{\small (S. Mac Lane, \emph{Category Theory for Working Mathematicians})}

%\skip
\noindent
Almost the whole of category theory focussed for years on the paradigm that the maps are the counterpart in the category at hand of total functions, and essential use of this was made in forming the definitions. 
%=======
%Almost the whole of category theory was based for years on the paradigm that the maps of the categories discussed represent total functions, and essential use of this is made in forming the %definitions. 
%2) vari approcci successivi all'astrazione di categorie come insiemi e Relazioni: Freyd, Carboni, Rosolini Robinson 
 
Late 1980s witnessed the emergence of a new perspective, aimed at exploring the relational aspects of algebra and logic.
%
During this period, three fundamental works were presented, shaping the development of categories aimed at abstracting the properties of relations: the paper on \emph{cartesian bicategories} by Carboni and Walters~\cite{Carboni_87} and the one on \emph{p-categories} by Robinson and Rosolini~\cite{Robinson88},  and the book on \emph{allegories} by Freyd and Scedrov~\cite{freyd1990categories}.
%=======
%During this period, three fundamental works were presented, shaping the development of categories aimed at abstracting the properties of relations: the work by A. Carboni and R. Walters~\cite{Carboni_87,cartesianbicatII}, where they introduced the notion of \emph{cartesian bicategory}, the paper by E. Robinson and G. Rosolini~\cite{Robinson88}, where they introduced the notion of \emph{p-category} and, finally, the book by P. Freyd and A. Scedrov~\cite{freyd1990categories}, where they introduced the notion of \emph{allegory}. 

The relevance of cartesian bicategories and p-categories in mathematics and computer science has increased in the last years, see for example~\cite{bonchi_seeber_sobocinski_18,Bonchi2017c,Fong19}. Indeed, the crucial intuitions behind such notions and their pervasiveness in many applications has led several 
authors to further develop this kind of structures.

Such a process has therefore led to the development and introduction of new categories, which are very similar in nature but originated in very different contexts and thus have different notations.

Relevant examples include the \emph{categories of information transformers}~\cite{Gol99} and the \emph{CD categories}~\cite{cho_jacobs_2019} and its \emph{affine} variant, which is the basis for a recent approach to categorical probability, where they are dubbed  \emph{Markov categories}~\cite{Fritz_2020} based on the interpretation of arrows as generalised Markov kernels.
 %2b) restriction?

%3)categorie gs, che generalizzano le precedenti
Independently, the notion of \emph{gs-monoidal category} was introduced to present a categorical characterisation of \textit{term graphs}~\cite{gadducci1996}, as well as its 2-categorical counterpart suitable to describe term graph \textit{rewriting}~\cite{CorradiniGadducci97}.
Their study was pursued in a series of papers (see e.g. \cite{CorradiniGadducci99,CorradiniGadducci99b} among others), including their application to the functorial semantics of relational and partial algebras~\cite{CorradiniGadducci02,FritzGCT23}.

% 3,5) dire qualcosa sulle monadi e le kleisli e della loro importanza in categorie e computer science (Moggi)
Another fundamental concept in category theory, which has deep connections with those mentioned above, is the notion of a monad. In particular, the Kleisli category associated with a monad provides a framework for understanding generalised morphisms between objects, capturing both the structure of mappings and the effects described by the monad. In computer science, Kleisli categories have been widely used to model \textit{computations} and \textit{effects}. This connection was first formalised in \cite{Moggi91}, where it was shown how monads provide a rigorous mathematical foundation for these notions.


%4)scopo del nostro lavoro
The purposes of this work is twofold. The first is to arrange and revisit these categorical structures, and possibly their 2-categorical versions in the form of preorder-enriched categories, in a modern and comprehensive way, appropriately comparing them to provide a single reference where they can be analysed.
In fact, the large number of similar notions, presented under different names and in different contexts, makes it increasingly challenging to navigate the literature and to have a clear framework connecting all these kinds of categories.
The second purpose is to conduct a study on the Kleisli categories of suitable monads on such categorical structures, with the aim of further investigating, for all the possible ramifications, the well-known fact that the Kleisli category of a symmetric monoidal monad on a cartesian category is symmetric monoidal, see \cite{kock71bis}. This analysis is also extended to the enriched context.




As a result, we provide an overview of the main categories that abstract the properties of relations, and we show how they are related to each other. To present the various notions, we will use the language of string diagrams~\cite{Selinger2011}, which is widely used today in both mathematics and computer science.

The key notions in this presentation are that of \textit{garbage} and \textit{share} categories, as well as their dual and enriched versions, which we identify as the crucial cores of all the other notions. Our presentation aims to be as modular as we can, in order to highlight the key differences between the various notions analysed, and so that the reader can easily navigate the paper and focus on the specific aspects of interest.
%
This modular approach will bring us to alternative characterisations of some known categories, such as restriction and Markov categories. 

Finally, in this work we provide a taxonomy for Kleisli categories in the context of gs-monoidal categories and to present new examples. Following the philosophy and motivations behind this work, the aim here as well is to showcase, recall, and present the notions of affine and relevant monads  \cite{Kock71,Jacobs1994} (and their enriched version), as well as the characterisations of their Kleisli categories, in the simplest, most general, and modular way possible. 
%The combination of these two notions defines the notion of gs-monoidal monad \cite{FritzGCT23}. 

Examples of this kind of monads arise in a quite natural way when we consider action monads. In this context, the affine or relevant monoidal structure of the monad is determined by the condition that the base category is connected or special respectively. Similar examples can be obtained by taking instances of the semiring monad, proving also examples in the enriched
case.




%5)struttura del lavoro
The paper is structured as follows. In Section~\ref{sec:on gs} we provide a background on the notions of interest, most importantly gs-monoidal categories, 
focussing on their relationships with Markov and restriction categories.  In Section~\ref{sec:taxonomy kleisli} we characterise the structure of Kleisli categories
for symmetric monoidal monads.
In Section~\ref{sec:oplax cartesian categories} we study oplax cartesian categories, the order-enriched version of gs-monoidal categories, and how they are related to cartesian bicategories. Finally, in Section~\ref{sec:conclusions} we draw some conclusions and outline future research directions.


\section{A taxonomy of gs-monoidal categories}


\label{sec:on gs}
The overall, one-dimensional taxonomy of the various ``categories for relations'' we are going to analyse here is represented by the diagram below, where the presence of an arrow 
$X \to Y$ means that a category falling in the class $X$ (e.g. restriction categories with restriction products) also belong to the class $Y$ (categories that are 
either gs-monoidal or with diagonals)

%\begin{figure}[H]
\begin{center}
% https://q.uiver.app/#q=WzAsOCxbMSwwLCJcXHRleHR7U2hhcmV9Il0sWzMsMCwiXFx0ZXh0e0dhcmJhZ2V9Il0sWzIsMSwiXFx0ZXh0e0dzLW1vbm9pZGFsfSJdLFswLDEsIlxcdGV4dHtEaWFnb25hbHN9Il0sWzQsMSwiXFx0ZXh0e1Byb2plY3Rpb25zfSJdLFsxLDIsIlxcdGV4dHtSZXN0cmljdGlvbn0iXSxbMywyLCJcXHRleHR7TWFya292fSJdLFsyLDMsIlxcdGV4dHtDYXJ0ZXNpYW59Il0sWzcsNV0sWzcsNl0sWzUsM10sWzYsNF0sWzQsMV0sWzMsMF0sWzIsMF0sWzIsMV0sWzUsMl0sWzYsMl1d
\small
\begin{tikzcd}
	&& {\text{Sym. Monoidal}} \\
	& {\text{Share}} && {\text{Garbage}} \\
	{\text{Diagonals}} && {\text{Gs-monoidal}} && {\text{Projections}} \\
	& {\text{Restriction R.P.}} && {\text{Markov}} \\
	&& {\text{Cartesian}}
	\arrow[from=2-2, to=1-3]
	\arrow[from=2-4, to=1-3]
	\arrow[from=3-1, to=2-2]
	\arrow[from=3-3, to=2-2]
	\arrow[from=3-3, to=2-4]
	\arrow[from=3-5, to=2-4]
	\arrow[from=4-2, to=3-1]
	\arrow[from=4-2, to=3-3]
	\arrow[from=4-4, to=3-3]
	\arrow[from=4-4, to=3-5]
	\arrow[from=5-3, to=4-2]
	\arrow[from=5-3, to=4-4]
\end{tikzcd}


\end{center}
%\caption{A taxonomy}
%\end{figure}
Monoidal categories were introduced by B\'enabou \cite{benabou_1963} and later finitely axiomatised by Mac Lane in \cite{maclane_1963}, with the term "monoidal category" first appearing in a 1966 paper by Eilenberg and Kelly \cite{closed_categories_1966}. Nowadays, the literature on monoidal categories is very extensive. Hence, for a standard presentation we refer to \cite{MacLane-1998}, while for a complete list of references we refer to the \textit{nlab} page\footnote{\url{https://ncatlab.org/nlab/show/monoidal+category}}.


A monoidal category is a category equipped with a tensor product operation and a unit object, satisfying certain coherence conditions. It provides a framework for studying structures where objects can be combined and interactions are modelled algebraically.
Cartesian categories are one of the leading examples of monoidal categories, where the tensor product is the categorical product.

As shown by Fox in \cite{Fox:CACC},  cartesian categories are precisely symmetric monoidal categories in which every object is equipped with a comonoidal structure given by $\nabla_X: X\to X\times X$ and  $!_X:X\to I$, both natural in $X$. 

By relaxing the requirement about the existence and the naturality of these two families of arrows one obtains a series of well-known categories.

Requiring the existence of both families but not their naturality, one obtains the notion of gs-monoidal category, introduced by Corradini and Gadducci in \cite{gadducci1996,CorradiniGadducci97} in the context of algebraic presentations of graphical formalisms. These categories abstract the properties of cartesian product of sets in the category of relations on one hand, and on the other they have the right structure to distinguish between relations, partial functions, total relations and functions.

Requiring the existence of both families but  naturality only for comultiplication $\nabla_X$, one obtains categories apt to abstract the notion of \textit{partial} function. These categories have been presented in equivalent forms with the names \textit{ restriction categories with restriction products} by Cockett and Lack in \cite{Cockett02,Cockett03,Cockett07}, \textit{p-categories} by Rosolini and Robinson in \cite{Robinson88} and \textit{partial categories} by Curien and Obtulowicz in \cite{CURIEN198950}.
%
Requiring only the existence of comultiplication $\nabla_X$ and its naturality, one obtains the notion of \textit{categories with diagonals}, introduced by Jacobs in \cite{Jacobs1994} in the context of linear logic.
%
The existence of comultiplication $\nabla_X$ without naturality gives rise to the notion of \textit{share categories}.

%On the right side of the diagram, r
Requiring the existence of both families but naturality only for counit $!_X$, one obtains categories apt to abstract the notion of \textit{total} relation. These categories have been presented in equivalent forms with the names \textit{affine CD-categories} by Cho and Jacobs in \cite{cho_jacobs_2019} and Markov categories by Fritz in \cite{Fritz_2020} in the context of categorical probability theory.
%
Requiring only the existence of discharger $!_X$ and its naturality, one obtains the notion of \textit{categories with projections}, introduced by Jacobs in \cite{Jacobs1994} in the context of linear logic.
%
The existence of counit $!_X$ without naturality gives rise to the notion of \textit{garbage categories}.

The above categories are among the most commonly used in the literature of ``categories for relations". However, we will also discuss some minor variants such as their duals, special, connected, Frobenius, and bialgebraic categories.







\iffalse

Originally introduced in the context of algebraic approaches to term graph rewriting~\cite{gadducci1996,CorradiniGadducci97}, the notion of 
\emph{gs-monoidal category} has been 
developed in a series of papers \cite{CorradiniGadducci99, CorradiniGadducci99b, CorradiniGadducci02}.
We recall here the basic definitions adopting the graphical formalism of string diagrams, referring to \cite{Selinger2011} for an overview of various notions of monoidal categories and their associated diagrammatic calculus.
\fi



\subsection{Share categories}
\label{share}
%\newpage
    \begin{mydefinition}
        A \textbf{share category} is a symmetric monoidal category $(\mC, \otimes, I)$ together with
        a commutative cosemigroup structure for each object $X$,
        consisting of a comultiplication 
        %a distinguished map $\freccia{X}{\textrm{copy}_X} {X\otimes X}$, graphically:
        \ctikzfig{copy}
        which is coassociative and cocommutative
        \ctikzfig{comonoid_share_cat}
        These cosemigroup structures must be multiplicative with respect to the monoidal structure, meaning that
        they satisfy the equations
\ctikzfig{comon-struct-mult-share-cat}
\end{mydefinition}

Symbolically, we also write $\nabla_X : X \to X \otimes X$ for the share structure arrow and call it \textbf{duplicator}.

\begin{myexample}\label{ex_Nat_is_share_cat}
	The monoidal category $(\mathcal{N},+,0)$, where $\mathcal{N}$ is the posetal category of natural numbers and the monoidal operation $+$ is given by the usual sum of natural numbers is a share category, where the comultiplication is given by the arrow $n\leq n+n$.
\end{myexample}
\begin{myexample}\label{pfunasshare}
	The leading example of share category is the category of sets and \emph{partial functions}. We will see in Example~\ref{example: due relazioni} how the share structure of this category can be derived from the gs-monoidal structure of the category of sets and relations.
\end{myexample}
\begin{myexample}\label{ex_Set}
The monoidal category $(\mathbf{Set},\times,\{\bullet\})$ of sets and functions with the direct product as the monoidal operator is a share category, where the comultiplication is 
given by the functions $X \to X \times X$ such that $a \mapsto \langle a, a \rangle$. \end{myexample}

%\fab{qualche altro esempio semplice, tipo funzioni e/o relazioni? Set con il prodotto cartesiano lo utilizziamo in effetti dopo.}

The following lemma shows that a share structure can be equivalently given in terms of a monoidal transformation.

\begin{mylemma}\label{rmk:equiv share}
	Given a symmetric monoidal category $(\mC, \otimes,I)$ and the two trivial strong symmetric monoidal functors given by the identity functor
\[(\id, \id_I, \id_{+\otimes -}):\mC \to \mC \]
and the functor $\otimes(-,-):\mC\to\mC$ which sends $X$ to $X\otimes X$
\[(\otimes(-,-), \lambda_I, \id_+\otimes \gamma_{+,-}\otimes \id_{-}):\mC\to \mC\]
a \textbf{share} structure is given by a monoidal transformation
\[\mathrm{copy}_{-}:(\id,\id_I,\id_{+\otimes -})\to (\otimes(-,-), \lambda_I, \id_+\otimes \gamma_{+,-}\otimes \id_{-}).\]
\end{mylemma}

\begin{myremark}
It is now immediate to see that a share category is a \emph{category with diagonals}~\cite[Def. 2.1]{Jacobs1994}
if $\mathrm{copy}_{-}$ is a natural transformation.
\end{myremark}

\begin{mydefinition}\label{def relevant functor}
	For share categories $\mC$ and $\mD$, a functor $\freccia{\mC}{F}{\mD}$ equipped with a lax symmetric monoidal structure
   \[
				\freccia{\otimes \circ \, (F\times F)}{\psi}{F\circ \otimes}, \qquad \freccia{I}{\psi_0}{F(I)} 
			\]
is \textbf{relevant} if the following diagram commutes for all $X$ in $\mC$

\begin{equation}\label{diagram: lax relevant}
\begin{tikzcd}[column sep=tiny]
	{F(X)} && {F(X\otimes X)} \\
	& {F(X)\otimes F(X)}
	\arrow["{F(\nabla_X)}", from=1-1, to=1-3]
	\arrow["{\nabla_{FX}}"', from=1-1, to=2-2]
	\arrow["{\psi_{X,X}}"', from=2-2, to=1-3]
\end{tikzcd}
\end{equation}
\end{mydefinition}


%Share categories, relevant functors and monoidal natural transformations form a 2-category denoted with $\mathcal{S{-}CAT}$.

\begin{mydefinition}
    An arrow $f:X\to Y$ in a share category is called \textbf{copyable} or \textbf{functional} if
    \ctikzfig{functional}
%	The sub-category of functional arrows is denoted by $\mC$-$\mathbf{Fun}$.
\end{mydefinition}

Therefore, the notion category with diagonals~\cite[Def. 2.1]{Jacobs1994} can be easily rephrased in terms of share category.
\begin{mylemma}
A share category has diagonals if and only if
every arrow is functional.
\end{mylemma}
\begin{myexample}
	The share categories $(\mathcal{N},+,0)$  and $(\mathbf{Set},\times,\{\bullet\})$ 
	presented in Example~\ref{ex_Nat_is_share_cat} and Example~\ref{ex_Set} respectively are both categories with diagonals.
\end{myexample}
\begin{mycorollary}
The sub-category $\mC$-$\mathbf{Fun}$ of functional arrows is a category with diagonals.
\end{mycorollary}

We close recalling the notion of positivity from~\cite[Def.~11.22]{Fritz_2020}.

\begin{mydefinition}\label{positive}
    A share category is called \textbf{positive} 
    if for every arrows $f:X\to Y$ and $g: Y \to W$ such that 
    $g \circ f$ is functional then
\ctikzfig{positivity}
\end{mydefinition}

\subsection{Categories with garbage}
\label{garbage}

\begin{mydefinition}
    A \textbf{category with garbage} is a symmetric monoidal category $(\mC, \otimes, I)$ together with
     a distinguished arrow for each object $X$
\ctikzfig{del}
        These arrows must be multiplicative with respect to the monoidal structure, meaning that
        they satisfy the equations
\ctikzfig{axiom_grb_cat}
\end{mydefinition}
Symbolically, we also we write $!_X : X \to I$ for the 
garbage structure arrow and call it \textbf{discharger}.
\begin{myexample}\label{ex_Nat_op_is_garbage_cat}
	The monoidal category $(\mathcal{N}^{\mathrm{op}},+,0)$, where $(\mathcal{N},+,0)$ is the category defined in Example~\ref{ex_Nat_is_share_cat}, is a category with garbage.
\end{myexample}


\begin{myexample}
	The leading example of a category with garbage is the category of sets and \emph{total relations},
	while a leading counterexample is the share category of partial functions of Example~\ref{pfunasshare}
	(and of course vice versa).	
	Also for this case we will see in Example~\ref{example: due relazioni} how the garbage structure of this category can be derived from the gs-monoidal structure of the category of sets and relations.
\end{myexample}

\begin{myexample}\label{ex_Set_G}
The monoidal category $(\mathbf{Set},\times,\{\bullet\})$ of sets and functions with the direct product of Example \ref{ex_Set} as the monoidal operator 
is also a category with garbage, where the arrows are given by the function $a \mapsto \bullet$.
\end{myexample}

%\fab{Anche qua, mini esempi come prima}

The following lemma shows that garbage structure can be equivalently given in terms of a monoidal transformation.

\begin{mylemma}\label{rmk:equiv garbage}
	Given a symmetric monoidal category $(\mC, \otimes,I)$ and the two trivial strong symmetric monoidal functors given by the identity
\[(\id, \id_I, \id_{+\otimes -}):\mC \to \mC \]
and the constant value $I$ functor
\[(I, \id_I, \lambda_I^{-1}):\mC\to \mC\]
a \textbf{garbage} structure is given by a monoidal transformation
\[\mathrm{dis}_{-}:(\id,\id_I,\id_{+\otimes -})\to (I, \id_I, \lambda_I^{-1}).\]
\end{mylemma}

\begin{myremark}
Similarly to what occurs for share categories, a category with garbage is a \emph{category with projections}~\cite[Def. 2.1]{Jacobs1994}
if $\mathrm{dis}_{-}$ is a natural transformation.
%
Notice that categories with projections are also called \emph{semicartesian categories} in \cite{Gerhold2022,Fritz_2020}. %A simple computation shows that semicartesian categories have weak finite products.
\end{myremark}


\begin{mydefinition}\label{def garbage functor}
	For categories with garbage $\mC$ and $\mD$, a functor $\freccia{\mC}{F}{\mD}$ equipped with a lax symmetric monoidal structure
   \[
				\freccia{\otimes \circ \, (F\times F)}{\psi}{F\circ \otimes}, \qquad \freccia{I}{\psi_0}{F(I)} 
			\]
is \textbf{affine}  if the following diagram commutes for all $X$ in $\mC$
\begin{equation}\label{diagram: lax affine}
\begin{tikzcd}[column sep=tiny]
	F(X) && {F(I)} \\
& I
\arrow["{F(!_X)}", from=1-1, to=1-3]
\arrow["{!_{F(X)}}"', from=1-1, to=2-2]
\arrow["{\psi_0}"', from=2-2, to=1-3]
\end{tikzcd}
\end{equation}
\end{mydefinition}



\begin{mydefinition}
	An arrow $f:X\to Y$ in a category with garbage is called \textbf{discardable} or \textbf{total} if 
	 \ctikzfig{full}
%	 The sub-category of total arrows by $\mC$-$\mathbf{Tot}$.
 \end{mydefinition}

\begin{mylemma}
A category with garbage has projections if and only if every arrow is total.
\end{mylemma} 

\begin{myremark}\label{terminal}
Notice that in a garbage category every arrow is total if and only if the object $I$ is terminal.
\end{myremark} 

\begin{mycorollary}
The sub-category $\mC$-$\mathbf{Tot}$ of total arrows is a category with projections.
\end{mycorollary}


\subsection{Gs-monoidal categories}
\label{gs-mon}
The original notion of \emph{gs-monoidal category} introduced in~\cite{gadducci1996,CorradiniGadducci97} can be presented combining the previous notions of share and garbage category.
\begin{mydefinition}\label{dfn: gs-monoidal}
	 A \textbf{gs-monoidal category} is a symmetric monoidal category $(\mC, \otimes, I)$ with share and garbage structures such that for each object $X$
	 \ctikzfig{copy_del_equation}
	\end{mydefinition}

In other words, each object is equipped with a commutative comonoid structure.
%
%Symbolically, we also write $\nabla_X : X \to X \otimes X$ for the share structure arrow and call it \textbf{duplicator}, and similarly we write $!_X : X \to I$ for the 
%garbage structure arrow and call it \textbf{discharger}.
%
We can now close this section with a result that is basically stated in~\cite{Fox:CACC}.

\begin{mylemma}
A gs-monoidal category is cartesian monoidal if and only if $\nabla_{-}$ and $!_{-}$ are natural transformations.
\end{mylemma}

%We can  combine the previous lemmata to obtain the characterisation below.
\begin{mycorollary}\label{cor_TFun_is_cartesian}
The sub-category $\mC$-$\mathbf{TFun}$ of total and functional arrows is cartesian monoidal.
\end{mycorollary}
%\begin{mylemma} If $(\mC, \otimes, I)$ is a gs-monoidal category  with projections then it has weak finite products. 
%	\begin{proof}
%		Since $I$ is terminal, every arrow is discardable. Hence, a weak product of two objects $X$ and $Y$ is given by $X\otimes Y$, with projections $\rho_X\circ (id_X\otimes \mathrm{del}_Y):X\otimes %Y\to Y$ and $\lambda_X\circ (\mathrm{del}_X\otimes id_Y ):X\otimes Y\to X$ where $\rho_X$ and $\lambda_X$ denote respectively the right and the left unitor.
%	\end{proof}
%	\end{mylemma}
%\begin{mydefinition}
  %  A \textbf{gs-monoidal category} is a symmetric monoidal category $(\mC, \otimes, I)$
  %  with a commutative comonoid structure on each object $X$ consisting of a comultiplication
  %  and a counit
  %  \ctikzfig{copy_del}
 %    which satisfy the commutative comonoid equations
    
 %   \ctikzfig{comonoid_equation}
 %   These comonoid structures must be multiplicative with respect to the monoidal structure,
  %  meaning that they satisfy the equations
 %   \ctikzfig{comon-struct-mult}
 %   \end{mydefinition}
%Symbolically, we also write $\nabla_X : X \to X \otimes X$ for the first structure arrow above and call it \textbf{duplicator}, and similarly $!_X : X \to I$ for the 
%\textbf{discharger}.
%
%\begin{mydefinition}
%    A morphism $f:X\to Y$ in a gs-monoidal category is called \textbf{copyable} or \textbf{functional} if
%    \ctikzfig{functional}
%   It is called \textbf{discardable},  \textbf{full} or \textbf{total} if 
%    \ctikzfig{full}
%	The sub-category of functional arrows is denoted by $\mC$-$\mathbf{Fun}$, the one of total arrows by $\mC$-$\mathbf{Tot}$, and the one of functional and total arrows by $\mC$-$\mathbf{TFun}$.
%   \end{mydefinition}
      
   \begin{myexample}\label{example: due relazioni}
   The category $(\mathbf{Set}, \times, \{\bullet\})$ of sets and functions with the direct product is gs-monoidal, and in fact cartesian monoidal.
    The category $(\Rel, \times,\{\bullet\})$ of sets and relations with the composition of $a\subseteq X\times Y$ with $b\subseteq Y\times Z$  given by
	\[b\circ a:=\{(x,z) \mid \exists y\in Y,(x,y)\in a \wedge (y,z)\in b \}\subseteq X\times Z\]
    and the monoidal operation %$\otimes : \Rel \times \Rel\to \Rel$ 
    given by the direct product of sets 
    is the leading example of gs-monoidal category \cite{CorradiniGadducci02}. 
    In this category, the copyable arrows are precisely the partial functions, and the discardable arrows are the total relations.
\end{myexample}
	
\begin{myexample}
Recently, an alternative category of relations $(\Rel^{\forall}, \otimes,\{\bullet\})$ has been investigated~\cite{bonchi2024,bonchi_trotta_2024}. The category $\Rel^{\forall}$ has the same objects and arrows as $\Rel$, but the composition of a relation $a\subseteq X\times Y$ with $b\subseteq Y\times Z$ is given by
	\[b\circ a=\{(x,z) \mid \forall y\in Y,(x,y)\in a \vee (y,z)\in b \}\subseteq X\times Z\]
	and the identity arrow is given by $\id_X=\{\left(x,y\right)| x\neq y\}\subseteq X\times X$.
The tensor product is defined on objects as the direct product of sets (as in the previous case), and on two arrows $a\subseteq X\times Y$ and $c\subseteq Z\times V$
is given by 
\[ a\otimes c= \{\left(\left(x,z\right), \left(y,v\right)\right) |\ \left(x,y\right)\in a \vee \left(z,v\right) \in c   \}\]
	This category is gs-monoidal with the following structure arrows
	\[\nabla_X=\{\left(x,\left( y,z\right) \right)| x\neq y \vee x\neq z\}\subseteq X\times (X \times X)\qquad !_X= \emptyset\subseteq X\times I\]
	Note that $\Rel^{\forall}$ is isomorphic to $\Rel$ via a ``complement functor'', which is strict symmetric monoidal~\cite{bonchi2024}.
	Although we are not aware of a characterisation of copyable relations for this case, it is easy to 
	show that the discardable relations are precisely those $a\subseteq X\times Y$ such that
	$\forall x \in X.\; \{y \in Y \mid (x, y) \in a\} \neq Y$.
	%$\neg \exists x\in X \forall y\in Y a(x,y)$.
%\fab{quale e' la struttura gs? cosa sono le copyable e i discardable?}
    \end{myexample}
     
We close now with an observation from \cite[Prop. 2.10\&2.11]{FritzGCT23}.
% the authors observe some properties of the sub-categories of total and functional arrows.


\begin{mylemma}\label{prop: 2.10 e 2.11 FGTC23}
	Let $(\mC,\otimes, I)$ be a gs-monoidal category. Then $\mC$-$\mathbf{Fun}$, $\mC$-$\mathbf{Tot}$, and $\mC$-$\mathbf{TFun}$ are gs-monoidal sub-categories of $\mC$.
\end{mylemma}

    As for functors between symmetric monoidal categories, also functors between gs-monoidal categories come in several variants. 
   \iffalse 
	\begin{mydefinition}\label{def gs monoidal functor}
        For gs-monoidal categories $\mC$ and $\mD$, a functor $\freccia{\mC}{F}{\mD}$ equipped with a lax symmetric monoidal structure
       \[
                    \freccia{\otimes \circ \, (F\times F)}{\psi}{F\circ \otimes}, \qquad \freccia{I}{\psi_0}{F(I)} 
                \]
is \textbf{lax affine}  if the following diagram commutes for all $X$ in $\mC$
\begin{equation}\label{diagram: lax affine}
	\begin{tikzcd}[column sep=tiny]
		F(X) && {F(I)} \\
    & I
    \arrow["{F(!_X)}", from=1-1, to=1-3]
    \arrow["{!_{FX}}"', from=1-1, to=2-2]
    \arrow["{\psi_0}"', from=2-2, to=1-3]
	\end{tikzcd}
\end{equation}
and it is \textbf{lax relevant} if the following diagram commutes for all $X$ in $\mC$

\begin{equation}\label{diagram: lax relevant}
	\begin{tikzcd}[column sep=tiny]
		{F(X)} && {F(X\otimes X)} \\
		& {F(X)\otimes F(X)}
		\arrow["{F(\nabla_X)}", from=1-1, to=1-3]
		\arrow["{\nabla_{FX}}"', from=1-1, to=2-2]
		\arrow["{\psi_{X,X}}"', from=2-2, to=1-3]
	\end{tikzcd}
\end{equation}
A functor which is both lax affine and lax relevant is \textbf{lax gs-monoidal}.
\end{mydefinition}
\fi
\begin{mydefinition}\label{def gs monoidal functor}
	For gs-monoidal categories $\mC$ and $\mD$, a functor $\freccia{\mC}{F}{\mD}$ equipped with a lax symmetric monoidal structure
   \[
				\freccia{\otimes \circ \, (F\times F)}{\psi}{F\circ \otimes}, \qquad \freccia{I}{\psi_0}{F(I)} 
			\]
is \textbf{gs-monoidal} if it is both relevant and affine.
\end{mydefinition}

\subsection{Recurring examples: spans and (weighted) relations}\label{ex_spans_are_gs}\label{example: span + gs monoidale}

 Recall from \cite{Bruni2003} that the category $\mathbf{Span}(\mA)$ of \emph{spans} associated with a category with finite limits  $\mA$ is a gs-monoidal category. The $\mathbf{Span}(\mA)$ has the same objects 
	as $\mA$, and an arrow from $X$ to $Y$ is a \emph{span}, \textit{i.e.}\ an equivalence class of diagrams of the form
		 $(X  \xleftarrow{a_X} A \xrightarrow{a_Y} Y)$ of $\mA$, where $(A,a_X,a_Y)\sim (B,b_X,b_Y)$ if there exists an isomorphism $i:A\to B$ such that $b_X\circ i=a_X$ and $b_Y\circ i=a_Y$. Identities and composition are defined as follows
	\begin{itemize}
		\item the identity of $X$ is the span $X \xleftarrow{\id_X}X  \xrightarrow{\id_X} X$;
		\item the composition of spans $X\leftarrow  A \xrightarrow{f} Y$ and $Y\xleftarrow{g}  B \rightarrow Z$ is given by the span $X\leftarrow A\times_{Y}B \to Z$ obtained through the pullback of $f$ and $g$
		\[% https://q.uiver.app/#q=WzAsNixbMiwwLCJBXFx0aW1lc19ZQiJdLFsxLDEsIkEiXSxbMiwyLCJZIl0sWzMsMSwiQiJdLFswLDIsIlgiXSxbNCwyLCJaIl0sWzEsMiwiZiJdLFszLDIsImciLDJdLFswLDFdLFswLDNdLFswLDIsIiIsMSx7InN0eWxlIjp7Im5hbWUiOiJjb3JuZXIifX1dLFsxLDRdLFszLDVdXQ==
		\begin{tikzcd}[row sep=small]
			&& {A\times_YB} \\
			& A && B \\
			X && Y && Z
			\arrow[from=1-3, to=2-2]
			\arrow[from=1-3, to=2-4]
			\arrow["\lrcorner"{anchor=center, pos=0.125, rotate=-45}, draw=none, from=1-3, to=3-3]
			\arrow[from=2-2, to=3-1]
			\arrow["f", from=2-2, to=3-3]
			\arrow["g"', from=2-4, to=3-3]
			\arrow[from=2-4, to=3-5]
		\end{tikzcd}\]
	\end{itemize}
	The tensor product is given via the categorical product $\times$ of $\mathcal{A}$ and the gs-monoidal structure is given by the arrows
	\[
	\nabla_X = (X \xleftarrow{\id} X \xrightarrow{\nabla_X} X \times X), \qquad !_X = (X \xleftarrow{\id} X \xrightarrow{!_X} 1).
\]
Although the duplicator  $\nabla_X$ and the discharger $!_X$ are natural in $\mathcal{A}$, in general they are not in $\mathbf{Span}(\mA)$ as observed first in \cite{Bruni2003} and, in an equivalent way, in \cite[Ex. 2.1.4(10)]{Cockett02} where it is observed that $\mathbf{Span}(\mA)$ is not a restriction category (see also Proposition \ref{gs as restriction}). We will refer to this gs-monoidal category as $(\mathbf{Span}(\mathcal{A}),\times,1)$. 
%
%On a similar note, also the category of \emph{cospans} $\mathbf{CoSpan}(\mA)$ associated with a category with finite colimits  $\mA$ is a gs-monoidal category. 

\begin{myremark}
It is worth to observe that the sub-category $\mathbf{Span}_m(\mA)$ of  $\mathbf{Span}(\mA)$  whose arrows are spans whose left leg is a mono is gs-monoidal as well (since monos are stable under pullbacks) and in fact it has diagonals. 
This category is a particular instance of a \emph{category of partial maps} $\mathbf{Par}(\mA, \mathcal{M})$ associated with a stable system of monics $\mathcal{M}$ presented in \cite[Sec.~3.1]{Cockett02}.


Moreover, the sub-category $\mathbf{Span}_e(\mA)$ of $\mathbf{Span}(\mathcal{A})$ 
of spans whose left leg is a split epimorphism\footnote{An arrow $f:A\to B$ splits if it has a \textit{section}, i.e.\ an arrow $s:B\to A$ such that $f\circ s=\id_B$.} 
has projections, see \cite[Prop.~5.4]{FritzGCT23}.

%Hence, the sub-category $\mathbf{Span}_i(\mA)$ of $\mathbf{Span}(\mathcal{A})$
%of spans whose left leg is an isomorphism is cartesian.
\end{myremark}

%\begin{myremark}\label{example: span + gs monoidale}
	If $\mathcal{A}$ is an extensive category  (see \cite{carboni1993introduction}) with finite limits, then the category $\mathbf{Span}(\mA)$
	has another gs-monoidal structure. The tensor product is given by the categorical sum $+$ and gs-monoidal structure is given by the arrows
	\[
	\nabla_X = (X \xleftarrow{(\id, \id)} X+X \xrightarrow{\id} X + X), \qquad !_X = (X \xleftarrow{\cobang} 0 \xrightarrow{\id} 0).
\]
This category is actually cartesian and we will refer to it as $(\mathbf{Span}(\mA),+,0)$.

\iffalse
\fab{si puo; ridire anche velocemente qualcosa di analogo a $\mathbf{Span}_m(\mA)$ e $\mathbf{Span}_e(\mA)$?}
\cipr{per $Span_e$ il delete è qualcosa di banale perché l'iniziale è stretto, quindi la ganba di sx sarebbe un iso con lo 0. Anzi si potrebbe quasi dire che non c'è del. quindi è solo share? 
Similmente per $Span_m$ succede che $\nabla_X$ non è una mappa. Quindi sembra essere solo garbage?}
\fi
%\end{myremark}



\subsubsection{From spans to relations}
\label{example: relazioni x sono gs monoidale }
	It is well-known that the notion of category of relations $\Rel$ of Example \ref{example: due relazioni} can be generalised to regular categories (see \cite[Ex. 1.4]{CARBONI198711}) or, more generally, to categories equipped with a proper, stable factorization system (see \cite{Kelly_92}). Indeed, let $\mathcal{A}$ be a regular category, then consider the category of relation $\mathbf{Rel}(\mathcal{A})$
	whose objects are those of $\mathcal{A}$ and arrows are given by jointly monic spans. The composition of relations $X\xleftarrow{f_X}  A \xrightarrow{f_Y} Y$ and $Y\xleftarrow{g_Y}  B \xrightarrow{g_Z} Z$ is given by first considering the span $X\leftarrow A\times_{Y}B \to Z$ obtained by taking the pullback of $f$ and $g$,
	and then considering the regular-epi/mono factorization of the induced arrow $\langle f_X\circ g_Y',g_Z\circ f_Y'\rangle:A\times_{Y}B\to X\times Z$.
	\[% https://q.uiver.app/#q=WzAsNyxbMCwyLCJYIl0sWzIsMiwiWSJdLFs0LDIsIloiXSxbMSwxLCJBIl0sWzMsMSwiQiJdLFsyLDEsIkFcXGNpcmMgQiJdLFsyLDAsIkFcXHRpbWVzX3tZfUIiXSxbMywwXSxbMywxLCJmIiwyXSxbNCwxLCJnIl0sWzQsMl0sWzUsMCwiIiwxLHsic3R5bGUiOnsiYm9keSI6eyJuYW1lIjoiZGFzaGVkIn19fV0sWzUsMiwiIiwxLHsic3R5bGUiOnsiYm9keSI6eyJuYW1lIjoiZGFzaGVkIn19fV0sWzYsMywiZyciLDJdLFs2LDQsImYnIl0sWzYsNSwiIiwxLHsic3R5bGUiOnsiYm9keSI6eyJuYW1lIjoiZGFzaGVkIn0sImhlYWQiOnsibmFtZSI6ImVwaSJ9fX1dXQ==
	\begin{tikzcd}
		&& {A\times_{Y}B} \\
		& A & {A\circ B} & B \\
		X && Y && Z
		\arrow["{g_Y'}"', from=1-3, to=2-2]
		\arrow[dashed, two heads, from=1-3, to=2-3]
		\arrow["{f_Y'}", from=1-3, to=2-4]
		\arrow[from=2-2, to=3-1]
		\arrow["f_Y"', from=2-2, to=3-3]
		\arrow[dashed, from=2-3, to=3-1]
		\arrow[dashed, from=2-3, to=3-5]
		\arrow["g_Y", from=2-4, to=3-3]
		\arrow[from=2-4, to=3-5]
	\end{tikzcd}\]
	 The categorical product $\times$ induces a gs-monoidal structure as in $( \mathbf{Span}(\mathcal{A}),\times,1)$, and this gs-monoidal category will be denoted as $(\mathbf{Rel}(\mathcal{A}), \times,1)$.
	From a logical perspective, the main reason why the $\mathbf{Rel}$-construction can be generalised in this setting is that regular categories are able to properly deal with the $(\exists,=,\wedge,\top)$-fragment of first-order logic (the \emph{regular fragment}). In particular, they have an ``internal'' notion of existential quantifier, which allows us to mimic the usual composition of relations (which is defined via the existential quantifier).
	Similarly, if $\mathcal{A}$ is also extensive, then the coproduct induces a cartesian monoidal structure as in $(\mathbf{Span}(\mathcal{A}),+,0)$, which
	we will denote as $(\mathbf{Rel}(\mathcal{A}),+,0)$.
%	On the same line, in recent works \cite{bonchi2021doctrines,bonchi_trotta_2024} the $\mathbf{Rel}$-construction has been considered in the more general context of elementary and existential doctrines, playing a crucial role in the characterization of the internal logic of cartesian and first-order bicategories.
%\end{myexample}
\begin{myremark}\label{spas as rel}
%We observe that, f
For a regular category $\mathcal{A}$ in which every regular epi splits, the category of relations  is equivalent to a suitable category of spans.
Namely, $\mathbf{Rel}(\mathcal{A})$ is equivalent to the category whose objects are those of $\mathcal{A}$ and whose arrows are equivalence classes of diagrams of the form
$(X  \xleftarrow{a_X} A \xrightarrow{a_Y} Y)$ of $\mA$, where $(A,a_X,a_Y)\sim (B,b_X,b_Y)$ if there exist two arrows $h:A\to B$ and $k:B\to A$ such that 
$$\begin{cases}
	b_X\circ h=a_X \\
	b_Y\circ h=a_Y
\end{cases} \qquad \begin{cases}
	a_X\circ k=b_X \\
	a_Y\circ k=b_Y
\end{cases} $$
Indeed, every span $(X  \xleftarrow{a_X} A \xrightarrow{a_Y} Y)$ is in the same equivalence class of $(X  \xleftarrow{m_X} M \xrightarrow{m_Y} Y)$, where $m:=\angbr{m_X}{m_Y}$ is the image of the factorization of $\angbr{a_X}{a_Y}$
\[% https://q.uiver.app/#q=WzAsMyxbMCwwLCJBIl0sWzEsMSwiWFxcdGltZXMgWSJdLFsyLDAsIk0iXSxbMCwyLCJlIiwyLHsic3R5bGUiOnsiaGVhZCI6eyJuYW1lIjoiZXBpIn19fV0sWzIsMSwibSIsMCx7InN0eWxlIjp7InRhaWwiOnsibmFtZSI6Im1vbm8ifX19XSxbMCwxXSxbMiwwLCJzIiwyLHsiY3VydmUiOjJ9XV0=
\begin{tikzcd}
	A && M \\
	& {X\times Y}
	\arrow["e"', two heads, from=1-1, to=1-3]
	\arrow["{\langle a_X, a_Y\rangle}"',from=1-1, to=2-2]
	\arrow["s"', curve={height=12pt}, from=1-3, to=1-1]
	\arrow["m", tail, from=1-3, to=2-2]
\end{tikzcd}\]
In particular, the above observation holds for $\mathcal{A}=\mathbf{Set}$, where this kind of spans are referred to 
as \textit{garbage equivalent} ones \cite{Gadducci08}.
\end{myremark}

\subsubsection{And now, weighted relations}
\label{WR}\label{example: monad semiring}
While the previous section generalised relations by changing the base category, an alternative approach
is to equip each relation with a \emph{weight}.
%
%\fab{scrivere sulla differenza con Rel(A)}
Consider a semiring $(M,\oplus,\odot, 0, 1)$. It is easy to check that 
there exists a functor $\mathcal{M}:\mathbf{Set}\to\mathbf{Set}$ that sends every set $X$ to 
 \[\mathcal{M}(X)=\left\{ h:X \to M \ |\  h\  \text{has finite support}\right\}\]
where \emph{finite support} means that $h(x)\neq 0$  for a finite number of elements $x\in X$, and every function $f:X\to Y$ to the function $\tilde{f}:\mathcal{M}(X)\to \mathcal{M}(Y)$ which sends every $M$-valued function $h:X\to M$ with finite support 
to 
\[\tilde{f}(h)(y)= \underset{x\in f^{-1}(y)}{\bigoplus} h(x)\]
Explicitly, the image of the composition $ g \circ f$ of $f:X\to Y$ and $g:Y\to Z$ is
\[\widetilde{g \circ f}(h)(z)= \tilde{g}(\underset{x\in f^{-1}(y)}{\bigoplus} h(x)) = \underset{y\in g^{-1}(z)}{\bigoplus}(\underset{x\in f^{-1}(y)}{\bigoplus} h(x))\]


Recall that $(\mathbf{Set}, \times, \{\bullet\})$ is cartesian monoidal with respect to the direct product.
The above functor is lax symmetric monoidal with respect to that monoidal structure, with the obvious 
coherence arrows 
\[\psi_{X,Y}:\mathcal{M}(X)\times \mathcal{M}(Y)\to \mathcal{M}(X\times Y)
\qquad\psi_0:\{\bullet\}\to \mathcal{M}(\{\bullet\})\]
given by $\psi_{X,Y}(h,k)(x,y)=h(x)\odot k(y)$ and $\psi_0(\bullet)(\bullet)=1$.


Finally, note that there are two additional endofunctors $\mathcal{M}_e$ and $\mathcal{M}_u$ on $\mathbf{Set}$ 
that are both lax symmetric monoidal with respect to the same monoidal structure $(\mathbf{Set}, \times, \{\bullet\})$. These are defined as follows
% \[\mathcal{M}_e(X):=\left\{ h:X \to M \ |\  h\  \text{has support at most 1 and } 1 \wedge \underset{x\in X}{\bigoplus}h(x) \text{ is idempotent}\right\}\]
%\[\mathcal{M}_u(X):=\left\{ h:X \to M \ |\  h\  \text{has finite support} \wedge \underset{x\in X}{\bigoplus}h(x) = 1\right\}\]
\[\mathcal{M}_e(X)=\left\{ h:X \to M \ |\  h\  \text{has support at most one and is idempotent}\right\}\]
\[\mathcal{M}_u(X)=\left\{ h:X \to M \ |\  h\  \text{has finite support and is normalised}\right\}\]
where idempotent means that $\forall x \in X.\, h(x) = h(x) \odot h(x)$ and normalised that $\bigoplus_{x \in X}h(x) = 1$.
%
What is noteworthy is that, besides being both lax symmetric monoidal functors, $\mathcal{M}_e$ is relevant and $\mathcal{M}_u$ is affine.

\begin{myremark}\label{power}
The setting above is general enough to recover various relational structures defined in the literature.
	If e.g. $M$ is the Boolean semiring $\{0,1\}$, then $\mathcal{M}$ is the lax symmetric monoidal 
	functor $\mathcal{P}$ associating to $X$ its finite subsets, and it is neither relevant nor affine.
	The relevant functor $\mathcal{P}_e$ is restricted to subsets of at most one element,
	while the affine functor $\mathcal{P}_u$ is restricted to subsets with at least one element. 
%\fab{parte di questo remark va dopo che abbiamo introdotto le monadi}
\end{myremark}

%Finally, note that there exists another functor $\mathcal{M}^{\forall}$ on $\mathbf{Set}$, which behaves as $\mathcal{M}$ on sets, yet \ldots
%
%\iffalse
%\cipr{La composizione di cosa? delle relazioni pesate? Se così allora ci chiediamo la composizione di frecce
%in $Kl_{\mathcal{M}}$. 
%Un conto veloce mi sembra dare questo risultato. Se $f:X\to Y$ e $Y\to Z$ sono frecce in $KL_{\mathcal{M}}$
%allora queste corrispondono a funzioni $X\to \mathcal{M}(Y)$ e $Y\to \mathcal{M}(Z)$ tra insiemi che a loro volta corrispondono
%a funzioni $f':X\times Y\to M$ e $g'Y\times Z\to M$. Velocemente il risultato 
%della composizione in $Kl_{\mathcal{M}}$ mi sembra 
%\[(g\circ^\sharp f)(x,z)= \underset{y\in Y}{\bigoplus} {f'(x,y)g'(y,z)}\]
%}
%\fi
%
%\cipr{DOMANDA. se le relazioni (pesate) con questa composizione sono una Kleisli. Le relazioni pesate con l'altra composizione sono una kleisli?
%Il funtore $\mathcal{M}^\forall:\mathbf{Set}\to \mathbf{Set}$
%definito sulle frecce
%\[\tilde{f}(h)(y):= \underset{x\in f^{-1}(y)}{\bigotimes} h(x)\]
%NON mi sembra essere una monade
%%\begin{itemize}
%%	\item è una monade? ($\eta$ definita allo stesso modo e $\mu$ definita con il prodotto)
%%	\item la composizione di frecce nella Kleisli è quella che scrivo sotto?
%%\end{itemize}
%
%
%Inoltre, la composizione delle relazioni con il $\forall$
%si srive in modo algebrico così
%\[(g\circ f)(x,z)= \underset{y\in Y}{\bigotimes} ({f'(x,y)+g'(y,z)})\]
%Questo NON soddisfa $id\circ f=f$ se il monoide è generico
%}
%
%
%
\iffalse
%CONTO CHE CI DICE CHE è RELEVANT SE HA SUPPORTO AL PIù UNO ED è IDEMPOTENTE
\cipr{Non fissiamo ancora $\mathcal{M}_e:\mathbf{Set}\to \mathbf{Set}$ ma assumiamo semplicemente siano
una restriione di $\mathcal{M}(X)$= supporto finito. Vediamo cosa vuol dire relevant in questo caso. 
% https://q.uiver.app/#q=WzAsMyxbMCwwLCJcXG1hdGhjYWx7TX1fZShYKSJdLFsxLDEsIlxcbWF0aGNhbHtNfV9lKFgpXFx0aW1lcyBcXG1hdGhjYWx7TX1fZShYKSJdLFsyLDAsIlxcbWF0aGNhbHtNfV9lKFhcXHRpbWVzIFgpIl0sWzAsMiwiXFxtYXRoY2Fse019X2UoXFxuYWJsYV9YKSJdLFswLDEsIlxcbmFibGFfe1xcbWF0aGNhbHtNfV9lKFgpfSIsMl0sWzEsMiwiY197WCxYfSIsMl1d
}
\[\begin{tikzcd}
	{\mathcal{M}_e(X)} && {\mathcal{M}_e(X\times X)} \\
	& {\mathcal{M}_e(X)\times \mathcal{M}_e(X)}
	\arrow["{\mathcal{M}_e(\nabla_X)}", from=1-1, to=1-3]
	\arrow["{\nabla_{\mathcal{M}_e(X)}}"', from=1-1, to=2-2]
	\arrow["{c_{X,X}}"', from=2-2, to=1-3]
\end{tikzcd}\]
\cipr{adesso, se non ho fatto male i conti $c_{X,X}(h,k)(x_1,x_2)=h(x_1)h(x_2)$ (lo avevo scritto e forse è meglio tenerlo), mentre
\(\mathcal{M}_e(\nabla_X)(h)(x_1,x_2)=\begin{cases}
		h(x_1)\ \text{if}\ x_1=x_2\\
		0\ \text{otherwise}
	\end{cases}\)  quindi dobbiamo chiederci quando 
	\[\mathcal{M}_e(\nabla_X)(h)=c_{X,X}(h,h)\]
	puntualmente su $(x_1,x_2)$
	\[\mathcal{M}_e(\nabla_X)(h)(x_1,x_2)=h(x_1)h(x_2)\]
	a me sembra che sia verificato solo se $\mathcal{M}_e(X)=\{\delta_x| x\in X\} \cup \{h(x)=0\}$
	Se così fosse, è un esempio utile?	(per $\delta$ intendo Dirac)
	}

	\fi

%\fab{riguardare in generale, in particolare se invece di =\ 0 ci vuole support non vuoto

%\begin{myexample}\label{example: monadi affini e relevant}
%%	Recall that $\mathbf{Set}$ is cartesian monoidal, hence gs-monoidal, with respect to the cartesian product.
%%	
%	The powerset monad on the category of sets $\mathcal{P}:\mathbf{Set}\to \mathbf{Set}$ is symmetric monoidal but neither relevant nor affine.
%	The monad becomes affine if restricted to non-empty subsets $\mathcal{P}^+:\mathbf{Set}\to \mathbf{Set}$. While the subsets with at most one element 
%$\mathcal{P}^-:\mathbf{Set}\to \mathbf{Set}$ provide a relevant monad.
%\end{myexample}

%\fab{dire che sopra vale $\mathbf{Set}$ e' cartesian monoidal}

%\fab{dire in un remark diverso cosa succede se il supporto e' non-vuoto, o se la somma dei supporti e' 1, visto che le utilizziamo dopo...}

%\begin{myexample}
%con diversa composizione, in modo da caratterizzare l'altra definizione di Remark~\ref{logics}.
%\end{myexample}

\iffalse
\cipr{%Ho aggiunto il seguente esempio che mi sembra valere. A voi torna? in fin dei conti copy e del sono j.monic e quindi mi sembra tornare tutto come in span. Inotre se due frecce sono mono, mi sembra che anche la loro somma lo sia. Questo serve per dire che + definisce un prodotto.
HO OSCURATO L'ESEMPIO PRECEDENTEMENTE MESSO. NON MI TORNA IL FATTO CHE LA SOMMA PRESERVA COPPIE DI FRECCE J.MONIC }

\begin{myexample}\label{example: rel è gs con +}
 If $\mathcal{A}$ is a regular and extensive category, then the structure in Example \ref{example: span + gs monoidale} gives another gs-monoidal structure on $\mathbf{Rel}(\mathcal{A})$ that we denote with $(\mathbf{Rel}(\mathcal{A}),+,0)$.
\end{myexample}
\fi 

\subsection{On duality}
All the definitions above about the gs-monoidal structure can be easily dualised.

\begin{mydefinition}
    A \textbf{cogs-monoidal category} is a symmetric monoidal category $(\mC, \otimes, I)$
    such that its dual category $\mC^{op}$ is a gs-monoidal category.
\end{mydefinition}
    
In words, each object $X$ is equipped with a monoid structure
$\mathrm{match}_X : X \otimes X \to X $ and $\mathrm{new}_X: I \to X$, namely there exists two arrows
\ctikzfig{co_copy_co_del}
satisfying the obvious axioms. Symbolically, we also write $\Delta_X : X \otimes X \to X$ for the arrow $\mathrm{match}_X$ and  $\cobang_X : I \to X$ for the 
arrow $\mathrm{new}_X$.

\begin{myremark}
We assume, without repeating the statements, that all the remarks in the previous sections concerning sharing and garbage can be dualised, so that e.g. 
$\Delta_X : X \otimes X \to X $ and $\cobang_X: I \to X$ can be described as suitable monoidal transformations, and that their naturality boils down to require 
that a cogs-monoidal category is actually cocartesian monoidal. See also Appendix~\ref{section: appendix dual results}.
\end{myremark}


\begin{myexample}\label{example: span x + are also cogs}
	The categories of spans introduced in Section \ref{ex_spans_are_gs} %and \ref{example: span + gs monoidale} 
	are also cogs-monoidal. Indeed,
	if $\mathcal{A}$ has finite limits, $(\mathbf{Span}(\mathcal{A}),\times,1)$ has a cogs-monoidal structure given by the arrows
	\[\Delta_X  = (X\times X \xleftarrow{\nabla_X} X \xrightarrow{\id} X ), \qquad \cobang_X = (1 \xleftarrow{!} X \xrightarrow{\id} X).
\]
If $\mathcal{A}$ is also an extensive category, $(\mathbf{Span}(\mathcal{A}),+,0)$ has cogs-monoidal structure 
(and it is in fact cocartesian) given by the arrows
\[	\Delta_X = (X+X \xleftarrow{\id} X+X \xrightarrow{(\id,\id)} X ), \qquad \cobang_X = ( 0\xleftarrow{\id} 0 \xrightarrow{\cobang} X).
\]
The same considerations apply for the category $\mathbf{Rel}(\mathcal{A})$ and the gs-monoidal structures presented in Examples \ref{example: relazioni x sono gs monoidale }. %and \ref{example: rel è gs con +} in the respective cases.
\end{myexample}

\begin{mydefinition}
    A \textbf{bigs-monoidal category} is a symmetric monoidal category $(\mC, \otimes, I)$ that is 
    both a gs-monoidal and a cogs-monoidal  category.
\end{mydefinition}

%Hence, in a bigs-monoidal category each object is actually a bimonoid. 
Notice that the definition of a bigs-monoidal category does not include any requirements regarding the interaction between the monoidal and the comonoidal structures. The rest of this section is devoted to studying some relevant axioms that establish the possible ways these structures interact with each other.

\begin{mydefinition}\label{dfn: specal and connected bimonoid}
	A bigs-monoidal category $(\mathcal{C},\otimes,I)$ is \textbf{special} if the law below on the left holds, while
	it is \textbf{connected} if the law below on the right holds\footnote{ See \text{https://ncatlab.org/nlab/show/bimonoid}.}
	\ctikzfig{actionmonadexamplestrict}
\end{mydefinition}

%\begin{myexample}
%
%\label{SpanSpecial}
%	Looking at Example \ref{example: span x + are also cogs}, we have that the category  of spans 
%	$(\mathbf{Span}(\mathcal{A}),\times,1)$ 
%	and $(\mathbf{Span}(\mathcal{A}), +,0)$ are both 
%	is a special bigs-monoidal category, but it is not connected. The same holds for the category of relations $(\mathbf{Rel}(\mathcal{A}),\times,1)$.
%	yet neither of them is connected.
%\end{myexample}

We will further elaborate on bigs-monoidality later in this paper, and we start by presenting two straightforward instances.

\begin{mydefinition}
    A \textbf{bialgebraic category} is a bigs-monoidal category $(\mC, \otimes, I)$ such that the following equalities hold
    \ctikzfig{cipr1eq}
	\ctikzfig{cipr3eq}
\end{mydefinition}

\begin{myremark}
Bialgebraic categories have been investigated in the flownomial calculus
for flowchart description proposed by G. \c{S}tef\u{a}nescu, see e.g. the axioms for angelic branching in \cite{NAGheorghe}
and the references therein.
\end{myremark}
 
Hence, in a bialgebraic category each object is a (bicommutative) bimonoid.

%\begin{mydefinition}\label{dfn: specal and connected bimonoid}
%	A bialgebraic category $(\mathcal{C},\otimes,I)$ is \textbf{special} if the law below on the left holds, while
%	it is \textbf{connected} if the law below on the right holds\footnote{ See \text{https://ncatlab.org/nlab/show/bimonoid}.}
%	\ctikzfig{actionmonadexamplestrict}
%\end{mydefinition}

\begin{myremark}
\label{isoToI}
If a bialgebraic category satisfies the law
\ctikzfig{hopf}
 then each object carries the structure of an Hopf algebra, where the antipode for an object $X$ is given by 
 $\id_X$\footnote{See \url{https://ncatlab.org/nlab/show/Hopf+algebra}.}.
Note that this property does not imply that such a category is special or,
in this case equivalently, connected.
In fact, if a bialgebraic category is connected then 
both $\cobang_X\circ !_X= \id_X$ and 
$!_X\circ \cobang_X= \id_I$ hold, 
that is, each object is isomorphic to $I$.
\end{myremark}

\begin{myexample}\label{example: bialgebra}
	Looking again at Example \ref{example: span x + are also cogs}, we have that the bigs-monoidal 
	categories of spans $(\mathbf{Span}(\mathcal{A}), +,0)$ and relations $(\mathbf{Rel}(\mathcal{A}), +,0)$
	are bialgebraic categories, and in fact they are bicartesian (i.e. both cartesian and cocartesian) monoidal.
	Neither is connected, but the latter is special, the former is not.
	% yet the former is neither special nor connected, while the latter is special but not connected
\end{myexample}

\begin{mydefinition}
	A \textbf{Frobenius category} is a bigs-monoidal category $(\mC, \otimes, I)$ such that the Frobenius law holds, i.e.
	\ctikzfig{frobeniuscorta}
\end{mydefinition}

\begin{myremark}
	The Frobenius law implies~\cite{pastro2009weak} 
	\ctikzfig{frobeniusconsequence}
	and if moreover an object is special~\cite{Bruni2003}
	\ctikzfig{frobeniusconsequence2}
\end{myremark}

\begin{myexample}\label{example: frobenius span}
	Looking once more at Example \ref{example: span x + are also cogs}, we have that the bigs-monoidal 
	categories of spans $(\mathbf{Span}(\mathcal{A}), \times,1)$ and relations $(\mathbf{Rel}(\mathcal{A}), \times,1)$ are Frobenius categories.
	Neither is connected, but both are special.
\end{myexample}

\begin{myremark}
Frobenius categories are also compact-closed (sometimes called 
rigid symmetric monoidal categories), where the dual of an object $X$ is 
the object itself and the unit and counit arrows are defined as $\nabla_A \circ \cobang_A: I \to A \otimes A$ and $\Delta_A \circ !_A: A \otimes A \to I$,
respectively, while the triangle identities trivially hold. 
\end{myremark}

\begin{myremark}
	The Frobenius law is incompatible with the bialgebraic structure: A category that is both Frobenius and bialgebraic
	is also connected, thus 
	each object $X$ 
	%carrying both structures 
	is isomorphic to $I$ (see Remark~\ref{isoToI}). Indeed, we have 
	\ctikzfig{frob_bialg_banale}
where the first equality is obtained through the bigs-monoidal structure, the second through the Frobenius law and the last two
through bialgebra equalities. 
%While $!_X\circ \cobang_X\ = \id_I$ follows from bialgebra structure.
\end{myremark}

%\fab{e anche qua un esempio da Rel o Span, anche se credo che (span,x) non si presti, e dovrebbe essere (span,+)}

%%%%%%%%%%%%%%%%new section%%%%%%%%%%%%%%%%%%%%%%%%%%%%

\subsection{On Markov and restriction categories}

Two categorical notions have come to the forefront in recent years: Markov categories~\cite{Fritz_2020} as models for probabilistic systems, and
restriction categories~\cite{Cockett02,Cockett03,Cockett07} as abstraction of partial functions. This section establishes their correspondence with the notions introduced 
in Sections~\ref{share},~\ref{garbage}, and~\ref{gs-mon}.

\begin{myproposition}\label{gs as restriction}
	Gs-monoidal categories with diagonals  correspond exactly to restriction categories with restriction products.
\end{myproposition}

\begin{proof}
Moving from gs-monoidal categories to restriction categories, the core of the proof occurs in \cite[Rem. 2.15]{FritzGCT23}, where gs-monoidal categories are shown to have enough structure to define the 
\textit{domain} of an arrow $f:X\to Y$ as
	\ctikzfig{domain}
Then, the axioms ($R.2$) and ($R.3$) of Definition~\ref{dfn: restriction category completa}
actually hold for any gs-monoidal category, while ($R.1$) and ($R.4$) hold for the presence of diagonals.
The proof that they have restriction products is a straightforward check. We refer to \cite{FritzGCT23,Cockett07} and the references therein for all the details.
\end{proof}

%\begin{myexample}\label{example: span x con left mono e' restriction}
%	The sub-category $\mathbf{Span}_m(\mathcal{A})$ of the gs-monoidal category $\langle \mathbf{Span}(\mathcal{A}),\times\rangle$, introduced in Example \ref{ex_spans_are_gs},
%	of spans whose left leg is a monomorphism has diagonals.
%\end{myexample}


\begin{myremark}
	Notice that the previous proposition can be dualised, providing the characterisation of \emph{corestriction categories}, see \cite[Ex. 2.1.3(12)]{Cockett02}, in terms of cogs-monoidal categories.
\end{myremark}
\begin{myremark}
An alternative proof of Proposition~\ref{gs as restriction} can be already found in~\cite[Ex. 2.1.3(6-7)]{Cockett02} and in \cite[Sec. 4]{Cockett07}.
Note that restriction categories with restriction products have been proved equivalent to several other notions, such as  p-categories with one-element object~\cite{Robinson88}, 
partial cartesian categories in the sense of Curien and Obtulowicz \cite{CURIEN198950}, and they are a special case of the bicategories of partial maps of Carboni~\cite{Carboni_87}.
See also Proposition~\ref{oplax as restriction} later.
\end{myremark}



\begin{myproposition}
	Gs-monoidal categories with projections  correspond exactly to Markov categories.
\end{myproposition}

\begin{proof}
Straightforward, since the presence of projections boils down to the unit $I$ being the terminal object (see Remark~\ref{terminal}), which
in fact is a precise characterisation of Markov categories.\footnote{A complete analysis of Markov categories and their equivalent presentations, together with relevant references, can be found in \url{https://ncatlab.org/nlab/show/Markov+category}.}
\end{proof}




\iffalse
We recall here the notion of \emph{p-category} from \cite{Robinson88}:

\begin{mydefinition}\label{dfn: p-category}
	A \textbf{p-category} is a category $\mC$ endowed with a functor $-\times -:\mC\times \mC\to \mC$, a natural family of arrows $\Delta:X\to X\times X$, and families $p_{X,Y}:X\times Y\to X$ natural in $X$, and $q_{X,Y}:X\times Y\to Y$ natural in $Y$, such that the following diagrams comute 
\[% https://q.uiver.app/#q=WzAsMTUsWzEsMCwiWCJdLFsxLDEsIlhcXHRpbWVzIFgiXSxbMCwxLCJYIl0sWzIsMSwiWCJdLFs0LDAsIlhcXHRpbWVzIFkiXSxbNCwxLCJYXFx0aW1lcyBZXFx0aW1lcyBYXFx0aW1lcyBZIl0sWzUsMSwiWFxcdGltZXMgWSJdLFsxLDMsIlhcXHRpbWVzIChZXFx0aW1lcyBaKSJdLFswLDQsIlhcXHRpbWVzIFkiXSxbMSw0LCJYIl0sWzIsNCwiWFxcdGltZXMgWiJdLFs1LDMsIihYXFx0aW1lcyBZKVxcdGltZXMgWiJdLFs0LDQsIlhcXHRpbWVzIFoiXSxbNSw0LCJaIl0sWzYsNCwiWVxcdGltZXMgWiJdLFswLDEsIlxcRGVsdGEiLDJdLFswLDIsImlkIiwyXSxbMCwzLCJpZCJdLFsxLDIsInAiXSxbMSwzLCJxIiwyXSxbNCw1LCJcXERlbHRhIiwyXSxbNSw2LCJwXFx0aW1lcyBxIiwyXSxbNCw2LCJpZCJdLFs3LDksInAiLDJdLFs3LDgsImlkXFx0aW1lcyBwIiwyXSxbNywxMCwiaWRcXHRpbWVzIHEiXSxbMTAsOSwicCJdLFs4LDksInAiLDJdLFsxMSwxMiwicFxcdGltZXMgaWQiLDJdLFsxMSwxMywicSJdLFsxMSwxNCwicVxcdGltZXMgaWQiXSxbMTIsMTMsInEiLDJdLFsxNCwxMywicSIsMl1d
\begin{tikzcd}[column sep=tiny]
	& X &&& {X\times Y} \\
	X & {X\times X} & X && {X\times Y\times X\times Y} & {X\times Y} \\
	\\
	& {X\times (Y\times Z)} &&&& {(X\times Y)\times Z} \\
	{X\times Y} & X & {X\times Z} && {X\times Z} & Z & {Y\times Z}
	\arrow["id"', from=1-2, to=2-1]
	\arrow["\Delta"', from=1-2, to=2-2]
	\arrow["id", from=1-2, to=2-3]
	\arrow["\Delta"', from=1-5, to=2-5]
	\arrow["id", from=1-5, to=2-6]
	\arrow["p", from=2-2, to=2-1]
	\arrow["q"', from=2-2, to=2-3]
	\arrow["{p\times q}"', from=2-5, to=2-6]
	\arrow["{id\times p}"', from=4-2, to=5-1]
	\arrow["p"', from=4-2, to=5-2]
	\arrow["{id\times q}", from=4-2, to=5-3]
	\arrow["{p\times id}"', from=4-6, to=5-5]
	\arrow["q", from=4-6, to=5-6]
	\arrow["{q\times id}", from=4-6, to=5-7]
	\arrow["p"', from=5-1, to=5-2]
	\arrow["p", from=5-3, to=5-2]
	\arrow["q"', from=5-5, to=5-6]
	\arrow["q"', from=5-7, to=5-6]
\end{tikzcd}  \]
 and the associativity isomorphism
 \[\alpha_{X,Y,Z}:=((id_X \times p_{Y,Z})\times q_{Y,Z}q_{X,Y\times Z})\Delta_{X\times(Y\times Z)}:X\times (Y\times Z)\to (X\times Y)\times Z \]
 and the commutativity isomorphism
 \[\tau_{X,Y}:=(q_{X,Y}\times p_{X,Y})\Delta_{X\times Y}: X\times Y\to Y\times X \]
 are natural in all variables.

 A \textbf{one-element object} is an object $I$ with a family of arrows $t_X:X\to I$ of $\mC$ for which $p_{X,I}:X\times I \to I$ is invertible with inverse
 \[X \overset{\Delta}{\to}X\times X\overset{id_X \times t_X}{\longrightarrow} {X\times I}.\]
\end{mydefinition}
We refer to  \cite[Sec.2.1.3, Ex. 6]{Cockett02} and \cite[Sec. 4.3]{Cockett07} for the following characterisation:
\begin{myproposition}
	Categories with diagonals and garbage  correspond exactly to p-categories with one-element object.
\end{myproposition}
\begin{myremark}

	The notion of p-category lies exactly between the notion of category with diagonals and category with diagonals and garbage. The main difference is that a p-category has a form of natural ``projections'' $X\otimes Y\to X$, but it has no arrows $X\to I$ in general.

	
	Recall that p-categories with one-element object have been proved to be equivalent to several other notions: restriction categories with restriction products~\cite{Cockett07}, partial cartesian categories in the sense of Curien and Obtulowicz \cite{CURIEN198950}, and they are a special case of the bicategories of partial maps of Carboni~\cite{Carboni_87}.
\end{myremark}




\iffalse
Recall from \cite{FritzGCT23} that gs-monoidal categories have enough structure to define the notion of \textit{domain} of an arrow.

\begin{mydefinition}\label{dfn: domain}
	Let $(\mC,\otimes,I)$ be a gs-monoidal category and let $f:X\to Y$ be an arrow of $\mC$. The \textbf{domain} of $f$ is the arrow 
	\ctikzfig{domain}
\end{mydefinition}

As observed in \cite{FritzGCT23}, the domain of total and functional arrows behaves as follows.

\begin{myproposition}\label{prop: FGT23 2.14}
	Let $(\mC, \otimes,I)$ be a gs-monoidal category and let $f:X\to Y$ be an arrow in $\mC$. Then 
	\begin{itemize}
		\item if $f$ is functional, then $\mathrm{dom}(f)$ is functional,
		\item if $f$ is total, then $\mathrm{dom}(f)= id_X$.
	\end{itemize}
	\qed
\end{myproposition}
The notion of domain allow to establish the link equivalence between restriction categories and gs-monoidal categories. See \cite{FritzGCT23} and the reference therein.
\begin{mydefinition}\label{dfn: restriction}
	Restriction categories with restriction products are equivalent to gs-monoidal categories with diagonals.
\end{mydefinition}

\fi







\begin{mydefinition}\label{dfn: Markov}
    A \textbf{Markov category} is a gs-monoidal category with projections.
\end{mydefinition}
\fi

\begin{myremark}
 In \cite{cho_jacobs_2019}, the authors introduced independently \textit{CD-categories} (where ``CD" stands for copy/discard) which are the same of gs-monoidal categories and use the term \textit{affine} CD-categories for what are today known as Markov categories. The term Markov category was introduced in \cite{Fritz_2020}. In the context of Markov categories, functional arrows are referred to as \textit{deterministic}.
\end{myremark}

%===============

\section{A taxonomy of Kleisli categories}
\label{sec:taxonomy kleisli}


Our taxonomy of Kleisli categories is done for symmetric monoidal monads, which are equivalent to commutative monads, see \cite{Kock72} and now \cite[Appendix~C]{Fritz_2021}. 
We discuss three classes of such monads: affine and relevant monads, first considered in \cite{Kock71,Jacobs1994}, and gs-monoidal monads (which are both affine and relevant). 

We then study the properties of their Kleisli categories depending on the categorical structures we previously considered. Indeed, rephrasing \cite[Thm. 4.3]{Jacobs1994}, the Kleisli category of a relevant/affine monad on a cartesian category is respectively a retriction/Markov category. Hence, we extend this result considering affine and relevant monads on our one-dimensional taxonomy. 

For instance, in the following diagram  we represent for an affine monad the corresponding nature of its Kleisli category using a dashed arrow. Hence, we prove that the Kleisli category of an affine monad on a Markov category is a Markov category, while the Kleisli category of an affine monad on a restriction category is just gs-monoidal.





\[% https://q.uiver.app/#q=WzAsNCxbMSwyLCJcXHRleHR7Q2FydGVzaWFufSJdLFswLDEsIlxcdGV4dHtSZXN0cmljdGlvbn0iXSxbMiwxLCJcXHRleHR7TWFya292fSJdLFsxLDAsIlxcdGV4dHtHcy1tb25vaWRhbH0iXSxbMCwxXSxbMSwzXSxbMiwzXSxbMCwyXSxbMiwyLCIiLDIseyJhbmdsZSI6OTAsInN0eWxlIjp7ImJvZHkiOnsibmFtZSI6ImRhc2hlZCJ9fX1dLFswLDIsIiIsMix7Im9mZnNldCI6MywiY3VydmUiOjMsInN0eWxlIjp7ImJvZHkiOnsibmFtZSI6ImRhc2hlZCJ9fX1dLFszLDMsIiIsMix7InN0eWxlIjp7ImJvZHkiOnsibmFtZSI6ImRhc2hlZCJ9fX1dLFsxLDMsIiIsMix7Im9mZnNldCI6LTMsImN1cnZlIjotMywic3R5bGUiOnsiYm9keSI6eyJuYW1lIjoiZGFzaGVkIn19fV1d
\begin{tikzcd}
	& {\text{Gs-monoidal}} \\
	{\text{Restriction}} && {\text{Markov}} \\
	& {\text{Cartesian}}
	\arrow[dashed, from=1-2, to=1-2, loop, in=55, out=125, distance=10mm]
	\arrow[from=2-1, to=1-2]
	\arrow[shift left=3, curve={height=-18pt}, dashed, from=2-1, to=1-2]
	\arrow[from=2-3, to=1-2]
	\arrow[dashed, from=2-3, to=2-3, loop, in=325, out=35, distance=10mm]
	\arrow[from=3-2, to=2-1]
	\arrow[from=3-2, to=2-3]
	\arrow[shift right=3, curve={height=18pt}, dashed, from=3-2, to=2-3]
\end{tikzcd}\]


A similar taxonomy for the enriched case is going to be provided in Section~\ref{sec:oplax cartesian categories}, 
where we consider the Kleisli categories of enriched monads that are \textit{colax relevant} and \textit{colax affine}, as well as
 \textit{colax gs-monoidal}.





\begin{mydefinition}[Symmetric monoidal monad]\label{dfn: symmetric monoidal monad}
	Let $\mathcal{C}$ be a symmetric monoidal category. Let $T : \mathcal{C} \to \mathcal{C}$ be a monad carrying the structure of a lax symmetric monoidal functor with structure maps $\freccia{\otimes \circ \, (T\times T)}{c}{T\circ \otimes}$ and $u : I \to TI$. Then $T$ is a \textbf{symmetric monoidal monad} if $u = \eta_I$ and the following two diagrams commute
\begin{equation}\label{diagram: symmetric monoidal monad 1}
\begin{tikzcd}
& X \otimes Y \arrow[dl, "\eta \otimes \eta"'] \arrow[dr, "\eta"] & \\
TX \otimes TY \arrow[rr, "c"'] & & T(X \otimes Y)
\end{tikzcd}
\end{equation}

\begin{equation}\label{diagram: symmetric monoidal monad 2}
	\begin{tikzcd}
TTX \otimes TTY \arrow[r, "c"] \arrow[d, "\mu \otimes \mu"'] & T(TX \otimes TY) \arrow[r, "Tc"] & TT(X \otimes Y) \arrow[d, "\mu"] \\
TX \otimes TX \arrow[rr, "c"'] & & T(X \otimes Y)
\end{tikzcd}
\end{equation}

\end{mydefinition}
\begin{myremark}
\label{rem:comm_vs_symmon}
	It is well-known that on a symmetric monoidal category,  symmetric monoidal monads are equivalent to commutative monads, see \cite[Th.~2.3]{Kock72} and \cite[Th.~3.2]{Kock70}. Definition \ref{dfn: symmetric monoidal monad} corresponds to the notion of monad internal to the 2-category of symmetric monoidal categories, lax functors and monoidal natural transformations. The commutativity of diagrams (\ref{diagram: symmetric monoidal monad 1}) and (\ref{diagram: symmetric monoidal monad 2}) is equivalent to requiring that $\mu$ and $\eta$ are monoidal natural transformations.
\end{myremark}

\begin{myexample}\label{monadM}
Let us consider the lax symmetric monoidal functor $\mathcal{M}$ introduced in Section~\ref{WR}.
This functor defines a symmetric monoidal monad on $(\mathbf{Set}, \times, \{\bullet\})$,
sometimes called the \emph{semiring monad}, 
with natural transformations $\eta_X: X\to \mathcal{M}(X)$ and $\mu_X:\mathcal{M}(\mathcal{M}(X))\to\mathcal{M}(X)$ given by
\begin{itemize}
	\item[-] \(\eta_X(x_0)(x)=\begin{cases}
		1\ \text{if}\ x=x_0\\
		0\ \text{otherwise}
	\end{cases}\)
	\item[-] $\mu_X(\lambda)(x)=\underset{h\in\mathcal{M}(X)}{\bigoplus}\lambda(h)\cdot h(x)$
	\end{itemize}
%\fab{Cosa succede con $M^{\forall}$?}
\end{myexample}

\begin{myremark}\label{weightMon}
	Notice that, in the previous example, if $M$ is the semiring of Booleans, then  $\mathcal{M}$ is the finite subsets monad $\mathcal{P}$, while %:\mathbf{Set}\to \mathbf{Set}$.
	if $M$ is the semiring of natural numbers $\mathbb{N}$, then $\mathcal{M}$ is the 
	monad of \textit{finite multisets}, see \cite{Gadducci08}. 
	Moreover, if $M$ is the semiring of positive real numbers $\mathbb{R}^+$, then $\mathcal{M}$ is the monad of \textit{unnormalised probability distributions}, see \cite{cho_jacobs_2019}, while if $M$ is either the semiring $([0,1],\text{max},\text{min}, 0, 1)$ or the semiring $([0,1],\text{max},\cdot, 0, 1)$ one obtains Golubtsov's categories of 
	\textit{fuzzy information transformer}, see~\cite{golubtsov2002monoidal}. 
\end{myremark}

%\fab{dire come sono agiscono i funtori sui singoli oggetti}

\begin{mydefinition}\label{dfn: various monads}
	A symmetric monoidal monad $(T,\mu, \eta,c, u)$  on a gs-monoidal category $\mC$ is
	\begin{itemize}
	\item \textbf{affine} if $(T,c,u)$ is an affine functor;
	 \item \textbf{relevant} if  $(T,c,u)$ is a relevant  functor;
	\item \textbf{gs-monoidal} if $(T,c,u)$ is a gs-monoidal functor.
	\end{itemize} 
\end{mydefinition}

\begin{myremark}
	Affine and relevant monads were introduced in \cite{Kock71} in the context of cartesian monoidal categories. While the term ``affine" is used in \cite{Kock71}, the term ``relevant" appears in \cite{Jacobs1994} due to a connection with \textit{relevant logic}.
In particular, in \cite[Th.~2.1]{Kock71} the author considers as definition of affine monad on a cartesian monoidal category one of the two equivalent conditions
\begin{itemize}
	\item the unit of the monad $\eta_I:I\to T(I)$ is an isomorphism
	\item the following diagram commutes
	\[% https://q.uiver.app/#q=WzAsMyxbMCwwLCJUKFgpXFxvdGltZXMgVChZKSJdLFsxLDAsIlQoWFxcb3RpbWVzIFkpIl0sWzEsMSwiVChYKVxcb3RpbWVzIFQoWSkiXSxbMCwxLCJcXHBzaV97WCxZfSJdLFsxLDIsIlxcbGFuZ2xlIFQoXFxwaV8xKSxUKFxccGlfMilcXHJhbmdsZSJdLFswLDIsImlkIiwyLHsibGV2ZWwiOjIsInN0eWxlIjp7ImhlYWQiOnsibmFtZSI6Im5vbmUifX19XV0=
	\begin{tikzcd}
		{T(X)\otimes T(Y)} & {T(X\otimes Y)} \\
		& {T(X)\otimes T(Y)}
		\arrow["{\psi_{X,Y}}", from=1-1, to=1-2]
		\arrow["id"', Rightarrow, no head, from=1-1, to=2-2]
		\arrow["{\langle T(\pi_1),T(\pi_2)\rangle}", from=1-2, to=2-2]
	\end{tikzcd}\] 
\end{itemize}
Indeed, if the category $\mC$ has projections, $I$ is terminal and it gives an equivalence between the second condition above, 
which relies on that of affine functor (diagram (\ref{diagram: lax affine}) of Definition~\ref{def garbage functor}), 
and the fact  that $\eta_I$ is an isomorphism. 
Hence, Definition \ref{dfn: various monads} is more general since it makes sense also in contexts in which $I$ is not terminal, such as in gs-monoidal categories.


For relevant monads the author of \cite{Kock71}  proves that the condition for relevant functors (diagram (\ref{diagram: lax relevant}) of Definition~\ref{def relevant functor}) 
is equivalent to the commutativity of the diagram
\[% https://q.uiver.app/#q=WzAsMyxbMCwwLCJUKFhcXG90aW1lcyBZKSJdLFsxLDAsIlQoWClcXG90aW1lcyBUKFkpIl0sWzEsMSwiVChYXFxvdGltZXMgWSkiXSxbMCwxLCJcXGxhbmdsZSBUKFxccGlfMSksVChcXHBpXzIpXFxyYW5nbGUiXSxbMSwyLCJcXHBzaV97WCxZfSJdLFswLDIsImlkIiwyLHsibGV2ZWwiOjIsInN0eWxlIjp7ImhlYWQiOnsibmFtZSI6Im5vbmUifX19XV0=
%
\begin{tikzcd}[column sep= huge]
	{T(X\otimes Y)} & {T(X)\otimes T(Y)} \\
	& {T(X\otimes Y)}
	\arrow["{\langle T(\pi_1),T(\pi_2)\rangle}", from=1-1, to=1-2]
	\arrow["id"', Rightarrow, no head, from=1-1, to=2-2]
	\arrow["{\psi_{X,Y}}", from=1-2, to=2-2]
\end{tikzcd}\]
see \cite[Prop.~2.2]{Kock71} and also \cite[Lem.~4.2]{Jacobs1994}.
\end{myremark}

\begin{myremark}
 Observe that for any object $X$ of a gs-monoidal category, the hom-set $\mC(X,I)$ has canonically 
 the structure of a commutative monoid. This simple observation can be used to generalise the ordinary notion of affine categories and monads, by requiring additional algebraic properties for such a monoid. Indeed, the more general notions \emph{weakly Markov} category and \emph{weakly affine} monad 
 has been introduced in~\cite{FritzGPT23}: Weakly Markov categories are gs-monoidal categories such that the monoid $\mC(X,I)$ is a group. For the corresponding weakly affine monads,
 instead of assuming $T(I)$ to be isomorphic to $I$,  it is just required that the commutative monoid structure of $T(I)$ is a group.
 We are not aware of a diagrammatic characterisation as for affine categories and monads, making it applicable to gs-monoidal and relevant categories.
\end{myremark}

\begin{myremark}\label{rmk: funtore C ---> C_T}
	It is well-known that a monad $(T,\mu,\eta,c,u)$ on a symmetric monoidal category $\mC$ induces a 
	symmetric monoidal structure on $\mC_T$ if the monad is symmetric monoidal. In particular, 
	the tensor product on $\mC_T$ is defined as $X\otimes^{\sharp} Y:= X\otimes Y$ and $f\otimes^{\sharp} g:= c_{X',Y'}\circ(f\otimes g)$, where $f:X\to X'$ and $g:Y\to Y'$ are arrows of $\mC_T$.
	The obvious functor $\mathcal{K}:\mC\to\mC_T$, which is the identity on objects and acts by post-composition with $\eta$ on the arrows, is strict symmetric monoidal
\[\otimes^{\sharp}\circ(F\times F)= F\circ \otimes\]
and the arrows defining the symmetric monoidal structure on $\mC_T$ are obtained as the image of those defining the structure of $\mC$ through the functor $\mathcal{K}$. %In Propositions \ref{prop: Kleisli FGTC}, if $\mC$ is gs-monoidal then it induces a gs-monoidal structure on $\mC_T$, where $\mathcal{K}(copy_X):=copy^\sharp_X$ and $\mathcal{K}(del_X):=del^\sharp_X$.

Hence, it is possible to conclude that $\mathcal{K}$ preserves equalities of arrows
$f=g$ of $\mC$, where $f$ and $g$ are obtained through compositions and products of the structural arrows of $\mC$. 
\end{myremark}

\begin{mylemma}
	Let $(T,\mu, \eta,c, u)$ be a symmetric monoidal monad on a symmetric monoidal category $\mC$. 
	If an object has a comonoid structure in $\mC$ then it does so in $\mC_T$.
%	The functor $\mathcal{K}:\mC\to \mC_T$ maps commutative (monoids) comonoids to commutative (monoids) comonoids.
\end{mylemma}
  
%  \fab{sembra appesa, forse vale la pena di spostarlo dentro la prova della proposition sotto.}
  
  The above argument was already observed in \cite[Prop.~4.4]{FritzGCT23} (see also \cite[Lem.~2]{jacobsdion}) and it is the key step in proving the following result.

  \begin{myproposition}\label{prop: Kleisli FGTC}
  Let $(T,\mu, \eta,c, u)$ be a symmetric monoidal monad on a gs-monoidal category $\mC$. Then the Kleisli category $\mC_T$ is a gs-monoidal category with 
   $\mathrm{copy}_X$ and $\mathrm{dis}_X$ given for every object $X$ by 
  \[\mathrm{copy}^\sharp_X:=\eta_{X\otimes X} \circ \mathrm{copy}_X,\qquad \mathrm{dis}^\sharp_X:=\eta_I \circ \mathrm{dis}_X.\] \qed
  \end{myproposition}

  In other words, $\mathcal{K}:\mC\to \mC_T$ is a (strict) gs-monoidal functor. 
  
  \begin{myexample}\label{exPesRelGS}
 % \fab{Esempi su relazioni pesate}
  The Kleisli category $\mathbf{Set}_{\mathcal{M}}$ of the semiring monad $\mathcal{M}$ introduced in Section~\ref{WR} is the category of sets
	and \textit{weighted relations}, which is a gs-monoidal category. 
	Some instances for different semirings $M$ are presented in Remark~\ref{weightMon}.
  \end{myexample}
  
  We also obtain a further instance of the above result.

\begin{myproposition}\label{prop: Kleisli of affine monads}
    Let $(T,\mu, \eta,c, u)$ be an affine monad on a gs-monoidal category $\mC$. If $\mC$ has projections then so does the Kleisli category $\mC_T$.
	\begin{proof}
If $I$ is terminal and the monad is affine, then $T(I)\cong I$. Hence, $I$ is terminal also in $\mC_T$. 
    \end{proof}
\end{myproposition}

\begin{myexample}
%  \fab{Esempi su relazioni pesate}
  As shown in Section~\ref{WR}, the functor $\mathcal{M}_u$ is affine, hence the Kleisli category $\mathbf{Set}_{\mathcal{M}_u}$ has projections. 
  For the Boolean semiring, the Kleisli category of $\mathcal{P}_u$ discussed in Remark~\ref{power}
	is the category whose objects are sets and whose arrows are total relations.
%	For the semiring of positive reals, the Kleisli category of $\mathcal{M}_u$
%	is the category whose objects are xxx and  arrows are yyy (probability distributions).
%	\dav{To be comp}
\end{myexample}


\begin{myremark}
As an instance of Proposition~\ref{prop: Kleisli of affine monads}, we have that if $\mC$ is a cartesian monoidal category and $T$ is affine, then the Kleisli category $\mC_T$ has projections.
Note instead that if $\mC$ either has diagonals or it is just a gs-monoidal category, then $\mC_T$ is a gs-monoidal category.
The taxonomy is strict since the identity functor is an affine monad.
%\dav{L'ultima frase si intende il funtore canonico dalla Kleisli alla categoria di partenza giusto?}
\end{myremark}
\begin{myproposition}\label{prop: Kleisli of relevant monads}
    Let $(T,\mu, \eta,c, u)$ be a relevant monad on a gs-monoidal category $\mC$. If $\mC$ has diagonals then so does the Kleisli category $\mC_T$.
	\begin{proof}
		We need to prove the naturality of $\nabla^\sharp$, i.e. for every arrow $f:X\rightarrow Y$ in $\mC_T$, which corresponds to an arrow $f:X\to T(Y)$ in $\mC$, we show that
				\[\nabla^\sharp_Y\circ^\sharp f = (f\otimes^\sharp f)\circ^\sharp \nabla^\sharp_X \] in $\mC_T$.
				Since the composition of two arrows $f:X\rightarrow Y$ and $g:Y\to Z$ in $\mC_T$ is obtained as $\mu_Z\circ T(g)\circ f$, then we obtain
				\begin{align}
					(f\otimes^\sharp f)\circ^\sharp \nabla^\sharp_X &= \mu_{Y\otimes Y}\circ T(c)\circ T(f\otimes f)\circ\eta_{X\otimes X}\circ\nabla_X \notag\\
					&= \mu_{Y\otimes Y}\circ \eta_{T(Y\otimes Y)}\circ c_{Y,Y}\circ (f\otimes f)\circ\nabla_X \tag{naturality of $\eta$}\\
					&= \mu_{Y\otimes Y}\circ \eta_{T(Y\otimes Y)}\circ c_{Y,Y}\circ \nabla_{TY}\circ f \tag{naturality of $\nabla$}\\
					&= \mu_{Y\otimes Y}\circ \eta_{T(Y\otimes Y)}\circ T(\nabla_Y)\circ f  \tag{relevant monad}\\
					&= \nabla_Y^\sharp \circ^\sharp f. \notag
				\end{align}
			\end{proof}
\end{myproposition}

\begin{myexample}\label{exPesRel}
%	\fab{Esempi su relazioni pesate}
	The Kleisli category of the relevant monad $\mathcal{P}_e$ discussed in Remark~\ref{power} is the category of sets
	  and partial functions, which has diagonals. 
	\end{myexample}

\begin{myremark}
As an instance of Proposition~\ref{prop: Kleisli of relevant monads}, we have that if $\mC$ is a cartesian monoidal category and $T$ is relevant, 
then the Kleisli category $\mC_T$ has diagonals.
Note instead that if $\mC$ either has projections or it is just a gs-monoidal category, then $\mC_T$ is a gs-monoidal category.
The taxonomy is strict since the identity functor is a relevant monad.
\end{myremark}

This result has been proved in \cite[Lem.~4.11]{FritzGCT23},
and it is now a consequence of the two propositions given above.

\begin{mycorollary}
    Let $(T,\mu, \eta,c, u)$ be a gs-monoidal monad on a gs-monoidal category $\mC$. If $\mC$ is cartesian monoidal then so is the Kleisli category $\mC_T$. 
\end{mycorollary}



%\begin{mycorollary}\label{prop: Kleisli of gs-monoidal monads}
%    Let $(T,\mu, \eta,c, u)$ be a gs-monoidal symmetric monoidal monad. Then if $\mC$ is a gs-monoidal category, or a Markov category, or a restriction category, or a cartesian monoidal category, then so is its Kleisli category $\mC_T$. 
%\qed
%\end{mycorollary}

\begin{myexample}
\label{CSMonad}
	Let $(\mathcal{C}, \otimes, I)$ be a gs-monoidal category and $M\in\mathcal{C}$ a monoid. Then the \emph{action monad} $(-)\otimes M:\mathcal{C}\to \mathcal{C}$ is a symmetric 
	monoidal monad with $u = \cobang_M$ and
	$c_{X,Y}:(X\otimes M) \otimes (Y\otimes M)\to (X\otimes Y)\otimes M$
	given by 
	\ctikzfig{ex3_19}
	%$(\id_X\otimes \id_Y \otimes \Delta_M)\circ (\id_X\otimes s_{M,Y} \otimes \id_Y)$ and $\eta_I=\cobang_M$.
If $M$ is connected (see Definition \ref{dfn: specal and connected bimonoid}),  then the monad is affine. If $M$ is special
(see again Definition \ref{dfn: specal and connected bimonoid}), then the monad is relevant.
\end{myexample}

%\begin{myremark}
%Note that for the functor  $(-)\otimes M:\mathcal{C}\to \mathcal{C}$ to be a symmetric monoidal monad with respect to an object $M\in\mathcal{C}$,
%it would suffice for $(\mathcal{C}, \otimes, I)$ to be a gs-monoidal category and for $M$ to be a bimonoid.
%\end{myremark}

%\cipr{è la posizione giusta per questi ultimi risultati della sezione?}
%
%\fab{esempio da rivedere}
%\cipr{L'ho spezzato in esempi 3.11 e 3.14}

We close by dualising Proposition~\ref{prop: Kleisli FGTC} and its consequences.

\begin{myproposition}\label{prop:Kleisli cogs}
	Let $(T,\mu, \eta,c, u)$ be a symmetric monoidal monad on a cogs-monoidal category $\mC$. Then the Kleisli category $\mC_T$ is a cogs-monoidal category with $\mathrm{match}_X$ and 
	$\mathrm{new}_X$ given for every object $X$ by 
\[\mathrm{match}^\sharp_X:= \eta_{X} \circ \mathrm{match}_X,\qquad \mathrm{new}^\sharp_X:=\eta_X \circ \mathrm{new}_X.\] \qed
%\begin{proof}
	%It is well-known that the category $\mC_T$ inherits a symmetric monoidal structure 
	%given by $X\otimes Y$ on objects, since the monad is commutative. 
    %The arrows defining  the monoidal structure of $\mC_T$ are obtained through the image of those defining the monoidal structure of $\mC$
	%through the obvious funcor $\mC\to \mC_T$. This functor is strict symmetic monoidal and hence preserves the commutative monoids 
	%and the others equations defining the cogs-structure of $\mC$.
%\end{proof}
\end{myproposition}

%\begin{mycorollary}\label{cor: kleisli biagebraic}
%	Let $(T,\mu, \eta,c, u)$ be a symmetric monoidal monad on a bialgebraic category $\mC$. Then the Kleisli category $\mC_T$ is a bialgebraic category.
%\qed
%\end{mycorollary}

\begin{mycorollary}\label{cor: kleisli bigs}
	Let $(T,\mu, \eta,c, u)$ be a symmetric monoidal monad on a bigs-monoidal/bialgebraic/Frobenius category $\mC$. Then so is the Kleisli category $\mC_T$.
%\qed
\end{mycorollary}

\begin{myremark}
As noted, a cogs-monoidal category is the dualisation of a gs-monoidal category. The same dualisation can be exploited by introducing coprojections and codiagonals, as well as
coaffine and corelevant functors.
The results  about Kleisli categories in this section can be simply restated for these notions. We leave them as well as their enriched versions
in Appendix \ref{section: appendix dual results}, which also decribes the dual of the results 
obtained in forthcoming Section \ref{subsection:enriched taxonomy}.
\end{myremark}
%%%%%%%%%%%%%%%%%%%%%%%%%%%%%%%%%%%%%%%%%%%%%%%%%%%%%%%%%%%%%%%%%%%%%%%%%%%%%%%%%%%%%%%%%%%%%%%%%%%%%%%%%%%%%%%%%%%%%%%%%%%%%%%%%%%%
\section{A taxonomy of the enriched context}\label{sec:oplax_cart}
\label{sec:oplax cartesian categories}



The 2-categorical formalism is a generalisation of the classical one, where the hom-sets are replaced by categories. This 
has been particular useful in the context of term and graph rewriting, where rewriting sequences are modelled as 2-cells (see e.g. \cite{GadducciHL99} and the references therein). In this work 
we consider as 2-categorical context the one of preorder-enriched categories, where the hom-sets are preorders. This approach is well-suited in the context of categories for relations, which 
have a natural two-dimensional structure, given by set-theoretic inclusion. It is important to note that this enrichment becomes trivial if we consider functions instead of relations. In other words, functions are inherently one-dimensional structures, while relations provide a more complex view, which can be described through a richer categorical structure, such as a bicategory or a 2-category. We can find two relevant families of 2-categories that abstract the poset-enriched structure of $\Rel$: Cartesian bicategories \cite{CARBONI198711,cartesianbicatII} by Carboni and Walters and allegories \cite{freyd1990categories} by Freyd and Scedrov.

Hence, in the enriched context, the notion of gs-monoidal category is replaced by the notion of \textit{preorder-enriched gs-monoidal category}. 
Requiring the oplax naturality of comultiplication $\nabla_X$ and of discharger $!_X$, one obtains the notion of \textit{oplax cartesian category}. 
%
%5The 2-categorical structure allows us also to speak about adjoints of morphisms of our categories. In an oplax cartesian category, the existence of these adjoints for the $\nabla_X$ and $!_X$ is quite important as it implies the existence of a monoidal structure on each object $\Delta_X:X\times X \to X $ and $\cobang_X:I\to X$. 
%
If one requires both a comonoidal and monoidal structure on the objects and their oplax naturality, one obtains the notion of \textit{oplax bicartesian category}. 
%
Cartesian bicategories can be then presented as oplax bicartesian categories in which the enrichment is posetal and satisfy three additional conditions.
%: lax speciality, lax connectedness and $\nabla_X\circ \Delta_X\leq \id$.
%
Special cases of cartesian bicategories are bicategories of relations (those satisfying \textit{Frobenius law}) and bicategories of bialgebras. Notice that these two structures are incompatible in the sense that a bicategory of relations  that is also a bicategory of bialgebras has to be the trivial one.

\[
% https://q.uiver.app/#q=WzAsNyxbMSwwLCJcXHRleHR7b3BsYXggY2FydGVzaWFuIH0iXSxbMSwxLCJcXHRleHR7b3BsYXggYmljYXJ0ZXNpYW59ICJdLFsxLDIsIlxcdGV4dHtjYXJ0ZXNpYW4gYmljYXRlZ29yaWVzfSJdLFswLDMsIlxcdGV4dHtiaWNhdGVnb3JpZXMgb3IgcmVsYXRpb25zfSJdLFsyLDMsIlxcdGV4dHtiaWNhdGVnb3JpZXMgb2YgYmlhbGdlYnJhfSJdLFsxLDZdLFsxLDQsIlxcbWF0aGJiezF9Il0sWzEsMF0sWzIsMV0sWzMsMl0sWzQsMl0sWzYsM10sWzYsNF1d
\begin{tikzcd}[column sep= tiny]
	& {\text{oplax cartesian }} \\
	& {\text{oplax bicartesian} } \\
	& {\text{cartesian bicategories}} \\
	{\text{bicategories of relations}} && {\text{bicategories of bialgebras}} \\
	&  \mathbf{1}
	\arrow[from=2-2, to=1-2]
	\arrow[from=3-2, to=2-2]
	\arrow[from=4-1, to=3-2]
	\arrow[from=4-3, to=3-2]
	\arrow[from=5-2, to=4-1]
	\arrow[from=5-2, to=4-3]
\end{tikzcd}\]




\subsection{Order-enriched categories}
%\fab{generalizzare la prima parte ai preordini?}
We have seen that gs-monoidal categories enjoy some features of $\Rel$ with respect to total and functional arrows. 
Our next step is to recall how to build on the notion of gs-monoidality to account for the usual poset-enrichment of $\Rel$.

\begin{mydefinition}\label{def poset-enriched gs-monoidal category}
	A \textbf{preorder-enriched gs-monoidal category} $\mC$ is a gs-monoidal category $\mC$ that is also a preorder-enriched monoidal category.
\end{mydefinition}

In the following we will often consider also poset-enriched categories. 
Recall that a preorder-enriched monoidal category consists of a preorder-enriched category $\mC$, an object $I$ of $\mC$, 
a preorder-enriched functor $\freccia{\mC\times \mC}{\otimes}{\mC}$,  and enriched natural monoidal structure isomorphisms
\[
	\freccia{I\otimes -}{\lambda}{\id_{\mC}}, \qquad \freccia{- {}\otimes{} I}{\rho}{\id_{\mC}}, \qquad \freccia{(-\otimes -)\otimes -}{\alpha}{-\otimes (-\otimes -)}
\]
such that the underlying category equipped with the underlying functor $\otimes$, the object $I$, and the natural isomorphisms $\lambda$, $\rho$, and $\alpha$ is a monoidal category (see \cite{Kelly05} for details).
%\tob{That reference doesn't actually consider enriched monoidal categories (as far as I know). But maybe we can ignore that}
Since a preorder-enriched functor is just an ordinary functor that is in addition monotone, the preorder structure and the monoidal structure are required to interact by the monotonicity of the tensor product $\otimes$; the preorder-enrichment of the structure isomorphisms $\lambda$, $\rho$, and $\alpha$ does not add any additional condition since preorder-enrichement for natural transformations between preorder-enriched functors is trivial.
\begin{myexample}\label{example: span is poset-enriched}
Let us consider the category of spans  $\mathbf{Span}(\mA)$  presented in Example~\ref{ex_spans_are_gs}. It is well-known that this category has a natural 2-categorical structure, where the 2-cells are defined as follows
\begin{itemize}
\item a 2-cell $\alpha: (X \leftarrow A \to Y) \Rightarrow (X \leftarrow  B \to Y)$ is an arrow $\alpha: A \to B$ in ${\mA}$  such that the following diagram commutes
		\[
			\begin{tikzcd}[row sep=2pt, column sep=8pt]
				& A \ar[dl, bend right] \ar[rd, bend left]\ar[dd, "\alpha"] & \\
				X  & & Y  \\
				& B \ar[ul, bend left] \ar[ur,  bend right]
			\end{tikzcd}
		\]
	\item vertical composition of 2-cells is given by composition in $\mA$;
	\item horizontal composition of 2-cells as well as associators and unitors are induced by the universal property of pullbacks.
	 \end{itemize}
We will refer to $\mathbf{PSpan}(\mathcal{A})$
as the preorder-enriched category obtained by considering the preorder-reflection of the previous 2-categorical structure,
and to $(\mathbf{PSpan}(\mathcal{A}),\times,1)$ and $(\mathbf{PSpan}(\mathcal{A}),+,0)$ as its gs-monoidal counterparts.
%
Given the correspondence presented in Remark~\ref{spas as rel}, for a regular category $\mA$ in which every regular epi splits the poset-reflection of the previous 2-category boils down 
to what we call $\mathbf{PRel}(\mathcal{A})$, i.e. the usual poset-enrichment of $\mathbf{Rel}(\mathcal{A})$.
\end{myexample}

\begin{myexample}\label{SetM_preordered}
	Recall that a semiring $(M, + , \cdot, 0, 1)$ comes equipped with a canonical preorder $\leq_M$, namely $a \leq_M b$ if there exists $c$ such that $a + c = b$,
	so the Kleisli category of the semiring monad $\mathcal{M}$ introduced in Example \ref{example: monad semiring} is 
	canonically a preorder-enriched gs-monoidal category. Indeed, it is gs-monoidal thanks to Proposition \ref{prop: Kleisli FGTC} and is preorder-enriched,
	assuming that
	for two arrows $f,g:X \to Y$ in $\mathbf{Set}_{\mathcal{M}}$,  $f\le g$ if $\forall x\in X\forall y\in Y\  f(x)(y)\le_M g(x)(y)$.
	
	Note that the same construction occurs if we have a general ordered semiring $(M, \leq, + , \cdot, 0, 1)$, i.e. where $(M, \leq)$ is a preorder
	and $+$ and $\cdot$ are functions that are order-preserving on both arguments.
\end{myexample}

%\fab{verificare se $\leq$ debba essere monotono o meno}
%\cipr{Facendo i conti per vedere se il $\le$ rispetta la composizione, mi sembra che serva che sia monotono per somma e prodotto. In quello canonico indotto 
%preserva ovviamente la somma che il prodotto. Se poi il semianello è quello degli interi, allora tutti sono in relazione: $2\le 0$ perché $2+(-2)=0$. Giusto?
%}

\begin{myexample}
\label{preord}
Consider now the sub-category  $\mathbf{PreOrd}$ of $\mathbf{Set}$ whose objects are preorders and whose arrows are monotone functions.
It is cartesian monoidal, inheriting the structure from the direct product in $\mathbf{Set}$, and it 
comes equipped with a canonical preorder-enrichment, assuming that
for two arrows $f,g: X \to Y$ in $\mathbf{PreOrd}$,  $f\le g$ if $\forall x\in X\  f(x)\le g(x)$ in $Y$.
%We call the latter $\mathbf{PPreOrd}$.
\end{myexample}

%\begin{myexample}
%Let us consider the combination of the two examples above. Given a preordered semiring ${\mathcal{M}}$, 
%we can define a functor ${\mathcal{M}}^\leq: \mathbf{PreOrd} \to \mathbf{PreOrd}$
%extending the monad semiring $\mathcal{M}: \mathbf{Set} \to \mathbf{Set}$ discussed in Example \ref{example: monad semiring} acting on preorders as
% \[\mathcal{M}^\leq(X)=\left\{ h:X \to M \ |\  h\  \text{is monotone and has finite support}\right\}\]
%Since the semiring operations of $\mathcal{M}$ are monotone with respect to its preorder,
%the monotone function $f:X\to Y$ is mapped to the monotone function $\tilde{f}:\mathcal{M}(X)\to \mathcal{M}(Y)$
%given as in Example \ref{example: monad semiring}, and $\mathcal{M}^\leq$ is a lax symmetric monoidal functor 
%with the same structural arrows of the functor ${\mathcal{M}}$.
%Again, you get what we call a preordered semiring monad using almost the same 
%natural transformations as in Example~\ref{monadM}, given as
%\begin{itemize}
%	\item[-] \(\eta_X(x_0)(x)=\begin{cases}
%		1\ \text{if}\ x\leq x_0\\
%		0\ \text{otherwise}
%	\end{cases}\)
%	\item[-] $\mu_X(\lambda)(x)=\underset{h\in\mathcal{M}(X)}{\bigoplus}\lambda(h)\cdot h(x)$
%	\end{itemize}
%whose components are easily shown to be monotone.
%\fab{ho scritto sciocchezze?}
%\cipr{Purtroppo se restringiamo ai preordini con ideali principali finiti non è detto che M(X) lo sia.}
%\end{myexample}

In a general preorder-enriched gs-monoidal category, no further compatibility with a gs-monoidal structure is required.
However, most often additional axioms hold.
%
We recall the notion of \emph{oplax cartesian category} \cite[Def.~3.2]{FritzGCT23}. 
%Notice that, with respect to the original definition, we will consider just the \emph{poset-enriched} case instead of the \emph{preorder-enriched} one. 
%We consider such an instance of the original definition to make smoother the comparison with other notions, such as cartesian bicategories.

\begin{mydefinition}\label{def oplax cartesian cat}
An \textbf{oplax cartesian category} $\mC$ is a preorder-enriched gs-monoidal category $\mC$ such that every arrow is \textbf{oplax copyable} and \textbf{oplax discardable}, 
i.e. the following inequalities hold for every arrow $\freccia{X}{f}{Y}$
	\ctikzfig{oplax_cart_def}
\end{mydefinition}
The notion of \emph{oplax cocartesian category} works as expected, starting from a cogs-monoidal category and
preserving the direction of inequalities,
and one gets the notion of \emph{oplax bicartesian category}
if both previous notions hold.
%by inverting the order in the axioms of oplax cartesian category.
%
In the following, we premise the adjective \emph{strict} if the underlying order is a poset.

\begin{myexample}\label{example: pspan e' oplax}
	The category $(\mathbf{PSpan}(\mathcal{A}), \times, 1)$, introduced in Example \ref{example: span is poset-enriched}, is oplax cartesian~\cite[Prop. 5.3]{FritzGCT23},
	while $(\mathbf{PRel}(\mathcal{A}), \times, 1)$ is strictly so. In fact, both are actually oplax bicartesian.
	As for $\mathbf{PreOrd}$ with the direct product, all arrows are actually copyable and discardable, since it is cartesian monoidal, hence it 
	is oplax cartesian.
\end{myexample}
%\fab{Quale struttura gs-monoidale di PSpan \`e oplax?}

\begin{myremark}
	The notion of oplax cartesian category could be split in two, considering separately the arrows that are \emph{oplax copyable} 
	and those that are \emph{oplax discardable}, as in the previous sections.
	Here we preferred to keep these two notions together, leaving implicit the obvious partition
	and generalisation of the forthcoming results, such as Proposition~\ref{prop: Kleisli colax cartesian}.
\end{myremark}

\begin{myexample}
\label{sub}
%	\fab{vediamo se e come aggiustare}
	Let ${\mathcal{M}}: \mathbf{Set} \to \mathbf{Set}$ be the functor introduced in Section~\ref{WR} for semiring $(M,\oplus,\odot,0,1)$  equipped with a preorder such as in Example \ref{SetM_preordered}. %and its preordered version
	%$\mathbf{Set} \to \mathbf{PreOrd}$ from Example~\ref{monadM}.
	%Consider now the pre-order enriched version of the Kleisli category $\mathbf{Set}_{\mathcal{M}}$ described in Example~\ref{exPesRelGS}.
	%\cipr{Facendo un conto veloce non mi sembra si possa dire nulla. Però, forse si potrebbe prendere}
	Consider the following slight modifications of $\mathcal{M}$
	\[\mathcal{M}_e^s(X)=\left\{ h:X \to M \ |\  h\  \text{has support at most one and is sub-idempotent}\right\}\]
	\[\mathcal{M}_u^s(X)=\left\{ h:X \to M \ |\  h\  \text{has finite support and is sub-normalised}\right\}\]
	where sub-idempotent means that $\forall x \in X.\, h(x)\le h(x) \odot h(x)$ and sub-normalised that $\bigoplus_{x \in X}h(x) \le 1$.
	In the Kleisli category $\mathbf{Set}_{{M}_e^s}$ every arrow is oplax copyable, while in 
	$\mathbf{Set}_{{M}_u^s}$ every arrow is oplax discardable.
	%Mi sembra siano ancora  monadi con tutta la struttura e in più colax affine e colax relevant. Quindi in $\mathbf{Set}_{{M}_e^\le}$ ogni freccia dovrebbe essere oplax copyable, mentre in
	%$\mathbf{Set}_{{M}_u^\le}$ ogni freccia dovrebbe essere oplax discardable.\\
	%In realt\`a la monade \`e sempre su Set quindi non possiamo parlare di monade arricchita, per\`o il ragionamento sulla colax copiabili\`a e totalit\`a continua a valere se si considerano le Kleisli.	
	\end{myexample}

\begin{myremark}
\label{laxbialg}
Note that in every bigs-monoidal category that is either oplax cartesian or oplax cocartesian the following inequalities hold
%\dav{Per questa credo che non basti avere anche la struttura oplax cocartesian, ma che i co-copy e co-bang siano gli aggiunti di copy e bang}
%Add graphically the interesting axioms holding
%due to the oplax structure, namely $!_X ; \Delta_X \leq !_X \otimes !_X$ (and symmetric) and 
%$\Delta_X ; \nabla_X \leq (\nabla_X \otimes \nabla_X) ; ( id_x \otimes \rho_{X, X} \otimes id_X) ; (\Delta_x \otimes \Delta_X)$
\ctikzfig{cipr1}
%and 
\ctikzfig{cipr3}

In other terms, each object is a \emph{lax} (bicommutative) bimonoid or, 
equivalently, such categories have a \emph{lax} bialgebraic structure.
\end{myremark}

%\begin{myremark}
%Note that it would be straightforward to specialise the definition above to e.g the notion of oplax cartesian category that is also a Markov
%category,
%and similarly for restriction and cartesian monoidal categories, in such a way that 
%Proposition~\ref{prop: Kleisli colax cartesian} below could be generalised. We refrain to do so, leaving it implicit in the rest of the paper.
%\end{myremark}
In the following proposition we observe that the notion of oplax cartesian category may subsume that of restriction category.
\begin{myproposition}\label{oplax as restriction}
Let $\mC$ be a strict oplax cartesian category such that the following inequality holds
	for every pair of arrows $\freccia{X}{f}{Y}$ and $\freccia{Y}{g}{W}$
\ctikzfig{condition_oplax_is_restriction}
Then $\mC$ is a restriction category.
% with the restiction structure given by the notion of domain. Moreover, $\mC$ has restriction products when 
%\ctikzfig{rest_products}
\end{myproposition}


\begin{proof}
We already noted in the proof of Proposition~\ref{gs as restriction}
that the axioms ($R.2$) and($R.3$) of Definition~\ref{dfn: restriction category completa} hold for any gs-monoidal category when we consider the restriction structure given by the domain of an arrow
	\ctikzfig{domain}
Now, the proof that ($R.1$) also holds follows from \cite[Prop.~3.6]{FritzGCT23}.
As for  ($R.4$), it is enough to employ the first axiom of oplax cartesian categories and the additional requirement holding for $\mC$.
\end{proof}

\begin{myremark}
The previous proposition shows that the notions of strict oplax cartesian category and restriction category are very closely related, since the former satisfies the first three requirements 
($R.1$)-($R.3$) of the latter,
and the last condition ($R.4$) is recovered by the additional inequality in Proposition~\ref{oplax as restriction}.
%
Assuming such inequality, the restriction structure induced by an oplax cartesian category satisfies all the necessary conditions 
for having restriction products in the sense of \cite[p. 20]{Cockett07}, except for an inequality of the final condition, namely
\ctikzfig{rest_products}
It is a simple check, and indeed a sanity check, that in strict oplax cartesian categories this condition corresponds via ($R.1$) to the naturality of duplicators.
\end{myremark}

We close the thread with a result concerning positivity (see Definition~\ref{positive}).

\begin{mycorollary}
Let $\mC$ be a strict oplax cartesian category.
If it is positive, then it is a restriction category.
\end{mycorollary}
\begin{proof}
It is easy to note that in a strict oplax cartesian category
for any arrow $h: X \to W$ we have that $!_W \circ h$ is functional.
Combined with positivity, it implies the inequality in the statement of 
Proposition~\ref{oplax as restriction}, hence ($R.4$).
\end{proof}

%\begin{myremark}
%As noted by Tobias Fritz (personal communication), the axiom ($R.4$) is a weaker version of \emph{positivity} in Markov categories~\cite[Def.~11.22]{Fritz_2020}.
%%which is required to hold whenever the arrow $g \circ f$ is functional.
%%also wituhout the !. Too complex to explain.
%\end{myremark}

%Now, to prove that $\mC$ has restriction products in the sense of \cite[p. 20]{Cockett07}, we start by obsering that both $\nabla$ and $\id\otimes !$ are `totals', i.e. their domains are identities (because of the the axioms of gs-monoidal category). Moreover, they also satisfy the first two conditions of restriction products (it follows from the axioms of gs-monoidal categories). Then, the third condition of restrictions products, i.e. 
%\ctikzfig{reast_prod_first_cond}
%follows from the first additional inequality  we required for $\mC$ (combined with the axioms of gs-monoidal categories). Similarly, we have that the last condition of restriction products, i.e. 
%\ctikzfig{rest_products_second_cond} 
%follows by combining the second additional inequation for $\mC$ with the axioms of oplax cartesian categories.
%\end{proof}


On functors between oplax cartesian categories, one often has additional inequalities, which take the following form.


	\begin{mydefinition}\label{def:(op) oplax cartesian functor}
		Let $\mC$ and $\mD$
		%$(\mC,\otimes_{\mC},I_{\mC},\nabla^{\mC},!^{\mC})$ and $(\mD,\otimes_{\mD},I_{\mD},\nabla^{\mD},!^{\mD})$ 
		be preorder-enriched gs-monoidal categories and  
		$\freccia{\mC}{F}{\mD}$ a preorder-enriched lax symmetric monoidal functor with structure arrows $\psi, \psi_0$.
		Then $F$ is called 
			\textbf{colax affine} if the following inequality holds 
\[
% https://q.uiver.app/#q=WzAsMyxbMCwwLCJGKFgpIl0sWzIsMCwiRihJKSJdLFsxLDEsIkkiXSxbMCwyLCIhX3tGKFgpfSIsMl0sWzIsMSwiXFxwc2lfMCIsMl0sWzAsMSwiRighX1gpIl0sWzIsNSwiIiwxLHsibGV2ZWwiOjEsInN0eWxlIjp7Im5hbWUiOiJhZGp1bmN0aW9uIn19XV0=
\begin{tikzcd}
	{F(X)} && {F(I)} \\
	& I
	\arrow[""{name=0, anchor=center, inner sep=0}, "{F(!_X)}", from=1-1, to=1-3]
	\arrow["{!_{F(X)}}"', from=1-1, to=2-2]
	\arrow["{\psi_0}"', from=2-2, to=1-3]
	\arrow["\le"{anchor=center, rotate=-90}, draw=none, from=2-2, to=0]
\end{tikzcd}
\]
	and it is called \textbf{colax relevant}  if the following inequality holds
	\[% https://q.uiver.app/#q=WzAsMyxbMCwwLCJGKFgpIl0sWzIsMCwiRihYXFxvdGltZXMgWCkiXSxbMSwxLCJGKFgpXFxvdGltZXMgRihYKSJdLFswLDIsIlxcbmFibGFfe0YoWCl9IiwyXSxbMiwxLCJcXHBzaV97WCxYfSIsMl0sWzAsMSwiRihcXG5hYmxhX1gpIl0sWzIsNSwiXFxsZSIsMSx7ImxldmVsIjoxLCJzdHlsZSI6eyJib2R5Ijp7Im5hbWUiOiJub25lIn0sImhlYWQiOnsibmFtZSI6Im5vbmUifX19XV0=
\begin{tikzcd}
	{F(X)} && {F(X\otimes X)} \\
	& {F(X)\otimes F(X)}
	\arrow[""{name=0, anchor=center, inner sep=0}, "{F(\nabla_X)}", from=1-1, to=1-3]
	\arrow["{\nabla_{F(X)}}"', from=1-1, to=2-2]
	\arrow["{\psi_{X,X}}"', from=2-2, to=1-3]
	\arrow["\le"{description,rotate=-90}, draw=none, from=2-2, to=0]
\end{tikzcd}\]
	If $F$ is both colax affine and colax relevant it is called \textbf{colax gs-monoidal}\footnote{The use of ``colax''  refers to the direction of the 2-cell, namely from $F(\nabla_A)$ to $\psi_{A,A}\circ \nabla_{FA}$. Note also that in \cite{FritzGCT23} colax gs-monoidal functors were called colax cartesian.}.
			%\item $F$ is \textbf{colax opcartesian} if it is an oplax symmetric monoidal functor with structure arrows $\phi, \phi_0$ 
		%	such that the following inequalities hold %for all $A$ in $\mC$;	 			
        %        \[\begin{tikzcd}[column sep=tiny]
        %         {F(A)} && {F(A\otimes A)}   &&& FA && {F(I)} \\
         %        & {F(A)\otimes F(A)} &&&&& I
        %            \arrow[""{name=0, anchor=center, inner sep=0}, "{F(\nabla_A)}", from=1-1, to=1-3]
        %            \arrow["{\nabla_{FA}}"', from=1-1, to=2-2]
        %            \arrow["{\phi_{A,A}}", from=1-3, to=2-2]
        %            \arrow[""{name=1, anchor=center, inner sep=0}, "{F(!_A)}", from=1-6, to=1-8]
        %            \arrow["{!_{FA}}"', from=1-6, to=2-7]
        %            \arrow["{\phi_0}", from=1-8, to=2-7]
         %                \arrow["\leq"{marking}, draw=none, from=0, to=2-2]
        %            \arrow["\leq"{marking, xshift=-2pt}, draw=none, from=1, to=2-7]
        %        \end{tikzcd}\]
		%	\item $F$ is \textbf{colax bicartesian} if it is colax cartesian and colax opcartesian in such a way as to become bilax monoidal.
			%\item $F$ is \textbf{colax Frobenius cartesian} if it is colax cartesian and colax opcartesian in such as way as to become Frobenius monoidal (Definition \ref{def:frobenius monoidalfunctor}).
	\end{mydefinition}

%\fab{cosa succede con PreOrd e i funtori M? E con i funtori dell'articolo con Fritz?}
\begin{myexample}
	Let $\mC$ be a locally small oplax cartesian category. Then, for every object $A$ of $\mC$, the representable functor $\mC(A,-):\mC\to \mathbf{PreOrd}$ has a canonical colax gs-monoidal structure given by 
	\[\psi_{X,Y}: \mC(A,X)\times \mC(A, Y)\to \mC(A,X\otimes Y)\]
	sending $f:A\to X$ and $g:A\to Y$ to $(f\otimes g)\circ \nabla_A$, and 
	\[\psi_0:I\to \mC(A,I)\]
	sending the unique element of $I$ to $!_A$. We refer to \cite[Thm. 6.3]{FritzGCT23} for details.
\end{myexample}
	





\subsection{An enriched taxonomy of Kleisli categories}\label{subsection:enriched taxonomy}
	\begin{mydefinition}
A symmetric monoidal monad $(T,\mu,\eta, c,u)$ on a preorder-enriched gs-monoidal category is said to be a \textbf{colax gs-monoidal monad} if $T$ is a 
%poset-enriched functor and $c$ and $u$ provide $T$ with the structure of 
colax gs-monoidal functor.
\end{mydefinition}
\begin{myproposition}\label{prop: Kleisli colax cartesian}
	Let $(T,\mu, \eta,c, u)$ be a colax gs-monoidal monad on a preorder-enriched gs-monoidal category $\mC$. 
	If $\mC$ is oplax cartesian then so is the Kleisli category $\mC_T$.
\end{myproposition}
\begin{proof}
Recall that $\mC_T$ is gs-monoidal thanks to Proposition \ref{prop: Kleisli FGTC} and inherits the preorder-enrichment from $\mC$. We now prove that it is oplax cartesian.

	We first show that for every arrow $f:X\to Y$ in $\mC_T$, i.e. $f:X\to TY$ in $\mC$, it holds that $!^\sharp_Y \circ^\sharp f\le !^\sharp_X$ in $\mC_T$, i.e. that 
	\[T(!_Y) \circ f\le \eta_I\circ !_X\]
	 in $\mC$.
	Applying first the assumption that $T$ is colax affine (i.e. that $T(!_Y)\le \eta_I\circ !_{T(Y)}$) and then  the fact that $\mC$ is oplax cartesian (in particular, that $!_{T(Y)}\circ f\leq  !_{T(X)}$), we obtain
	\[T(!_Y)\circ f\le \eta_I\circ !_{T(Y)}\circ f\leq \eta_I\circ !_{T(X)}.\]
	

	It remains to prove that for every $f:X\to Y$ in $\mC_T$ it holds that $\nabla^\sharp_Y\circ^\sharp f\le  (f\otimes^{\sharp} f)\circ^\sharp\nabla^\sharp_X$ in $\mC_T$, i.e. that 
	\[T(\nabla_Y)\circ f\le  (\mu_{Y\otimes Y} \circ T(c_{Y,Y})\circ T(f\otimes f))\circ (\eta_{X\otimes X}\circ \nabla_X)\]
	in $\mC$. Notice that, using the naturality of $\eta$ and $\mu$, the previous inequality happens to be equivalent to
\[T(\nabla_Y)\circ f\le  c_{Y,Y}\circ (f\otimes f)\circ  \nabla_X\]	
	in $\mC$. But this can be easily derived as follows
	\[
	T(\nabla_Y)\circ f  \le c_{Y,Y}\circ \nabla_{T(Y)}\circ f \le c_{Y,Y}\circ (f\otimes f) \circ \nabla_X \]
	by first using the fact that $T$ is colax relevant (i.e. $T(\nabla_Y) \le c_{Y,Y}\circ \nabla_{T(Y)}$), and then the oplax cartesianity of $\mC$ (in particular, that $\nabla_{T(Y)}\circ f \le  (f\otimes f) \circ \nabla_X$).
\end{proof}

%\fab{esempi con PreOrd?}

\begin{myexample}
\label{preset}
Given a preorder $(X, \leq)$, the Hoare preorder on its subsets is given by $U \leq_d V$ if 
for any $x \in U$ there exists $y \in V$ such that $x \leq y$. Or, equivalently,
if $\downarrow U \subseteq \downarrow V$, for 
$\downarrow U = \{x \in X \mid \exists u \in U. x \leq u\}$
the \emph{downward-closure} of $X$.

Consider now the finite subset functor $\mathcal{P}$ on $\mathbf{Set}$,
and recall that the sub-category $\mathbf{PreOrd}$ is preorder-enriched
(see Example~\ref{preord}). 
%
Now, $\mathcal{P}$ can be extended to a preorder-enriched functor
$\mathbf{PreOrd} \to \mathbf{PreOrd}$
with the preorder $\leq_d$ on subsets given above.
Indeed, if $f \leq g: (X, \leq) \to (Y, \leq)$, then
$\mathcal{P}(f) \leq \mathcal{P}(g)$
since $\mathcal{P}(f)(U) = \bigcup_{x \in U} f(x)
\leq_d \bigcup_{x \in U} g(x) = \mathcal{P}(g)(U)$
for all $U \subseteq X$.

%
%given by $X \leq Y$ if for all $x \in X$ there exists 
%$y \in Y$ such that $x \leq y$. 
%
In fact, $\mathcal{P}$ is a colax gs-monoidal monad and it becomes affine or relevant
if one restricts respectively to subsets with  at least or at most one element,
as done in Remark \ref{power}. 
\end{myexample}

%\begin{myexample}
%Consider now the sub-category $\mathbf{PO}$ of $\mathbf{PreOrd}$ whose objects are posets.
%Given a poset $(S, \leq)$ and a subset $X \subseteq S$, we define its \emph{downward closure} $\downarrow X$ as 
%$\{s \in S \mid \exists x \in X. s \leq x\}$, and we say that a subset $X$ is saturated if 
%$ \downarrow X =    X$.
%Then consider the saturated subset monad $\mathcal{P}^\downarrow:\mathbf{Pos}\to \mathbf{Pos}$ assigning to 
%each poset $(S, \leq)$ its saturated subsets: the preorder defined in Example~\ref{preset} is now a partial order.
%As for the functions, if  $f:S\to T$, then $\mathcal{P}^\downarrow(f)$ sends a saturated subset $X$ of $S$ into $\downarrow f(X)$. 
%
%It is easy to check that $\mathcal{P}^\downarrow$ is a poset-enriched functor.
%It is also colax gs-monoidal and in fact a symmetric monoidal monad, 
%with basically the same structure arrows as $\mathcal{P}$, so that e.g. 
%$\eta_S:S\to \mathcal{P}^{\downarrow}(S)$ sends $s\in S$ to $\downarrow \{s\}$.
%
%This monad becomes strictly affine when restricted to non-empty saturated subsets and pointed posets.
%\fab{questo come diventa nell'esempio precedente?}
%\end{myexample}

\begin{myremark}
Consider now the full sub-category $\mathbf{PO}$ of $\mathbf{PreOrd}$ whose objects are posets,
which is poset-enriched.
%Given a poset $(S, \leq)$ and a subset $X \subseteq S$, we define its \emph{downward closure} $\downarrow X$ as 
%$\{s \in S \mid \exists x \in X. s \leq x\}$, and we say that a subset $X$ is saturated if 
%$ \downarrow X =    X$.
The functor $\mathcal{P}^\downarrow:\mathbf{PO}\to \mathbf{PO}$ assigns to 
each poset $(X, \leq)$ its (possibly infinite) downward-closed subsets: the preorder $\leq_d$ in Example~\ref{preset} 
is a partial order, since it coincides with subset inclusion.
As for functions, if  $f: (X, \leq) \to (Y, \leq)$, then $\mathcal{P}^\downarrow(f)$ sends a downward-closed  
subset $U$ of $X$ into the  downward-closed subset $\downarrow \bigcup_{x \in U} f(x)$ of $Y$.

It is easy to check that $\mathcal{P}^\downarrow$ is a poset-enriched functor.
It is a colax gs-monoidal monad, 
with essentially the same structure arrows as $\mathcal{P}$, so that e.g. 
$\eta_X: (X, \leq) \to (\mathcal{P}^{\downarrow}(X), \subseteq)$ sends $x\in X$ to $\downarrow \{x\}$.
This monad becomes affine when restricted to non-empty downward-closed subsets.

The monad $\mathcal{P}^\downarrow$ coincides with the \emph{lower subset monad} in~\cite{banaschewski1991projective}
Also, as shown in~\cite{banaschewski1991projective}, the algebras of $\mathcal{P}^\downarrow$ are posets with arbitrary sups, 
and the free algebras are supercoherent posets, i.e. posets with arbitrary suprema and where every element 
is a union of supercompact elements. 
Similar considerations can be made for the category $\mathbf{PO}_\bot$ of pointed posets~\cite{kozen2013kleene}.
%
\end{myremark}

\begin{myexample}
\label{laxSpecial}
	Let $\mathcal{C}$ be a preorder-enriched gs-monoidal category and $M\in\mathcal{C}$ a monoid 
	that is \emph{lax}
	special and \emph{lax} connected, i.e. 
	such that it holds
	\ctikzfig{actionmonadexample}
%
	Then \textit{action monad} $(-)\otimes M:\mathcal{C}\to \mathcal{C}$ is colax gs-monoidal with 
	the same structure maps as in Example~\ref{CSMonad}.
	%i.e. $\eta_I=\cobang_M$ and
	%\[c_{X,Y} = (\id_X\otimes \id_Y \otimes \Delta_M)\circ (\id_X\otimes s_{M,Y} \otimes id_Y)\]

	Looking at Example~\ref{example: frobenius span} and Example~\ref{example: pspan e' oplax}, we have that 
	$(\mathbf{PSpan}(\mathcal{A}),\times,1)$ 
	is a special lax connected oplax bicartesian category 
	%that is lax connected 
	(i.e. it is oplax bicartesian and each object is special and lax connected), 
	yet not connected.
\end{myexample}

%\begin{myremark}
%As for Example~\ref{CSMonad}, it would suffice for 
%$\mathcal{C}$ to be a preorder-enriched gs-monoidal category 
%and for $M$ to be a bimonoid.
%\end{myremark}

%\begin{myexample}
%Consider now a \emph{tropical} semiring  ${\mathcal{M}} = (M, +, \cdot, 0, 1)$, i.e. such that
%its addition $+$ is idempotent. 
%It is well-known that the canonical preorder $\leq_M$
%is now a partial order, and it can be equivalently defined as $a \leq_M b$ if $a + b = b$.
%In fact, $\leq_M$ is a join-semilattice, since $a \vee b = a + b$ and $\bot = 0$.
%We can then define a functor ${\mathcal{M}}^i: \mathbf{PO} \to \mathbf{PO}$
%generalising the monad semiring $\mathcal{M}: \mathbf{Set} \to \mathbf{Set}$ discussed in Example \ref{example: monad semiring} 
%acting on partial orders as
% \[\mathcal{M}^i((S, \leq_S))=(\left\{ m:S \to M \mid m\  \text{is monotone and has finite support}\right\}, \leq^S_d)\]
%where $\leq^S_d$ is the generalisation of the Hoare preorder in Example~\ref{preset} defined as 
%$m_1 \leq^S_d m_2$ if for any $x \in supp(m_1)$ there exists $y \in supp(m_2)$ such that 
%$x \leq_S y$ and $m_1(x) \leq_M m_2(y)$.
%\fab{ma $\leq^S_d$ \`e un ordine parziale? Da controllare. E comunque \`e necessario, o si pu\`o lavorare su PreOrd diretti (vedi dopo)?}
%Every monotone function $f:(S, \leq_S) \to (T, \leq_T)$ is mapped to the monotone function $\tilde{f}:(\mathcal{M}(S), \leq^S_d) \to (\mathcal{M}(T), \leq^T_d)$ 
%which sends every $M$-valued function $m:S\to M$ that is monotone and with finite support 
%to 
%\[\tilde{f}(m)(t)= \underset{s\in f^{-1}(t)}{\bigvee} m(s)\]
%To prove that $\tilde{f}$ is monotone, consider $m_1$ and $m_2$ such that $m_1 \leq^S_d m_2$:
%we need to show that  $\tilde{f} (m_1)  \leq^T_d \tilde{f} (m_2)$.
%Consider now $D_t = f^{-1}(t) \cap supp(m_1)$: it is finite and for each $s \in D_t$ there exists 
%$\hat{s} \in S$ such that $s \leq_S \hat{s}$ and $m_1(s) \leq_M m_2(\hat{s})$.
%Consider also $\hat{t} = f(\bigvee_{s\in D_t} \hat{s})$: we have that $t \leq_T \hat{t}$ since $f$ is monotone and
%\[\tilde{f}(m_1)(t) = \underset{s\in f^{-1}(t)}{\bigvee} m_1(s) =  \underset{s\in D_t}{\bigvee} m_1(s) \leq_M 
% \underset{s\in D_t}{\bigvee} m_2(\hat{s}) \leq_M  m_2 (\underset{s\in D_t}{\bigvee} \hat{s})\]
% the latter inequality due to $m_2$ being monotone
% and since $\bigvee_{s\in D_t} \hat{s} \in f^{-1} (\hat{t})$
% \[m_2 (\underset{s\in D_t}{\bigvee} \hat{s}) \leq_M \underset{s\in f^{-1}(\hat{t})}{\bigvee} m_2(s) = \tilde{f}(m_2)(\hat{t})\]
% \fab{Bisogna anche assumere che $\leq_S$ sia un ordine parziale diretto, ovvero che ogni sottoinsieme finito $X \subseteq S$ abbia un upper bound, cosi anche se non 
% esiste $\underset{s\in D_t}{\bigvee} \hat{s}$ esiste un upper bound di $\{ \hat{s} \mid s \in D_t\}$: per la prova basta}
%\fab{resta da vedere se e' lax symmetric monoidal e se e' una monade}
%%Let us consider the combination of the two examples above. Given a preordered semiring ${\mathcal{M}}$, 
%%we can define a functor ${\mathcal{M}}^\leq: \mathbf{PreOrd} \to \mathbf{PreOrd}$
%%extending the monad semiring $\mathcal{M}: \mathbf{Set} \to \mathbf{Set}$ discussed in Example \ref{example: monad semiring} acting on preorders as
%% \[\mathcal{M}^\leq(X)=\left\{ h:X \to M \ |\  h\  \text{is monotone and has finite support}\right\}\]
%%Since the semiring operations of $\mathcal{M}$ are monotone with respect to its preorder,
%%the monotone function $f:X\to Y$ is mapped to the monotone function $\tilde{f}:\mathcal{M}(X)\to \mathcal{M}(Y)$
%%given as in Example \ref{example: monad semiring}, and 
%\cipr{Il discorso sopra mi sembra tornare. Ma mi sembra che:
%\begin{itemize}
%\item non utilizziamo mai che la somma sia idempotente (che serve solo per avere un poset). Forse basta solo che $0$ sia minimo.
%\item sono necessari preordini diretti?  la dimostrazione non si può fare per assurdo? Nel caso generale sembra valga senza ipotesi sui preordini.
%\item $\tilde{f}(m)$ è monotona? non lo vedo.
%\item se $f\le g:X\to Y$ vale che $\tilde{f}\le \tilde{g}$? mi sembra ricompaiano problemi sulle antiimmagini.
%\end{itemize}
%}
%
%CLAIM: $\mathcal{M}^i$ is a lax symmetric monoidal functor 
%with the same structural arrows of the functor ${\mathcal{M}}$
%and you get a monad using the same 
%natural transformations as in Example~\ref{monadM}, given as
%\begin{itemize}
%	\item[-] \(\eta_S(s_0)(s)=\begin{cases}
%		1\ \text{if}\ s = s_0\\
%		0\ \text{otherwise}
%	\end{cases}\)
%	\item[-] $\mu_S(\lambda)(s)=\underset{m\in\mathcal{M}^i(S)}{\bigvee}\lambda(m)\cdot m(s)$
%	\end{itemize}
%whose components are easily shown to be monotone.
%Indeed, if $s_0 \leq_S s_1$, we have that $\eta_S(s_0) \leq^S_d \eta_S(s_1)$.
%%\fab{ho scritto sciocchezze?}
%%\cipr{Purtroppo se restringiamo ai preordini con ideali principali finiti non è detto che M(X) lo sia.}
%\end{myexample}

\begin{myexample}
	Given a preorder $(X, \leq)$, we define the \emph{upward-closure} of a subset $ U \subseteq X$ as $\uparrow U = \{x \in X \mid \exists u \in U.\, u \leq x\}$. 
	Consider also a semiring $(M,\oplus, \odot, 0, 1)$ and the functor $\mathcal{M}:\mathbf{Set}\to \mathbf{Set}$ discussed in Example~\ref{example: monad semiring}.
	If $M$ is preorder-enriched and $0$ is a minimal element for such preorder, then $\mathcal{M}(X)$ can be equipped with a preorder given as $h\le_u k: X\to M$ whenever
	for every upward-closed subset $U\subseteq X$ 
	\[\underset{x\in U}{\bigoplus}h(x)\le \underset{x\in U}{\bigoplus}k(x)\]
	 We can then extend $\mathcal{M}$ to a preorder-enriched functor $\mathcal{M}:\mathbf{PreOrd}\to \mathbf{PreOrd}$.
	 Let $f:(X, \leq) \to (Y, \leq)$ be a monotone function and note that for every
	 upward-closed subset $V\subseteq Y$ it holds that $f^{-1}(V)$ is upward-closed.
	 To prove that $\tilde{f}$ is monotone it suffices to see that for $h \leq_u k$ and $V$ upward-closed we have
%	 \[ \underset{y\in V}{\bigoplus}\underset{x\in f^{-1}(y)}{\bigoplus}h(x)\le  \underset{y\in V}{\bigoplus}\underset{x\in f^{-1}(y)}{\bigoplus}k(x) \]
	 \[ \underset{y\in V}{\bigoplus}\underset{x\in f^{-1}(y)}{\bigoplus}h(x) = \underset{x\in f^{-1}(V)}{\bigoplus}h(x) \le\underset{x\in f^{-1}(V)}{\bigoplus}k(x) = \underset{y\in V}{\bigoplus}\underset{x\in f^{-1}(y)}{\bigoplus}k(x) \]
	%since the set $U:=\{x \in f^{-1}(y) | y\in V\}$ is upward closed. 
	To prove that it is preorder-enriched we need to show that if
	$f\le g:X\to Y$ then for every $h:X\to M$ it holds $\tilde{f}(h)\le_u \tilde{g}(h): Y \to M$. Indeed,
	for every $V\subseteq Y$ upward-closed, if $y\in V$ and $x\in f^{-1}(y)$ then $y\le g(x)$ and $g(x)\in V$. Hence
	\[ \underset{y\in V}{\bigoplus}\underset{x\in f^{-1}(y)}{\bigoplus}h(x)\le  \underset{y\in V}{\bigoplus}\underset{x\in g^{-1}(y)}{\bigoplus}h(x) \]
	since $0$ is a minimal element for the preorder on $\mathcal{M}$ and thus $a \leq a \oplus b$ always holds. 
	Moreover, $\mathcal{M}$ is a symmetric monoidal monad with the usual 
	structure arrows
	 \begin{itemize}
		\item[-] \(\eta_X(x_0)(x)=\begin{cases}
			1\ \text{if}\ x=x_0\\
			0\ \text{otherwise}
		\end{cases}\)
		\item[-] $\mu_X(\lambda)(x)=\underset{h\in\mathcal{M}(X)}{\bigoplus}\lambda(h)\odot h(x)$
		\end{itemize}
	 Recall now Example~\ref{sub}.
	 The monad $\mathcal{M}$ just defined becomes colax affine if one restricts $\mathcal{M}(X)$ to functions with finite support and sub-normalised; it is affine if one further restricts to normalised functions.
	 Moreover, it is colax relevant if one restricts $\mathcal{M}(X)$ to sub-idempotent functions 
	 (thanks to 0 being minimal we may drop the requirement that the support is at most one for $\mathcal{M}^s_e(X)$ in Example~\ref{sub}); 
	 it is relevant if one further restricts to idempotent functions  with support at most one. 
		If we consider the semiring of positive real numbers and restrict $\mathcal{M}(X)$ to normalised functions, then the above monad can be viewed as a 
	 discrete version of the preordered Kantorovich monad, as defined in~\cite{fritz2020stochastic}.
\end{myexample}

%\begin{myremark}
%If a semiring is idempotent then the canonical preorder is actually a join-semilattice with 
%$a \vee b = a + b$ and $0$ as its bottom.
%Moreover, the preorder on $\mathcal{M}(X)$ given in the example above is now a partial order
%\end{myremark}
\begin{myexample}
Let us follow up on the previous example and assume that $M$ is a
\emph{dioid}, i.e. that the addition is idempotent.
%\footnote{They are also called \emph{idempotent} semirings or \emph{dioids}.}.
%in \url{https://ncatlab.org/nlab/show/idempotent+semiring}}, i.e. such that its addition $+$ is idempotent. 
It is well-known that the canonical preorder $\leq_M$
is now a partial order, and it can be equivalently defined as $a \leq_M b$ if $a \oplus b = b$.
In fact, $\leq_M$ is a join-semilattice, since $a \vee b = a \oplus b$ and $\bot = 0$.
Let us further assume that such a partial order is actually total. Then, the 
preorder $\leq_u$ can be described as
$h \leq_u k$ if for any $x \in supp(h)$ there exists $y \in supp(k)$ such that 
$x \leq y$ and $h(x) \leq_M k(y)$.
For $\mathcal{M}$ the Boolean semiring, we recover the Hoare preorder
of Example \ref{preset}.
\end{myexample}

%\cipr{Mi sembra si possa fare anche la versione non arricchita dell'esempio di sopra. In particolare se prendiamo una
%bigs ed un oggetto $M$ per cui le disuguaglianze sopra sono uguaglianze, allora dovremmo ottenere
%una monade simmetrica monoidale affine e relevant. Su nlab \text{https://ncatlab.org/nlab/show/bimonoid}
%i bimonoidi per cui vale l'uguaglianza a sx di sopra sono chiamati SPECIALI. quelli per cui vale l'uguaglianza a dx sono chiamati CONNESSI. 
%Ne avete mai sentito parlare? si usano da qualche parte?}
%\fab{lo scriverei direttamente per le bicats, notando in un remark che vale anche per oggetti M che sono lax monoidi}

%\fab{L'assioma di Proposition~\ref{oplax as restriction} e' preservato da una colax gs-monoidal monad?}
%\cipr{Ho fatto i conti nel caso g=id. Purtoppo non viene la disuguaglianza di 4.8. Entrambi i termini mi vengono $\le$ di un altro. Usado le disugualglanze di colax gs-monoidal e quella data da 4.8}
%\fab{Nel caso g=id le frecce dell'assioma 4.8 sono uguali nella categoria di partenza, e quindi devono esserlo anche nella Kleisli... :-)}

\subsection{Some properties of maps}
Following the usual notation adopted in the context of cartesian bicategories \cite{CARBONI198711}, we recall the notion of \emph{adjoint} and \emph{map}.
	
\begin{mydefinition}
	Let $\mC$ a poset-enriched category. An arrow $f:X\to Y$ is left adjoint to an arrow $f^*:Y\to X$ if
\ctikzfig{adjoints}
Equivalently, $f^*$ is right adjoint to $f$. In symbols $f\dashv f^*$.
\end{mydefinition}

\begin{mydefinition}
		Let $\mC$ be a strict oplax cartesian category. An arrow $\freccia{X}{f}{Y}$ is a \textbf{map} if it has a right adjoint $f\dashv f^*$.
	\end{mydefinition}
	
We denote by $\mathbf{Map}(\mC)$ the poset-enriched sub-category of $\mC$ whose objects are those of $\mC$, and whose arrows are maps. 


\begin{mylemma}
Let $\mC$ be a strict oplax cartesian category. Then every map is $\mC$-functional and $\mC$-total.
\end{mylemma}
\begin{proof}
By definition of oplax cartesian category, we just need to prove that every map  $\freccia{X}{f}{Y}$ satisfies
\ctikzfig{map_are_tot_and_fun}
To prove that every map is $\mC$-total we just need to observe that, by definition of left adjoint, we have that
\ctikzfig{proof_map_are_tot}
where the last inequality follows by the second axiom of oplax cartesian categories (applied to $f^*$).
%
Similarly, we can prove that every map is $\mC$-functional as follows
\ctikzfig{proof_map_are_fun} 
where the first and the last inequalities follow from the fact that $f\dashv f^*$, and the second one follows from the first inequality in the definition of oplax cartesian category.
\end{proof}
\begin{myremark}\label{rem_map_sub_cat_of_TFun}
Notice that in general $\mathbf{Map}(\mC)$ does not coincide with $\mC$-$\mathbf{TFun}$, i.e. an arrow that is both $\mC$-total and $\mC$-functional is not
necessarily a map. Indeed,  $\mathbf{Map}(\mC)$ is just a poset-enriched monoidal sub-category of $\mC$-$\mathbf{TFun}$.
\end{myremark}


\subsection{Cartesian bicategories through a gs-monoidal lens}
Now we recall the well-known notion of \emph{cartesian bicategory}~\cite[Def. 1.2]{CARBONI198711}, presenting it in terms of oplax cartesian category.
\begin{mydefinition}\label{def_cartesian_bicategory}
	A \textbf{cartesian bicategory} $\mC$ is a strict oplax cartesian category such that
 for every object $X$, the arrows $\nabla_X : X \to X \otimes X$ and $!_X : X \to I$ are maps, i.e. there are arrows $\Delta_X : X \otimes X \to X $ and $\cobang_X: I \to X$ such that 
 \ctikzfig{cart_bicat_def}
\ctikzfig{cart_bicat_def_2}
\end{mydefinition}

These inequalities act on orthogonal components, and thus hold for a large number of constructions. 
As partly shown later, they hold for bicategories of spans and cospans and coproduct on 
$\mathbf{Set}$, sometimes with an equality~\cite{Bruni2003}.

The overloaded choice of the symbols $\Delta_X$ and $\cobang_X$ is by no means by chance. Indeed, the following property holds.


\begin{myproposition}\label{prop: every cartesian bicategory is oplax cocartesian}
	Every cartesian bicategory $\mC$ is oplax cocartesian.
	\begin{proof}
	First, notice that $\mC$ is a cogs-monoidal category. Indeed, observe that all the equations that $\Delta_X$ and $\cobang_X$
	have to satisfy follow from the fact that the right adjoint is unique and from the right adjoint inequalities 
	in Definition \ref{def_cartesian_bicategory}, see also \cite[Rem. 1.3]{CARBONI198711}. The two inequalities required in the definition of oplax cocartesian category can be easily proved as follows: for every arrow $f:X\to Y$
\ctikzfig{oplax_cocart_prop_1}
where the first and the last inequality follow by the properties of right adjoints, while the second one follows from the oplax cartesianity. Similarly, for every arrow $f:X\to Y$
\ctikzfig{oplax_cocart_prop_2}
	\end{proof}	
\end{myproposition}

%\begin{myexample}
%  \fab{Esempi}
%\end{myexample}

We can then sum up the relationship between cartesian bicategories and the gs-monoidal framework.

\begin{mycorollary}
Cartesian bicategories correspond exactly to strict oplax bicartesian categories 
that are lax special and lax connected and satisfy
\ctikzfig{cart_bicat_mezzo}
\end{mycorollary}
\begin{proof}
The proposition above implies that a cartesian bicategory
is an oplax bicartesian category, see Definition~\ref{def oplax cartesian cat}.
Now, Remark~\ref{laxbialg} ensure that 
an oplax bicartesian category is also lax algebraic, hence 
the rightmost axiom on the bottom of Definition~\ref{def_cartesian_bicategory}
holds. Then, being lax special and lax connected, see Example~\ref{laxSpecial},
 the additional requirement ensures the correspondence with the remaining inequalities
holding for cartesian bicategories.
\end{proof}

%\fab{incertezza su un assioma}
%\cipr{come faccio a verificare $match;\nabla\le id\otimes id$?}

\begin{mydefinition}\label{dfn: bicategory of bialgebras}
A \textbf{bicategory of bialgebras} is a cartesian bicategory 
that is also a bialgebraic category.
\end{mydefinition}

%\fab{bicategory of bialgebras? direi che si ottiene Span(A), +}

As before, a key difference are the additional laws for their interaction~\cite{BruniGM02}.


\begin{mydefinition}\label{dfn: bicategory of relations}
A \textbf{bicategory of relations} is a cartesian bicategory 
that is also a Frobenius category.
%\ctikzfig{frobeniuscorta}
\end{mydefinition}



\begin{myexample}
As expected, $(\mathbf{PRel}(\mathcal{A}),\times,1)$ is a bicategory of relations and $(\mathbf{PRel}(\mathcal{A}),+,0)$ a bicategory of bialgebras.
\end{myexample}

\begin{myremark}
%It is worth to recall that t
The notion of bicategory of relations is equivalent to the notion of \emph{unitary pretabular allegory} in the sense of Freyd and Scedrov~\cite{freyd1990categories}. Moreover, a bicategory of relations $\mC$ happens to be biequivalent to the regular category $\mathbf{Map}(\mC)$ under the further assumption of being \emph{functionally complete}, as shown in \cite[Th. 3.5]{CARBONI198711}. We refer to \cite{Fong19} for more detail.
\end{myremark}

%\begin{myremark}\label{logics}
%	In recent works \cite{bonchi2024,bonchi_trotta_2024} it has been considered also the notion of \emph{cocartesian bicategory},  that is essentially a category satisfying the same axioms of a cartesian bicategory, but with the reversed order $\geq$.
%	This notion abstracts the structure of the category of relations when we consider another natural way for composing two relations: recall from Example~\ref{example: due relazioni}that the ordinary composition (providing the structure of cartesian bicategory) of a relation $a\subseteq X\times Y$ with $b\subseteq Y\times Z$ is defined as 
%	\[b\circ a:=\{(x,z)|\exists y\in Y,(x,y)\in a \wedge (y,z)\in b \}\subseteq X\times Z\]
%	while the second composition (providing the structure of cocartesian bicategory) is defined as 
%	\[b\bullet a:=\{(x,z)|\forall y\in Y,(x,y)\in a \vee (y,z)\in b \}\subseteq X\times Z.\]
%    The category of sets and relations together with the second composition provides the main example of cocartesian category in~\cite{bonchi2024}. 
%
%	In the same way a lax cartesian category is defined by inverting the order in the axioms an oplax cartesian category, a cocartesian category satisfies the same axioms of a cartesian bicategory but with the reversed order.
%\end{myremark}
%
%\fab{parte dovrebbe essere anticipata quando parliamo di relazioni pesate in esempio \ref{due relazioni} ed esempio \ref{example: monad semiring}}

\begin{myremark}
One of the useful consequence of the Frobenius law is that, as it happens in $\mathbf{Rel}$, in every cartesian bicategory of relations, if $f\leq g$ and both are maps then they are equal~\cite{CARBONI198711}, i.e. the order in $\mathbf{Map}(\mC)$  is discrete.
\end{myremark}

\begin{myremark}\label{rem_map_prod}
We recall from \cite{CARBONI198711} that the category $\mathbf{Map}(\mC)$ of a cartesian bicategory is actually cartesian. Moreover, combining Corollary~\ref{cor_TFun_is_cartesian} with it is a cartesian sub-category of  $\mC$-\textbf{TFun}. However, the cartesian structure is not enough to conclude that every $\mC$-function and $\mC$-total arrow is a map. Indeed, this happens when $\mC$ is a bicategory of relations~\cite[Lem. 2.5]{CARBONI198711}.
\end{myremark}




\begin{myproposition}
	Let $(T,\mu, \eta,c, u)$ be a colax gs-monoidal monad on a poset-enriched gs-monoidal category $\mC$. 
	If $\mC$ is a cartesian bicategory then so is the Kleisli category $\mC_T$.
\end{myproposition}
\begin{proof}
%It is enough to define $\Delta^{\sharp}_X : X \otimes X \to X $ and $\cobang^{\sharp}_X: I \to X$ in $\mC_T$ as the arrows $\Delta^{\sharp}_X:=\eta_X\circ \Delta_X$ and $\cobang^{\sharp}_X:= \eta_X\circ\cobang_X$ of $\mC$. It is direct to check (just using the naturality of the unit of the monad $\eta$ and the hypothesis that $\nabla_X$ and $!_X$ have right adjoints in $\mC$) that these arrows provide right adjoints to the arrows $\nabla_X$ and $!_X$. For example, by definition of composition in $\mC_T$, we have that
%\[ \Delta^{\sharp}_X\circ \nabla^{\sharp}_X=\mu_X \circ T(\eta_X\circ \Delta_X) \circ (\eta_{X\otimes X}\circ \nabla_X )=\eta_X\circ (\Delta_X\circ \nabla_X)\leq \eta_X\]
%i.e. $\Delta^{\sharp}_X\circ \nabla^{\sharp}_X\leq \id_X^{\sharp}$ in $\mC_T$.
%Similarly, one can check that if $\mC$ satisfies the Frobenius law, then also $\mC_T$ does as well.




By Proposition \ref{prop: Kleisli colax cartesian}, we have that $\mC_T$ is oplax cartesian. Moreover, by defining $\Delta^{\sharp}_X : X \otimes X \to X $ and $\cobang^{\sharp}_X: I \to X$ in $\mC_T$ as the arrows $\Delta^{\sharp}_X:=\eta_X\circ \Delta_X$ and $\cobang^{\sharp}_X:= \eta_X\circ\cobang_X$ of $\mC$, we obtain right adjoints
to $\nabla^\sharp$ and $!^\sharp$ since the functor $\mathcal{K}:\mC\to \mC_T$ is poset-enriched. \end{proof}

\begin{mycorollary}
	Let $(T,\mu, \eta,c, u)$ be a colax gs-monoidal monad on a poset-enriched gs-monoidal category $\mC$. 
	If $\mC$ is a bicategory of bialgebras/of relations then so is the Kleisli category $\mC_T$.
\end{mycorollary}
\begin{proof}
It follows from the fact that the functor $\mathcal{K}: \mC \to \mC_T$ preserves the equalities of arrows obtained through composition and products of structural maps, see Remark \ref{rmk: funtore C ---> C_T}.	
\end{proof}

\section{Conclusions and further work}
\label{sec:conclusions}
The aim of the paper has been twofold. On the one side, we strove for putting some order in the  array of those variants of symmetric monoidal categories, possibly poset-enriched,  
that has been proposed with computational and graphical aims in recent years. We hope that our streamlined presentation in terms of gs-monoidal categories could be beneficial to the 
interested community.
On the other side, we presented a series of results concerning the gs-monoidal and oplax cartesian structures of Kleisli categories, putting some order also on 
those variants of enrichments of commutative monads proposed in the literature.

%Also, any bialgebraic category additionally has a lax Hopf algebra structure, where the antipode for an arrow $f: X \to Y$ is given by
%\[a(f) = (id_Y \otimes \cobang_X) ;  (id_Y \otimes \nabla_X); (id_Y \otimes f \otimes id_X) ; (\Delta_Y \otimes id_X) ; (!_Y \otimes id_X): Y \to X \]
%and the characterising diagram holds in a lax form
%\[XXX\]
 
 Future threads of investigations include the combinatorial presentation of free gs-monoidal categories in terms of classes of graphs, as 
 shown for string diagrams and hyper-graphs in e.g.~\cite{fabio2022}.
 %extending those discussed in~\cite{CorradiniGadducci99}.
 Another line of work concerns the extension of the taxonomy towards \emph{traced} gs-monoidal categories~\cite{Joyal_tracedcategories,CorradiniGadducci99b}, 
 in order to account for systems with feedback, and the lively area investigating alternative notions of Markov categories~\cite{LavoreR23,FritzGPT23}
 and affine monads~\cite{Jacobs16,FritzGPT23}, aimed  at distilling a categorical presentation of probability theory, see e.g.~\cite{Jacobs18}.
  And finally, the exploration of completeness theorems for functorial semantics, see 
 e.g.~\cite{FritzGCT23}.
 % and the references therein.
 
% \fab{inserire anche riferimenti a teoremi di completezza?}
 
%\cipr{Quello che mi sembra di capire, come già scritto nella dimostrazione di Proposizione \ref{prop:Kleisli cogs}
%è che la struttura monoidale di $\mC_T$ è indotta da $\mC$ nel senso che $X\otimes_T Y:=X\otimes Y$
%e $f\otimes_T g:= c_{T(X'),T(Y')}\circ(f\otimes g)$, dove $f:X\to T(X')$ e $g:Y\to T(Y')$ sono frecce di $\mC_T$.
%Inoltre, se denotiamo con $F:\mC\to \mC_T$ il funtore ovvio (identita sugli oggetti e post-composizione con $\eta$ sulle frecce), allora abbiamo 
%che $F$ è strict simmetric monoidal
%\[\otimes_T(F\times F)= F\circ \otimes\]
%osservando pure che le mappe di struttura che definiscono la struttura simmetrica monoidal di $\mC_T$, come ad esempio il braiding, sono date 
%dall'immagine di $F$ sulle mappe di struttura di $\mC$, e che lo stesso vale
%per $F(copy_X):=copy^\sharp_X$, $F(del_X):=del^\sharp_X$, $F(match_X):=match^\sharp_X$ and $F(new_X):=new^\sharp_X$
%allora mi sembra possibile dedurre che ogni uguaglianza $f=g$ in $\mC$, dove $f$ e $g$ sono ottenute
%attraverso la composizione e i prodotti delle mappe di sopra, si trasferisce automaticamente a $\mC_T$, cioè $F(f)=F(g)$. Inoltre, se 
%poniamo come poset-enrichment su $\mC_T(X,Y)$  quello di $\mC(X,Y)$ otteniamo pure che $F$ preserva l'ordine.
%Dunque anche le varie disuguaglianze in $\mC$ si trasferiscono a $\mC_T$.


%Quello che scrivo è giusto o non vedo l'ELEFANTE NELLA STANZA?}



%=============================




%============================
\bibliographystyle{plain}
\bibliography{references}

%==============

\begin{appendices}
    \section{Lax monoidal functors}
\label{sec:lax_app}

This section recalls the details of the definition of lax monoidal functor, see e.g.~\cite{aguiar2010}.
Throughout, $\mC$ and $\mD$ are symmetric monoidal categories with tensor functor $\otimes$ and monoidal unit $I$, and we assume that 
$\otimes$ strictly associates without loss of generality in order to keep the diagrams simple.
Left and right unitors are denoted by $\lambda$ and $\rho$, respectively\footnote{Strict unitality could be assumed, but it would make some diagrams potentially confusing.}, and braidings by $\gamma$.

\begin{mydefinition}\label{def:lax monoidal functor}
A functor $\freccia{\mC}{F}{\mD}$ is \textbf{lax monoidal} if it is equipped with a natural transformation 
\[\freccia{\otimes \circ \, (F\times F)}{\psi}{F\circ \otimes}\]
and an arrow $\freccia{I}{\psi_0}{F(I)}$ such that the associativity diagrams
% https://q.uiver.app/?q=WzAsNCxbMCwwLCJGQVxcb3RpbWVzX3tcXG1hdGhjYWx7RH19RkJcXG90aW1lc197XFxtYXRoY2Fse0R9fUZDIl0sWzMsMCwiRkFcXG90aW1lc197XFxtYXRoY2Fse0R9fUYoQlxcb3RpbWVzX3tcXG1hdGhjYWx7Q319QykiXSxbMCwyLCJGKEFcXG90aW1lc197XFxtYXRoY2Fse0N9fUIpXFxvdGltZXNfe1xcbWF0aGNhbHtEfX1GQyJdLFszLDIsIkYoQVxcb3RpbWVzX3tcXG1hdGhjYWx7Q319Qlxcb3RpbWVzX3tcXG1hdGhjYWx7Q319QykiXSxbMCwxLCJpZFxcb3RpbWVzX3tcXG1hdGhjYWx7RH19XFxwc2lfe0IsQ30iXSxbMCwyLCJcXHBzaV97QSxCfVxcb3RpbWVzX3tcXG1hdGhjYWx7RH19aWQiLDJdLFsyLDMsIlxccHNpX3tBXFxvdGltZXNfe1xcbWF0aGNhbHtDfX1CLEN9IiwyXSxbMSwzLCJcXHBzaV97QSxCXFxvdGltZXNfe1xcbWF0aGNhbHtDfX1DfSJdXQ==
\[\begin{tikzcd}[column sep=3ex]
	{F(A)\otimes F(B)\otimes F(C)} &&& {F(A)\otimes F(B\otimes C)} \\
	\\
	{F(A\otimes B)\otimes F(C)} &&& {F(A\otimes B\otimes C)}
	\arrow["{\id\otimes \,\psi_{B,C}}", from=1-1, to=1-4]
	\arrow["{\psi_{A,B}\otimes {}\id}"', from=1-1, to=3-1]
	\arrow["{\psi_{A\otimes B,C}}"', from=3-1, to=3-4]
	\arrow["{\psi_{A,B\otimes C}}", from=1-4, to=3-4]
\end{tikzcd}\]
and the unitality diagrams commute
\begin{equation}
\label{eq:lax_monoidal_unitality}
\begin{tikzcd}[column sep=1ex]
	{I\otimes F(A)} && F(A) && {F(A)\otimes  I} && F(A) \\
	\\
	{F(I)\otimes F(A)} && {F(I\otimes A)} && {F(A)\otimes F(I)} && {F(A\otimes  I).}
	\arrow["{\psi_{I,A}}"', from=3-1, to=3-3]
	\arrow["{F(\lambda_A)}", from=1-3, to=3-3]
	\arrow["{\psi_0\otimes {}\id}"', from=1-1, to=3-1]
	\arrow["{\lambda_{FA}}"', from=1-3, to=1-1]
	\arrow["{\rho_{FA}}"', from=1-7, to=1-5]
	\arrow["{\id\otimes \psi_0}"', from=1-5, to=3-5]
	\arrow["{\psi_{A,I}}"', from=3-5, to=3-7]
	\arrow["{F(\rho_A)}", from=1-7, to=3-7]
\end{tikzcd}
\end{equation}
%commute.

$F$ is called \textbf{lax symmetric monoidal} if also the following diagram commutes


	% https://q.uiver.app/#q=WzAsNCxbMCwxLCJGKEFcXG90aW1lcyBCKSJdLFsyLDEsIkYoQlxcb3RpbWVzIEEpIl0sWzAsMCwiRihBKVxcb3RpbWVzIEYoQikiXSxbMiwwLCJGKEIpXFxvdGltZXMgRihBKSJdLFswLDEsIkYoXFxnYW1tYV57XFxtYXRoY2Fse0N9fV97QSxCfSkiLDJdLFsyLDAsIlxccHNpX3tBLEJ9Il0sWzIsMywiXFxnYW1tYV57XFxtYXRoY2Fse0R9fV97RkEsRkJ9Il0sWzMsMSwiXFxwc2lfIHtCLEF9IiwyXV0=
\begin{tikzcd}
	{F(A)\otimes F(B)} && {F(B)\otimes F(A)} \\
	{F(A\otimes B)} && {F(B\otimes A)}
	\arrow["{\gamma^{\mathcal{D}}_{FA,FB}}", from=1-1, to=1-3]
	\arrow["{\psi_{A,B}}", from=1-1, to=2-1]
	\arrow["{\psi_ {B,A}}"', from=1-3, to=2-3]
	\arrow["{F(\gamma^{\mathcal{C}}_{A,B})}"', from=2-1, to=2-3]
\end{tikzcd}
\end{mydefinition}

For example, if $\mC$ is the terminal monoidal category with only one object $I$ and $\id_I$ as the only arrow, then $F$ is simply a monoid in $\mD$.
We do not spell out the following dual version in full detail.



We also have \textbf{strong symmetric monoidal functors}, which are lax symmetric monoidal functors with invertible structure arrows; and \textbf{strict symmetric monoidal functors}, in which the structure arrows are identities.

\begin{mydefinition}
	A \textbf{monoidal transformation} between lax monoidal functors $\freccia{(F,\psi_0,\psi)}{\epsilon}{(F',\psi'_0,\psi')}:\mC\to\mD$ is a family of arrows $\epsilon_X:F(X)\to F'(X)$, for $X\in\mC$, satisfying
\[% https://q.uiver.app/#q=WzAsNyxbMCwwLCJGKFgpXFxvdGltZXMgRihZKSJdLFsyLDAsIkYnKFgpXFxvdGltZXMgRicoWSkiXSxbMCwxLCJGKFhcXG90aW1lcyBZKSJdLFsyLDEsIkYnKFhcXG90aW1lcyBZKSJdLFs0LDAsIkkiXSxbNiwwLCJGKEkpIl0sWzUsMSwiRicoSSkiXSxbMCwxLCJcXGVwc2lsb25fWFxcb3RpbWVzXFxlcHNpbG9uX1kiXSxbMSwzLCJcXHBzaSciXSxbMiwzLCJcXGVwc2lsb25fe1hcXG90aW1lcyBZfSIsMl0sWzAsMiwiXFxwc2kiLDJdLFs0LDYsIlxccHNpJ18wIiwyXSxbNSw2LCJcXGVwc2lsb25fSSJdLFs0LDUsIlxccHNpXzAiXV0=
\begin{tikzcd}[column sep=tiny]
	{F(X)\otimes F(Y)} && {F'(X)\otimes F'(Y)} && I && {F(I)} \\
	{F(X\otimes Y)} && {F'(X\otimes Y)} &&& {F'(I)}
	\arrow["{\epsilon_X\otimes\epsilon_Y}", from=1-1, to=1-3]
	\arrow["\psi"', from=1-1, to=2-1]
	\arrow["{\psi'}", from=1-3, to=2-3]
	\arrow["{\psi_0}", from=1-5, to=1-7]
	\arrow["{\psi'_0}"', from=1-5, to=2-6]
	\arrow["{\epsilon_I}", from=1-7, to=2-6]
	\arrow["{\epsilon_{X\otimes Y}}"', from=2-1, to=2-3]
\end{tikzcd}\]
When $\epsilon$ is also a natural transformation between the underlying functors $F,F'$ it is called \textbf{monoidal natural transformation}.
\end{mydefinition}
A monoid and comonoid structure on an object in a symmetric monoidal category often interact in a nice way, either such that they form a \emph{bimonoid} or a \emph{Frobenius monoid} (and sometimes both).



\section{Restriction categories}


\begin{mydefinition}\label{dfn: restriction category completa}
	A category $\mC$ is a \textbf{restriction category} if there exists an assignment which sends every arrow $f:X\to Y$ of $\mC$ to an arrow $\overline{f}:X\to X$ such that the following conditions hold
\begin{itemize}
	\item[(R.1)] $f \circ \overline{f}=f$,
	\item[(R.2)] $\overline{f} \circ \overline{g} = \overline{g} \circ \overline{f}$ for $g: X \to W$,
	\item[(R.3)] $\overline{g \circ \overline{f}}= \overline{g} \circ \overline{f}$ for $g: X \to W$,
	\item[(R.4)] $\overline{g} \circ f= f\circ \overline{g \circ f}$ for $g: Y \to W$.
\end{itemize}
An arrow $f:X\to Y$ in $\mC$ is \textbf{total} if $\overline{f}=id_X$.

A restriction category has \textbf{restriction terminal object} if there exists an object $I$ such that for every $X\in\mC$ there is a total arrow $t_X:X\to I$ such that $t_I=id_I$ and for every $f:X\to Y$ we have $t_Y \circ f=t_X \circ \overline{f}$.

A restriction category has \textbf{restriction binary products} if there exist a restriction functor 
$-\times -: \mC\times \mC\to \mC$
(i.e.\ such that $\overline{f\times g}=\overline{f}\times \overline{g}$) and total arrows $\Delta: X\to X\times X$, $p:X\times Y\to X$ and $q:X\times Y\to Y$ satisfying

% https://q.uiver.app/#q=WzAsMjAsWzEsMCwiWCJdLFsxLDEsIlhcXHRpbWVzIFgiXSxbMCwxLCJYIl0sWzIsMSwiWCJdLFs0LDAsIlhcXHRpbWVzIFkiXSxbNCwxLCJYXFx0aW1lcyBZXFx0aW1lcyBYXFx0aW1lcyBZIl0sWzUsMSwiWFxcdGltZXMgWSJdLFswLDMsIlhcXHRpbWVzIFkiXSxbMSwzLCJYXFx0aW1lcyBZIl0sWzIsMywiWFxcdGltZXMgWSJdLFswLDQsIlgiXSxbMiw0LCJZIl0sWzAsNSwiWCciXSxbMSw1LCJYJ1xcdGltZXMgWSciXSxbMiw1LCJZJyJdLFs0LDMsIlgiXSxbNSwzLCJYIl0sWzUsNCwiWFxcdGltZXMgWCJdLFs1LDUsIlgnXFx0aW1lcyBYJyJdLFs0LDUsIlgnIl0sWzAsMSwiXFxEZWx0YSIsMl0sWzAsMiwiaWQiLDJdLFswLDMsImlkIl0sWzEsMiwicCJdLFsxLDMsInEiLDJdLFs0LDUsIlxcRGVsdGEiLDJdLFs1LDYsInBcXHRpbWVzIHEiLDJdLFs0LDYsImlkIl0sWzgsNywiXFxvdmVybGluZXtmfVxcdGltZXNcXG92ZXJsaW5le2d9IiwyXSxbOCw5LCJcXG92ZXJsaW5le2Z9XFx0aW1lc1xcb3ZlcmxpbmV7Z30iXSxbNywxMCwicCIsMl0sWzEwLDEyLCJmIiwyXSxbOCwxMywiZlxcdGltZXMgZyIsMl0sWzksMTEsInEiXSxbMTEsMTQsImciXSxbMTMsMTIsInAiXSxbMTMsMTQsInEiLDJdLFsxNSwxNiwiXFxvdmVybGluZXtmfSJdLFsxNiwxNywiXFxEZWx0YSJdLFsxNywxOCwiZlxcdGltZXMgZiJdLFsxNSwxOSwiZiIsMl0sWzE5LDE4LCJcXERlbHRhIiwyXV0=
\[\begin{tikzcd}[column sep= small]
	& X &&& {X\times Y} \\
	X & {X\times X} & X && {X\times Y\times X\times Y} & {X\times Y} \\
	\\
	{X\times Y} & {X\times Y} & {X\times Y} && X & X \\
	X && Y &&& {X\times X} \\
	{X'} & {X'\times Y'} & {Y'} && {X'} & {X'\times X'}
	\arrow["id"', from=1-2, to=2-1]
	\arrow["\Delta"', from=1-2, to=2-2]
	\arrow["id", from=1-2, to=2-3]
	\arrow["\Delta"', from=1-5, to=2-5]
	\arrow["id", from=1-5, to=2-6]
	\arrow["p", from=2-2, to=2-1]
	\arrow["q"', from=2-2, to=2-3]
	\arrow["{p\times q}"', from=2-5, to=2-6]
	\arrow["p"', from=4-1, to=5-1]
	\arrow["{\overline{f}\times\overline{g}}"', from=4-2, to=4-1]
	\arrow["{\overline{f}\times\overline{g}}", from=4-2, to=4-3]
	\arrow["{f\times g}"', from=4-2, to=6-2]
	\arrow["q", from=4-3, to=5-3]
	\arrow["{\overline{f}}", from=4-5, to=4-6]
	\arrow["f"', from=4-5, to=6-5]
	\arrow["\Delta", from=4-6, to=5-6]
	\arrow["f"', from=5-1, to=6-1]
	\arrow["g", from=5-3, to=6-3]
	\arrow["{f\times f}", from=5-6, to=6-6]
	\arrow["p", from=6-2, to=6-1]
	\arrow["q"', from=6-2, to=6-3]
	\arrow["\Delta"', from=6-5, to=6-6]
\end{tikzcd}\]
A restriction category has \textbf{restriction products} if it has restriction terminal object and restriction binary products.
\end{mydefinition}

%\begin{mydefinition}\label{dfn: p-category}
%	A \textbf{p-category} is a category $\mC$ endowed with a functor $-\times -:\mC\times \mC\to \mC$, a natural family of arrows $\Delta:X\to X\times X$, and families $p_{X,Y}:X\times Y\to X$ natural in $X$, and $q_{X,Y}:X\times Y\to Y$ natural in $Y$, such that the following diagrams comute 
%\[% https://q.uiver.app/#q=WzAsMTUsWzEsMCwiWCJdLFsxLDEsIlhcXHRpbWVzIFgiXSxbMCwxLCJYIl0sWzIsMSwiWCJdLFs0LDAsIlhcXHRpbWVzIFkiXSxbNCwxLCJYXFx0aW1lcyBZXFx0aW1lcyBYXFx0aW1lcyBZIl0sWzUsMSwiWFxcdGltZXMgWSJdLFsxLDMsIlhcXHRpbWVzIChZXFx0aW1lcyBaKSJdLFswLDQsIlhcXHRpbWVzIFkiXSxbMSw0LCJYIl0sWzIsNCwiWFxcdGltZXMgWiJdLFs1LDMsIihYXFx0aW1lcyBZKVxcdGltZXMgWiJdLFs0LDQsIlhcXHRpbWVzIFoiXSxbNSw0LCJaIl0sWzYsNCwiWVxcdGltZXMgWiJdLFswLDEsIlxcRGVsdGEiLDJdLFswLDIsImlkIiwyXSxbMCwzLCJpZCJdLFsxLDIsInAiXSxbMSwzLCJxIiwyXSxbNCw1LCJcXERlbHRhIiwyXSxbNSw2LCJwXFx0aW1lcyBxIiwyXSxbNCw2LCJpZCJdLFs3LDksInAiLDJdLFs3LDgsImlkXFx0aW1lcyBwIiwyXSxbNywxMCwiaWRcXHRpbWVzIHEiXSxbMTAsOSwicCJdLFs4LDksInAiLDJdLFsxMSwxMiwicFxcdGltZXMgaWQiLDJdLFsxMSwxMywicSJdLFsxMSwxNCwicVxcdGltZXMgaWQiXSxbMTIsMTMsInEiLDJdLFsxNCwxMywicSIsMl1d
%\begin{tikzcd}[column sep=tiny]
%	& X &&& {X\times Y} \\
%	X & {X\times X} & X && {X\times Y\times X\times Y} & {X\times Y} \\
%	\\
%	& {X\times (Y\times Z)} &&&& {(X\times Y)\times Z} \\
%	{X\times Y} & X & {X\times Z} && {X\times Z} & Z & {Y\times Z}
%	\arrow["id"', from=1-2, to=2-1]
%	\arrow["\Delta"', from=1-2, to=2-2]
%	\arrow["id", from=1-2, to=2-3]
%	\arrow["\Delta"', from=1-5, to=2-5]
%	\arrow["id", from=1-5, to=2-6]
%	\arrow["p", from=2-2, to=2-1]
%	\arrow["q"', from=2-2, to=2-3]
%	\arrow["{p\times q}"', from=2-5, to=2-6]
%	\arrow["{id\times p}"', from=4-2, to=5-1]
%	\arrow["p"', from=4-2, to=5-2]
%	\arrow["{id\times q}", from=4-2, to=5-3]
%	\arrow["{p\times id}"', from=4-6, to=5-5]
%	\arrow["q", from=4-6, to=5-6]
%	\arrow["{q\times id}", from=4-6, to=5-7]
%	\arrow["p"', from=5-1, to=5-2]
%	\arrow["p", from=5-3, to=5-2]
%	\arrow["q"', from=5-5, to=5-6]
%	\arrow["q"', from=5-7, to=5-6]
%\end{tikzcd}  \]
% and the associativity isomorphism
% \[\alpha_{X,Y,Z}:=((id_X \times p_{Y,Z})\times q_{Y,Z}q_{X,Y\times Z})\Delta_{X\times(Y\times Z)}:X\times (Y\times Z)\to (X\times Y)\times Z \]
% and the commutativity isomorphism
% \[\tau_{X,Y}:=(q_{X,Y}\times p_{X,Y})\Delta_{X\times Y}: X\times Y\to Y\times X \]
% are natural in all variables.
%
% A \textbf{one-element object} is an object $I$ with a family of arrows $t_X:X\to I$ of $\mC$ for which $p_{X,I}:X\times I \to I$ is invertible with inverse
% \[X \overset{\Delta}{\to}X\times X\overset{id_X \times t_X}{\longrightarrow} {X\times I}.\]
%\end{mydefinition}


\iffalse
\section{Commutative monads {\color{red} serve?}}
\label{sec: strong and commutative monad}

\iffalse
Here we start by recalling the definition of Kleisli category of a monad.

\begin{mydefinition}
The \textbf{Kleisli category} $\mC_{T}$ of a monad $(T,\mu,\eta)$ on a category $\mC$ has as objects those of $\mC$, and for every $X$ and $Y$ in $\mC_{T}$ the hom-set
\[
	\mC_{T}(X,Y) := \mC(X,T(Y)).
\]
Writing $\freccia{X}{f^\sharp}{T(Y)}$ for the representative in $\mC$ of an arrow $\freccia{X}{f}{Y}$ in $\mC_T$, composition of $f\in \mC_{T}(X,Y)$ and $g\in \mC_{T}(Y,Z)$ is defined as
\[
	(g\circ f)^\sharp := \mu_Z \circ T(g^\sharp) \circ f^\sharp.
\]
\end{mydefinition}

In particular, we have $\id_X^\sharp = \eta_X$, since the naturality of $\eta$ and the unitality equations for $T$ imply
\[
	\mu_Y \circ T(f^\sharp) \circ \eta_X = \mu_Y \circ T(\eta_Y) \circ f^\sharp = f^\sharp,
\]
and this shows the required equations $f \circ \id_X = \id_Y \circ f = f$.
%Given a monad $\freccia{\mC}{T}{\mC}$ on a symmetric monoidal category $(\mC,\otimes,I,\alpha,r,l,\gamma)$, in general  the monoidal product $\otimes$ is not preserved by the monad $T$. More generally, we haven't any canonical choice of arrow $T(X)\otimes T(Y)\rightarrow T(X\otimes Y) $.

\medskip
\fi
A \emph{strength} and a \emph{costrength} for a monad on a monoidal category are structures relating the monad with the tensor product of the category at least \emph{in one direction}. A monad equipped with a strength is called a strong monad.
This notion was introduced by Kock in \cite{Kock72,Kock70} as an alternative description of enriched monads.
Strong monads have been successfully used in computer science, playing a fundamental role in Moggi’s theory of computation \cite{Moggi89,Moggi91}.

We recall these concepts in the following definitions.
\begin{mydefinition}
A \textbf{strong monad} $(T,\mu,\eta,t)$ on a symmetric monoidal category $\mC$ is a monad $(T,\mu,\eta)$ on $\mC$ together with a natural transformation
\[
	\freccia{X\otimes T(Y)}{t_{X,Y}}{T(X\otimes Y)},
\]
called \textbf{strength}, such that the following diagrams commute for all objects $X$, $Y$, $Z$ of $\mC$

% https://q.uiver.app/?q=WzAsMTQsWzAsMiwiSVxcb3RpbWVzIFQoWCkiXSxbMSwyLCJUKElcXG90aW1lcyBYKSJdLFsxLDMsIlQoWCkiXSxbMCw0LCJYXFxvdGltZXMgWVxcb3RpbWVzIFQoWikiXSxbMSw1LCJYXFxvdGltZXMgVChZXFxvdGltZXMgWikiXSxbMCw2LCJYIFxcb3RpbWVzIFReMihZKSJdLFsxLDYsIlQoWFxcb3RpbWVzIFQoWSkpIl0sWzIsNiwiVF4yKFhcXG90aW1lcyBZKSJdLFswLDcsIlhcXG90aW1lcyBUKFkpIl0sWzIsNywiVChYXFxvdGltZXMgWSkiXSxbMCwwLCJYXFxvdGltZXMgWSJdLFsxLDAsIlhcXG90aW1lcyBUKFkpIl0sWzEsMSwiVChYXFxvdGltZXMgWSkiXSxbMiw0LCJUKFhcXG90aW1lcyBZXFxvdGltZXMgWikiXSxbMCwyLCJcXGxhbWJkYV57LTF9X3tUKFgpfSIsMl0sWzAsMSwidF97SSxYfSJdLFsxLDIsIlQoXFxsYW1iZGFeey0xfV9YKSJdLFs1LDYsInRfe1gsVChZKX0iXSxbNiw3LCJUKHRfe1gsWX0pIl0sWzUsOCwiaWRfWFxcb3RpbWVzIFxcbXVfWSIsMl0sWzgsOSwidF97WCxZfSIsMl0sWzcsOSwiXFxtdV97WFxcb3RpbWVzIFl9Il0sWzEwLDExLCJpZF9YXFxvdGltZXMgXFxldGFfWSJdLFsxMCwxMiwiXFxldGFfe1hcXG90aW1lcyBZfSIsMl0sWzExLDEyLCJ0X3tYLFl9Il0sWzQsMTMsInRfe1gsWVxcb3RpbWVzIFp9IiwyXSxbMyw0LCJpZF9YXFxvdGltZXMgdF97WSxafSIsMl0sWzMsMTMsInRfe1hcXG90aW1lcyBZLFp9Il1d

\[\begin{tikzcd}
	{X\otimes Y} & {X\otimes T(Y)}  & {I\otimes T(X)} & {T(I\otimes X)} \\
	& {T(X\otimes Y)} && {T(X)}\\
%	{I\otimes T(X)} & {T(I\otimes X)} \\
%	& {T(X)} \\
	{X\otimes Y\otimes T(Z)} && {T(X\otimes Y\otimes Z)} \\
	& {X\otimes T(Y\otimes Z)} \\
	{X \otimes T^2(Y)} & {T(X\otimes T(Y))} & {T^2(X\otimes Y)} \\
	{X\otimes T(Y)} && {T(X\otimes Y)}
	\arrow["{\lambda^{-1}_{T(X)}}"', from=1-3, to=2-4]
	\arrow["{t_{I,X}}", from=1-3, to=1-4]
	\arrow["{T(\lambda^{-1}_X)}", from=1-4, to=2-4]
	\arrow["{t_{X,T(Y)}}", from=5-1, to=5-2]
	\arrow["{T(t_{X,Y})}", from=5-2, to=5-3]
	\arrow["{\id_X\otimes \mu_Y}"', from=5-1, to=6-1]
	\arrow["{t_{X,Y}}"', from=6-1, to=6-3]
	\arrow["{\mu_{X\otimes Y}}", from=5-3, to=6-3]
	\arrow["{\id_X\otimes \eta_Y}", from=1-1, to=1-2]
	\arrow["{\eta_{X\otimes Y}}"', from=1-1, to=2-2]
	\arrow["{t_{X,Y}}", from=1-2, to=2-2]
	\arrow["{t_{X,Y\otimes Z}}"', from=4-2, to=3-3]
	\arrow["{\id_X\otimes t_{Y,Z}}"', from=3-1, to=4-2]
	\arrow["{t_{X\otimes Y,Z}}", from=3-1, to=3-3]
\end{tikzcd}\]
\end{mydefinition}

\begin{myexample}
	The list monad $\freccia{\Set}{T_{\mathrm{list}}}{\Set}$ is strong. Given two sets $X$ and $Y$, the strength component
	\[
		\freccia{X \times T_{\mathrm{list}}(Y)}{t_{X,Y}}{T_{\mathrm{list}}(X \times Y)}
	\]
	is given by the function assigning to an element $(x,[y_1,\dots,y_m]) $ of $ X\times T_{\mathrm{list}}(Y)$ the element $[(x,y_1),\dots,(x,y_m)]$ of $ T_{\mathrm{list}}(X\times Y)$.
	
	In fact, \emph{any} monad on the cartesian category $\Set$ is strong in a unique way, where the strength can be defined similarly to the strength of the list monad. We refer to  \cite{Kock72,Kock70} for more details.
\end{myexample}

\begin{myremark}
The braiding $\gamma$ of $\mC$ let us define a \textbf{costrength} with components
\[\freccia{T(X)\otimes Y}{t_{X,Y}'}{T(X\otimes Y)}\]
given by
\[t_{X,Y}':=T(\gamma_{Y,X}) \circ t_{Y,X} \circ \gamma_{T(X),Y}.\]
It satisfies axioms that are analogous to those of strength.
\end{myremark}

\begin{mydefinition}
\label{commutative_monad}
A strong monad $(T,\mu,\eta,t)$ on a symmetric monoidal category $\mC$ is said to be \textbf{commutative} if the following 
diagram commutes for every object $X$ and $Y$
\iffalse
\begin{equation}
	\begin{split}
		\label{commutative_monad_diag}
		\xymatrix@+1pc{
			T(X)\otimes T(Y) \ar[r]^{t_{T(X),Y}} \ar[d]_{t'_{X,T(Y)}} & T(T(X)\otimes Y) \ar[r]^{T(t'_{X,Y})} & T^2(X\otimes Y) \ar[d]^{\mu_{X\otimes Y}}\\
			T(X\otimes T(Y))\ar[r]_{T(t_{X,Y})} & T^2(X\otimes Y) \ar[r]_{\mu_{X\otimes Y}} & T(X\otimes Y)
		}
	\end{split}
\end{equation}
\fi
\begin{equation}
	\label{commutative_monad_diag}
% https://q.uiver.app/?q=WzAsNixbMCwwLCJUKFgpXFxvdGltZXMgVChZKSJdLFsxLDAsIlQoVChYKVxcb3RpbWVzIFkpIl0sWzIsMCwiVF4yKFhcXG90aW1lcyBZKSJdLFswLDEsIlQoWFxcb3RpbWVzIFQoWSkpIl0sWzIsMSwiVChYXFxvdGltZXMgWSkiXSxbMSwxLCJUXjIoWFxcb3RpbWVzIFkpIl0sWzAsMSwidF97VChYKSxZfSJdLFsxLDIsIlQodCdfe1gsWX0pIl0sWzAsMywidCdfe1gsVChZKX0iLDJdLFszLDUsIlQodF97WCxZfSkiLDJdLFs1LDQsIlxcbXVfe1hcXG90aW1lcyBZfSIsMl0sWzIsNCwiXFxtdV97WFxcb3RpbWVzIFl9Il1d
\begin{tikzcd}
	{T(X)\otimes T(Y)} & {T(T(X)\otimes Y)} & {T^2(X\otimes Y)} \\
	{T(X\otimes T(Y))} & {T^2(X\otimes Y)} & {T(X\otimes Y)}
	\arrow["{t_{T(X),Y}}", from=1-1, to=1-2]
	\arrow["{T(t'_{X,Y})}", from=1-2, to=1-3]
	\arrow["{t'_{X,T(Y)}}"', from=1-1, to=2-1]
	\arrow["{T(t_{X,Y})}"', from=2-1, to=2-2]
	\arrow["{\mu_{X\otimes Y}}"', from=2-2, to=2-3]
	\arrow["{\mu_{X\otimes Y}}", from=1-3, to=2-3]
\end{tikzcd}
\end{equation}

\end{mydefinition}

\begin{myremark}
	\label{rem:comm_vs_symmon}
	It is well-known that on a symmetric monoidal category, commutative monads are equivalent to symmetric monoidal monads, see \cite[Theorem~2.3]{Kock72} and \cite[Theorem~3.2]{Kock70}, i.e.\ monads internal to the 2-category of symmetric monoidal categories, lax functors and monoidal natural transformations.
	Indeed, if $(T,\mu,\eta,t)$ is a strong commutative monad then the strenght induces a lax monoidal structure on $T$ defined as
	
	
	
	$$\psi_0 := \eta_I:I\to T(I)$$
	and $\psi_{X,Y}:T(X)\otimes T(Y)\to T(X\otimes Y)$ given by the composition
	\[% https://q.uiver.app/#q=WzAsNCxbMCwwLCJUKFgpXFxvdGltZXMgVChZKSJdLFsxLDAsIlQoWFxcb3RpbWVzIFQoWSkpIl0sWzIsMCwiVF4yKFhcXG90aW1lcyBZKSJdLFszLDAsIlQoWFxcb3RpbWVzIFkpIl0sWzAsMSwidF97WCxUKFkpfSJdLFsxLDIsIlQodCdfe1gsWX0pIl0sWzIsMywiXFxtdV97WFxcb3RpbWVzIFl9Il1d
\begin{tikzcd}
	{T(X)\otimes T(Y)} & {T(X\otimes T(Y))} & {T^2(X\otimes Y)} & {T(X\otimes Y)}
	\arrow["{t_{X,T(Y)}}", from=1-1, to=1-2]
	\arrow["{T(t'_{X,Y})}", from=1-2, to=1-3]
	\arrow["{\mu_{X\otimes Y}}", from=1-3, to=1-4]
\end{tikzcd}\]
moreover, the conditions to verify that $\mu$ and $\eta$ provide monoidal natural transformations reduce to the followings:
\[% https://q.uiver.app/#q=WzAsNCxbMCwwLCJYXFxvdGltZXMgWSJdLFsyLDAsIlQoWClcXG90aW1lcyBUKFkpIl0sWzAsMSwiWFxcb3RpbWVzIFkiXSxbMiwxLCJUKFhcXG90aW1lcyBZKSJdLFswLDIsIiIsMCx7ImxldmVsIjoyLCJzdHlsZSI6eyJoZWFkIjp7Im5hbWUiOiJub25lIn19fV0sWzAsMSwiXFxldGFfWFxcb3RpbWVzIFxcZXRhX1kiXSxbMiwzLCJcXGV0YV97WFxcb3RpbWVzIFl9IiwyXSxbMSwzLCJcXHBzaV97WCxZfSJdXQ==
\begin{tikzcd}
	{X\otimes Y} && {T(X)\otimes T(Y)} \\
	{X\otimes Y} && {T(X\otimes Y)}
	\arrow["{\eta_X\otimes \eta_Y}", from=1-1, to=1-3]
	\arrow[Rightarrow, no head, from=1-1, to=2-1]
	\arrow["{\psi_{X,Y}}", from=1-3, to=2-3]
	\arrow["{\eta_{X\otimes Y}}"', from=2-1, to=2-3]
\end{tikzcd}\]
\[% https://q.uiver.app/#q=WzAsNSxbMCwwLCJUKFQoWCkpXFxvdGltZXMgVChUKFkpKSJdLFsyLDAsIlQoWClcXG90aW1lcyBUKFkpIl0sWzAsMSwiWFxcb3RpbWVzIFkiXSxbMiwxLCJUKFhcXG90aW1lcyBZKSJdLFsxLDEsIlQoVChYXFxvdGltZXMgWSkpIl0sWzAsMiwiXFxwc2lfe1QoWClcXG90aW1lcyBUKFkpfSIsMl0sWzAsMSwiXFxtdV9YXFxvdGltZXMgXFxtdV9ZIl0sWzEsMywiXFxwc2lfe1gsWX0iXSxbNCwzLCJcXG11X3tYXFxvdGltZXMgWX0iLDJdLFsyLDQsIlQoXFxwc2lfe1gsWX0pIiwyXV0=
\begin{tikzcd}
	{T(T(X))\otimes T(T(Y))} && {T(X)\otimes T(Y)} \\
	{X\otimes Y} & {T(T(X\otimes Y))} & {T(X\otimes Y)}
	\arrow["{\mu_X\otimes \mu_Y}", from=1-1, to=1-3]
	\arrow["{\psi_{T(X)\otimes T(Y)}}"', from=1-1, to=2-1]
	\arrow["{\psi_{X,Y}}", from=1-3, to=2-3]
	\arrow["{T(\psi_{X,Y})}"', from=2-1, to=2-2]
	\arrow["{\mu_{X\otimes Y}}"', from=2-2, to=2-3]
\end{tikzcd}\]
Vice versa, if one takes a symmetric monoidal monad $(T,\psi_0,\psi,\mu,\eta)$, then $\psi_0=\eta_I$ and one can define a strenght and a costrenght
\[t_{X,Y}:X\otimes T(Y)\to T(X\otimes Y)\qquad \freccia{T(X)\otimes Y}{t_{X,Y}'}{T(X\otimes Y)}\]
respectively as $t_{X,Y}:=\psi\circ(\eta_X \otimes id)   $ and $t'_{X,Y}:=\psi\circ(id \otimes\eta_Y)$
which provide always a commutative monad $(T,\mu,\sigma)$. See also \cite{GOUBAULT_LARRECQ_2008} for details.
\end{myremark}
\fi
%%%%%%%%%%%%%%%%%%%%%%%%%%%%%%%%%%%%%%%%%%%%%%%%%%%%%%%%%%%%%%%%%%%%%%%%%%%%%%%%%%%%%%%%%%%%%%%%%%%%%%%%%%%%%%%%%%%%%%%%%%%%%%%%%%%%%%%%%%%%%%%%%%%%%%%%%%%%
%%%%%%%%%%%%%%%%%%%%%%%%%%%%%%%%%%%%%%%%%%%%%%%%%%%%%%%%%%%%%%%%%%%%%%%%%%%%%%%%%%%%%%%%%%%%%%%%%%%%%%%%%%%%%%%%%%%%%%%%%%%%%%%%%%%%%%%%%%%%%%%%%%%%%%%%%%%%%%%%%%%%%%%%%%%%%%%%%%%%%%
\section{More on cogs-monoidal categories}\label{section: appendix dual results}
\begin{mydefinition}\label{def cogs monoidal functor}
	For cogs-monoidal categories $\mC$ and $\mD$, a functor $\freccia{\mC}{F}{\mD}$ equipped with a lax symmetric monoidal structure
   \[
				\freccia{\otimes \circ \, (F\times F)}{\psi}{F\circ \otimes}, \qquad \freccia{I}{\psi_0}{F(I)} 
			\]
is \textbf{coaffine}  if the following diagram commutes for all $X$ in $\mC$
\begin{equation}\label{diagram: lax coaffine}
\begin{tikzcd}[column sep=tiny]
	F(X) && {F(I)} \\
& I
\arrow["{F(\cobang_X)}"', from=1-3, to=1-1]
\arrow["{\cobang_{FX}}", from=2-2, to=1-1]
\arrow["{\psi_0}"', from=2-2, to=1-3]
\end{tikzcd}
\end{equation}
and it is \textbf{corelevant} if the following diagram commutes for all $X$ in $\mC$

\begin{equation}\label{diagram: lax corelevant}
\begin{tikzcd}[column sep=tiny]
	{F(X)} && {F(X\otimes X)} \\
	& {F(X)\otimes F(X)}
	\arrow["{F(\Delta_X)}"', from=1-3, to=1-1]
	\arrow["{\Delta_{FX}}", from=2-2, to=1-1]
	\arrow["{\psi_{X,X}}"', from=2-2, to=1-3]
\end{tikzcd}
\end{equation}
A functor which is both coaffine and corelevant is \textbf{cogs-monoidal}.
\end{mydefinition}

\iffalse
begin{mydefinition}\label{dfn: corestriction}
	A \textbf{corestriction category} is a cogs-monoidal category such that every arrow is cocopyable.
\end{mydefinition}

\begin{mydefinition}\label{dfn: coMarkov}
	A \textbf{coMarkov category} is a cogs-monoidal category such that every arrow is codiscardable.
\end{mydefinition}
\fi

\begin{myproposition}\label{prop: Kleisli of coaffine monads}
    Let $(T,\mu, \eta,c, u)$ be a symmetric monoidal monad on a cogs-monoidal category $\mC$. If $\mC$ has coprojections then so does the Kleisli category $\mC_T$.
	\begin{proof}
If $I$ is initial then it is so also in $\mC_T$ since the functor $\mC\to \mC_T$ preserves colimits.
    \end{proof}
\end{myproposition}
\begin{myremark}
	In the above proposition it is not necessary to assume the monad to be coaffine. Indeed,
	if $I$ is initial then any symmetric monoidal monad is trivially coaffine.
\end{myremark}



\begin{myproposition}\label{prop: Kleisli of corelevant monads}
	Let $(T,\mu, \eta,c, u)$ be a corelevant monad on a cogs-monoidal category $\mC$. If $\mC$ has codiagonals then so does the Kleisli category $\mC_T$.
\begin{proof}
	We have to prove that for every $f:X\to Y$ in $\mC_T$, which corresponds to an arrow $f:X\to T(Y)$ in $\mC$, it holds that $f\circ^\sharp\Delta^\sharp_X= \Delta^\sharp_Y \circ^\sharp (f\otimes_T f)$ in $\mC_T$. 
	Indeed
	\begin{align}
		f\circ^\sharp \Delta^\sharp_X &= f\circ  \Delta_X \tag*{}\\
		&= \Delta_{T(Y)}\circ (f\otimes f) \tag{$\mC$ codiagonals}\\
		&= T(\Delta_Y) \circ c_{Y,Y}\circ (f\otimes f) \tag{$T$ corelevant}\\
		&= \Delta^\sharp_Y\circ^\sharp (f\otimes_T f) \tag*{}
	\end{align}
\end{proof}
\end{myproposition}

%\cipr{La coaffinit\`a torna a servire nella versione arricchita}

\begin{mydefinition}\label{def: oplax cocartesian functor}
	Let $\mC$ and $\mD$
	%$(\mC,\otimes_{\mC},I_{\mC},\nabla^{\mC},!^{\mC})$ and $(\mD,\otimes_{\mD},I_{\mD},\nabla^{\mD},!^{\mD})$ 
	be oplax cocartesian categories and  
	$\freccia{\mC}{F}{\mD}$ a preorder-enriched lax symmetric monoidal functor with structure arrows $\psi, \psi_0$.
	The $F$ is called
		\textbf{colax coaffine} if the following inequality holds 
\[
% https://q.uiver.app/#q=WzAsMyxbMCwwLCJGKFgpIl0sWzIsMCwiRihJKSJdLFsxLDEsIkkiXSxbMCwyLCIhX3tGKFgpfSIsMl0sWzIsMSwiXFxwc2lfMCIsMl0sWzAsMSwiRighX1gpIl0sWzIsNSwiIiwxLHsibGV2ZWwiOjEsInN0eWxlIjp7Im5hbWUiOiJhZGp1bmN0aW9uIn19XV0=
\begin{tikzcd}
{F(X)} && {F(I)} \\
& I
\arrow[""{name=0, anchor=center, inner sep=0}, "{F(\cobang_X)}"', from=1-3, to=1-1]
\arrow["{\cobang_{F(X)}}", from=2-2, to=1-1]
\arrow["{\psi_0}"', from=2-2, to=1-3]
\arrow["\le"{anchor=center}, draw=none, from=2-2, to=0]
\end{tikzcd}
\]
and it is called \textbf{colax corelevant}  if following inequality holds
\[% https://q.uiver.app/#q=WzAsMyxbMCwwLCJGKFgpIl0sWzIsMCwiRihYXFxvdGltZXMgWCkiXSxbMSwxLCJGKFgpXFxvdGltZXMgRihYKSJdLFswLDIsIlxcbmFibGFfe0YoWCl9IiwyXSxbMiwxLCJcXHBzaV97WCxYfSIsMl0sWzAsMSwiRihcXG5hYmxhX1gpIl0sWzIsNSwiXFxsZSIsMSx7ImxldmVsIjoxLCJzdHlsZSI6eyJib2R5Ijp7Im5hbWUiOiJub25lIn0sImhlYWQiOnsibmFtZSI6Im5vbmUifX19XV0=
\begin{tikzcd}
{F(X)} && {F(X\otimes X)} \\
& {F(X)\otimes F(X)}
\arrow[""{name=0, anchor=center, inner sep=0}, "{F(\Delta_X)}"', from=1-3, to=1-1]
\arrow["{\Delta_{F(X)}}", from=2-2, to=1-1]
\arrow["{\psi_{X,X}}"', from=2-2, to=1-3]
\arrow["\le"{description}, draw=none, from=2-2, to=0]
\end{tikzcd}\]
If $F$ is both colax affine and colax relevant it is called \textbf{colax cogs-monoidal}.
\end{mydefinition}

\begin{myproposition}\label{prop: Kleisli colax cocartesian}
	Let $(T,\mu, \eta,c, u)$ be a colax cogs-monoidal monad on a preorder-enriched cogs-monoidal category $\mC$. 
	If $\mC$ is oplax cocartesian then so is the Kleisli category $\mC_T$.
\end{myproposition}
\begin{proof}
	$\mC_T$ is cogs-monoidal thanks to Proposition \ref{prop:Kleisli cogs} and inherits the preorder-enrichment from $\mC$. We now prove that it is oplax cocartesian.

	We first show that for every arrow $f:I\to T(X)$ in $\mC_T$ it holds that $f\le \cobang^\sharp_X$.
	Since $\mC$ is oplax cocartesian we have that $f\le \cobang_{T(X)}$, and because $T$ il colax coaffine it implies
	that $\cobang_{T(X)}\le T(\cobang_X)\circ \eta_I= \cobang^\sharp_X$.

	It remains to prove that for every $f:X\to Y$ in $\mC_T$, which corresponds to an arrow $f:X\to T(Y)$ in $\mC$, it holds that $f\circ^\sharp \Delta^\sharp_X\le \Delta^\sharp_Y \circ^\sharp (f\otimes_T f)$ in $\mC_T$. Indeed
	\begin{align}
		f\circ^\sharp \Delta^\sharp_X &= f \circ\Delta_X \tag*{}\\
		&\le \Delta_{T(Y)}\circ (f\otimes f) \tag{$\mC$ oplax cocartesian}\\
		&\le T(\Delta_Y)\circ c_{Y,Y}\circ (f\otimes f) \tag{$T$ colax corelevant}\\
		&= \Delta^\sharp_Y\circ^\sharp (f\otimes_T f) \tag*{}
	\end{align}
	\end{proof}

\iffalse
\section{A string diagram presentation of restriction categories}\label{sec: restricats}

\begin{myproposition}\label{def: C-functions is a restriction category}
Let $\mC$ be a gs-monoidal category. Then the category $\pfn{\mC}$ of $\mC$-functional arrows is a restriction category
 with respect to the structure given by $\dom (-)$.
\end{myproposition}
\begin{IEEEproof} 
By \cite{Cockett02}, we have to verify the following equations, that we also present graphically by making explicit the definition of $\dom (-)$:
\begin{enumerate}
    \item $f\circ \dom (f)=f$ for any arrow $\freccia{A}{f}{B}$
\begin{figure}[H]
\centering
\begin{tikzpicture}[scale=0.60, transform shape]
	\begin{pgfonlayer}{nodelayer}
		\node [style=none] (31) at (4.75, 1.5) {};
		\node [style=none] (32) at (6.25, 2) {};
		\node [style=none] (33) at (6.25, 1) {};
		\node [style=none] (34) at (7, 1) {};
		\node [style=nodonero] (35) at (5.75, 1.5) {};
		\node [style=none] (36) at (5.25, 1.75) {A};
		\node [style=box] (37) at (7, 1) {$f$};
		\node [style=nodonero] (38) at (8.25, 1) {};
		\node [style=none] (39) at (8.25, 2) {};
		\node [style=none] (40) at (8.25, 2.25) {B};
		\node [style=none] (41) at (7.75, 0.75) {B};
		\node [style=none] (42) at (9.75, 1.5) {};
		\node [style=none] (43) at (10.25, 1.75) {A};
		\node [style=none] (44) at (9, 1.5) {$=$};
		\node [style=box] (45) at (7, 2) {$f$};
		\node [style=box] (46) at (10.75, 1.5) {$f$};
		\node [style=none] (47) at (11.75, 1.5) {};
		\node [style=none] (48) at (11.25, 1.75) {B};
	\end{pgfonlayer}
	\begin{pgfonlayer}{edgelayer}
		\draw (31.center) to (35);
		\draw [bend left, looseness=1.25] (35) to (32.center);
		\draw [bend right] (35) to (33.center);
		\draw (33.center) to (34.center);
		\draw (34.center) to (37.center);
		\draw (37) to (38);
		\draw (32.center) to (45);
		\draw (45) to (39.center);
		\draw (42.center) to (46);
		\draw (46) to (47.center);
	\end{pgfonlayer}
\end{tikzpicture}
\end{figure}
 In fact, $(B \otimes {!B})\circ(f \otimes f)\circ \nabla_A =\mbox{ [$f$ functional] } (B \otimes {!B})\circ \nabla_B \circ f = id_B \circ f = f$.
    \item $\dom (f) \circ \dom (g)=\dom (g) \circ \dom (f) $ if the source of $g$ is the same of $f$
\begin{figure}[H]
\centering
\begin{tikzpicture}[scale=0.60, transform shape]
	\begin{pgfonlayer}{nodelayer}
		\node [style=none] (31) at (4.75, 1.5) {};
		\node [style=none] (32) at (6.25, 2) {};
		\node [style=none] (33) at (6.25, 1) {};
		\node [style=none] (34) at (7, 1) {};
		\node [style=nodonero] (35) at (5.75, 1.5) {};
		\node [style=none] (36) at (5.25, 1.75) {A};
		\node [style=box] (37) at (7, 1) {$g$};
		\node [style=nodonero] (38) at (8.25, 1) {};
		\node [style=none] (39) at (8.25, 2) {};
		\node [style=none] (40) at (7.25, 2.25) {A};
		\node [style=none] (41) at (7.75, 0.75) {C};
		\node [style=none] (44) at (12, 1.5) {$=$};
		\node [style=none] (50) at (8.75, 2.5) {};
		\node [style=none] (51) at (8.75, 1.5) {};
		\node [style=none] (52) at (9.5, 1.5) {};
		\node [style=nodonero] (53) at (8.25, 2) {};
		\node [style=box] (55) at (9.5, 1.5) {$f$};
		\node [style=nodonero] (56) at (10.75, 1.5) {};
		\node [style=none] (57) at (10.75, 2.5) {};
		\node [style=none] (58) at (10.75, 2.75) {A};
		\node [style=none] (59) at (10.25, 1.25) {B};
		\node [style=none] (60) at (13, 1.5) {};
		\node [style=none] (61) at (14.5, 2) {};
		\node [style=none] (62) at (14.5, 1) {};
		\node [style=none] (63) at (15.25, 1) {};
		\node [style=nodonero] (64) at (14, 1.5) {};
		\node [style=none] (65) at (13.5, 1.75) {A};
		\node [style=box] (66) at (15.25, 1) {$f$};
		\node [style=nodonero] (67) at (16.5, 1) {};
		\node [style=none] (68) at (16.5, 2) {};
		\node [style=none] (69) at (15.5, 2.25) {A};
		\node [style=none] (70) at (16, 0.75) {B};
		\node [style=none] (71) at (17, 2.5) {};
		\node [style=none] (72) at (17, 1.5) {};
		\node [style=none] (73) at (17.75, 1.5) {};
		\node [style=nodonero] (74) at (16.5, 2) {};
		\node [style=box] (75) at (17.75, 1.5) {$g$};
		\node [style=nodonero] (76) at (19, 1.5) {};
		\node [style=none] (77) at (19, 2.5) {};
		\node [style=none] (78) at (19, 2.75) {A};
		\node [style=none] (79) at (18.5, 1.25) {C};
	\end{pgfonlayer}
	\begin{pgfonlayer}{edgelayer}
		\draw (31.center) to (35);
		\draw [bend left, looseness=1.25] (35) to (32.center);
		\draw [bend right] (35) to (33.center);
		\draw (33.center) to (34.center);
		\draw (34.center) to (37.center);
		\draw (37) to (38);
		\draw (32.center) to (39.center);
		\draw [bend left, looseness=1.25] (53) to (50.center);
		\draw [bend right] (53) to (51.center);
		\draw (51.center) to (52.center);
		\draw (52.center) to (55.center);
		\draw (55) to (56);
		\draw (50.center) to (57.center);
		\draw (60.center) to (64);
		\draw [bend left, looseness=1.25] (64) to (61.center);
		\draw [bend right] (64) to (62.center);
		\draw (62.center) to (63.center);
		\draw (63.center) to (66);
		\draw (66) to (67);
		\draw (61.center) to (68.center);
		\draw [bend left, looseness=1.25] (74) to (71.center);
		\draw [bend right] (74) to (72.center);
		\draw (72.center) to (73.center);
		\draw (73.center) to (75);
		\draw (75) to (76);
		\draw (71.center) to (77.center);
	\end{pgfonlayer}
\end{tikzpicture}
\end{figure}
\andrea{Proof missing: see comment below}
    \item $\dom (g\circ \dom (f))=\dom (g) \circ \dom (f) $ if the source of $g$ is the same of $f$
\begin{figure}[H]
\centering
\begin{tikzpicture}[scale=0.60, transform shape]
	\begin{pgfonlayer}{nodelayer}
		\node [style=none] (44) at (10, -2.5) {$=$};
		\node [style=none] (60) at (11, -2.5) {};
		\node [style=none] (61) at (12.5, -2) {};
		\node [style=none] (62) at (12.5, -3) {};
		\node [style=none] (63) at (13.25, -3) {};
		\node [style=nodonero] (64) at (12, -2.5) {};
		\node [style=none] (65) at (11.5, -2.25) {A};
		\node [style=box] (66) at (13.25, -3) {$f$};
		\node [style=nodonero] (67) at (14.5, -3) {};
		\node [style=none] (68) at (14.5, -2) {};
		\node [style=none] (69) at (13.5, -1.75) {A};
		\node [style=none] (70) at (14, -3.25) {B};
		\node [style=none] (71) at (15, -1.5) {};
		\node [style=none] (72) at (15, -2.5) {};
		\node [style=none] (73) at (15.75, -2.5) {};
		\node [style=nodonero] (74) at (14.5, -2) {};
		\node [style=box] (75) at (15.75, -2.5) {$g$};
		\node [style=nodonero] (76) at (17, -2.5) {};
		\node [style=none] (77) at (17, -1.5) {};
		\node [style=none] (78) at (17, -1.25) {A};
		\node [style=none] (79) at (16.5, -2.75) {C};
		\node [style=none] (80) at (4, -2) {};
		\node [style=none] (81) at (5.5, -1.5) {};
		\node [style=none] (82) at (5.5, -2.5) {};
		\node [style=none] (83) at (6.25, -2.5) {};
		\node [style=nodonero] (84) at (5, -2) {};
		\node [style=none] (85) at (4.5, -1.75) {A};
		\node [style=none] (88) at (8.75, -1.5) {};
		\node [style=none] (92) at (6.25, -2.5) {};
		\node [style=none] (93) at (6.75, -2) {};
		\node [style=none] (94) at (6.75, -3) {};
		\node [style=none] (95) at (7.5, -3) {};
		\node [style=nodonero] (96) at (6.25, -2.5) {};
		\node [style=box] (97) at (7.5, -3) {$f$};
		\node [style=nodonero] (98) at (8.75, -3) {};
		\node [style=none] (101) at (8.25, -3.25) {B};
		\node [style=none] (103) at (8.25, -1.25) {A};
		\node [style=box] (104) at (7.5, -2) {$g$};
		\node [style=nodonero] (105) at (8.75, -2) {};
		\node [style=none] (106) at (8.25, -2.25) {C};
	\end{pgfonlayer}
	\begin{pgfonlayer}{edgelayer}
		\draw (60.center) to (64);
		\draw [bend left, looseness=1.25] (64) to (61.center);
		\draw [bend right] (64) to (62.center);
		\draw (62.center) to (63.center);
		\draw (63.center) to (66);
		\draw (66) to (67);
		\draw (61.center) to (68.center);
		\draw [bend left, looseness=1.25] (74) to (71.center);
		\draw [bend right] (74) to (72.center);
		\draw (72.center) to (73.center);
		\draw (73.center) to (75);
		\draw (75) to (76);
		\draw (71.center) to (77.center);
		\draw (80.center) to (84);
		\draw [bend left, looseness=1.25] (84) to (81.center);
		\draw [bend right] (84) to (82.center);
		\draw (82.center) to (83.center);
		\draw (81.center) to (88.center);
		\draw [bend left, looseness=1.25] (96) to (93.center);
		\draw [bend right] (96) to (94.center);
		\draw (94.center) to (95.center);
		\draw (95.center) to (97.center);
		\draw (97) to (98);
		\draw (93.center) to (104);
		\draw (104) to (105);
	\end{pgfonlayer}
\end{tikzpicture}
\end{figure}
\andrea{Proof missing: see comment below}

    \item $\dom (g) \circ f=f \circ \dom (g \circ f)$ if the source of $g$ is equal to the target of $f$
\begin{figure}[H]
\centering
\begin{tikzpicture}[scale=0.60, transform shape]
	\begin{pgfonlayer}{nodelayer}
		\node [style=none] (32) at (6.25, 2) {};
		\node [style=none] (33) at (6.25, 1) {};
		\node [style=none] (34) at (7, 1) {};
		\node [style=nodonero] (35) at (5.75, 1.5) {};
		\node [style=none] (36) at (4, 1.75) {A};
		\node [style=box] (37) at (7, 1) {$g$};
		\node [style=nodonero] (38) at (8.25, 1) {};
		\node [style=none] (39) at (8.25, 2) {};
		\node [style=none] (40) at (5.25, 1.75) {B};
		\node [style=none] (41) at (7.75, 0.75) {C};
		\node [style=none] (44) at (9, 1.5) {$=$};
		\node [style=none] (49) at (9.75, 1.5) {};
		\node [style=none] (50) at (11.25, 2) {};
		\node [style=none] (51) at (11.25, 1) {};
		\node [style=none] (52) at (12, 1) {};
		\node [style=nodonero] (53) at (10.75, 1.5) {};
		\node [style=none] (54) at (10.25, 1.75) {A};
		\node [style=box] (55) at (12, 1) {$f$};
		\node [style=nodonero] (56) at (14.75, 1) {};
		\node [style=none] (57) at (14.75, 2) {};
		\node [style=none] (58) at (13.25, 2.25) {B};
		\node [style=none] (59) at (12.75, 0.75) {B};
		\node [style=box] (60) at (12, 2) {$f$};
		\node [style=box] (61) at (13.5, 1) {$g$};
		\node [style=none] (62) at (14.25, 0.75) {C};
		\node [style=box] (63) at (4.5, 1.5) {$f$};
		\node [style=none] (64) at (3.25, 1.5) {};
		\node [style=none] (65) at (7.75, 2.25) {B};
	\end{pgfonlayer}
	\begin{pgfonlayer}{edgelayer}
		\draw [bend left, looseness=1.25] (35) to (32.center);
		\draw [bend right] (35) to (33.center);
		\draw (33.center) to (34.center);
		\draw (34.center) to (37.center);
		\draw (37) to (38);
		\draw (49.center) to (53);
		\draw [bend left, looseness=1.25] (53) to (50.center);
		\draw [bend right] (53) to (51.center);
		\draw (51.center) to (52.center);
		\draw (52.center) to (55);
		\draw (55) to (56);
		\draw (50.center) to (60);
		\draw (60) to (57.center);
		\draw (64.center) to (63);
		\draw (63) to (35);
		\draw (32.center) to (39.center);
	\end{pgfonlayer}
\end{tikzpicture}
\end{figure}
\andrea{Proof missing: see comment below}
        \end{enumerate}
\end{IEEEproof}

Clearly, axioms $2$ and $3$ are satisfied by any gs-monoidal category. 
\tob{This is very confusing: why do we list axioms 2 and 3 as conditions if they're automatically satisfied? (I had turned this into a proposition from a definition; maybe this was actually supposed to be the defn of restriction category, as the section title indicates? But it's clearly not since a general restriction category doesn't even have a monoidal structure, so also the section title seems problematic to me)}
\andrea{I tried to turn the above into a meaningful statement. The missing proofs can be added if we decide to keep the Proposition, which is not original according to the Tobias comments that I've found along the paper.}   
\fi
%\section{Omitted proofs, Section~\ref{sec:on gs}}
%
%For simplicity of notation, we assume without loss of generality that $\mC$ is strict.

%\tob{I think that most of these proofs would be easier to follow with consistent use of string diagrams, but I don't know if either of us would have the patience to draw all of them ;)}

%\subsection{Proof of Proposition \ref{prop: ! and nabla are total and functional}}\label{proof: prop: ! and nabla are total and functional}
%The proofs of these statements are straightforward. We just show that both $!$ and $\nabla$ are $\mC$-total and $\mC$-functional: every arrow $!_A$ is $\mC$-total because $!_{\I}!_A=\id_I!_A$ since $!_{\I}=\id_{\I}$ by  Definition \ref{def gs-monoidal cat}. Moreover $!_A$ is $\mC$-functional because
%\[
%	(!_A\ox {}!_A)\nabla_A=(!_A\ox \id_I)(\id_A\ox \,!_A)\nabla_A = {}!_A = \nabla_{\I} !_A
%\]
%by strictness of $\mC$ and the axioms of Definition \ref{def gs-monoidal cat}.
%The same calculation also shows that every duplicator $\nabla_A$ is $\mC$-total.
%It is $\mC$-functional by the first monoidal multiplicativity axiom combined with the commutative comonoid equations.

	
%\subsection{Proof of Proposition~\ref{prop:dom is functionional and dom(f)=id iff f total}}\label{proof:prop:dom is functionional and dom(f)=id iff f total}
%
%\begin{enumerate}
%	\item By Proposition \ref{prop: ! and nabla are total and functional}, $\mC$-functional arrows are closed under composition and tensor product. Since $!$ and $\nabla$ are $\mC$-functional arrows and $f$ is $\mC$-functional by hypothesis, we have that $\dom (f)= (\id_A\otimes \,!_Bf)\nabla_A$ is $\mC$-functional.
%
%	\item If $f$ is $\mC$-total, then $!_Bf={}!_A$, hence
%\[
%	\dom (f)=(\id_A\otimes \,!_Bf)\nabla_A=(\id_A\otimes \,!_A)\nabla_A=\id_A
%\]
%by the second axiom of gs-monoidal categories. Conversely, if $\dom (f)=\id_A$, then $!_A\dom(f)= {}!_A$, hence $!_A(\id_A\otimes \,!_Bf)\nabla_A={}!_A$. 
%Since  $!_A(\id_A\otimes \,!_Bf)\nabla_A={}!_Bf(!_A\otimes \id_A)\nabla_A={}!_Bf$, we can conclude that $!_Bf={}!_A$, i.e., that $f$ is $\mC$-total.
%\end{enumerate}

%\comment{
%\subsection{Proof of Proposition \ref{thm: C-funtions is a restriction category}}\label{proof: thm: C-funtions is a restriction category}
%
%\begin{enumerate}
%\item Using that $f$ is $\mC$-functional and applying the second axiom of gs-monoidal categories gives
%\begin{figure}[H]
%\centering
%\begin{tikzpicture}[scale=0.60, transform shape]
%	\begin{pgfonlayer}{nodelayer}
%		\node [style=none] (31) at (4.75, 1.5) {};
%		\node [style=none] (32) at (6.25, 2) {};
%		\node [style=none] (33) at (6.25, 1) {};
%		\node [style=none] (34) at (7, 1) {};
%		\node [style=nodonero] (35) at (5.75, 1.5) {};
%		\node [style=none] (36) at (5.25, 1.75) {A};
%		\node [style=box] (37) at (7, 1) {$f$};
%		\node [style=nodonero] (38) at (8.25, 1) {};
%		\node [style=none] (39) at (8.25, 2) {};
%		\node [style=none] (40) at (8.25, 2.25) {B};
%		\node [style=none] (41) at (7.75, 0.75) {B};
%		\node [style=none] (42) at (16, 1.5) {};
%		\node [style=none] (43) at (16.5, 1.75) {A};
%		\node [style=none] (44) at (9, 1.5) {$=$};
%		\node [style=box] (45) at (7, 2) {$f$};
%		\node [style=box] (46) at (17, 1.5) {$f$};
%		\node [style=none] (47) at (18, 1.5) {};
%		\node [style=none] (48) at (17.5, 1.75) {B};
%		\node [style=none] (50) at (12.75, 2) {};
%		\node [style=none] (51) at (12.75, 1) {};
%		\node [style=nodonero] (53) at (12.25, 1.5) {};
%		\node [style=none] (54) at (11.75, 1.75) {B};
%		\node [style=nodonero] (56) at (14, 1) {};
%		\node [style=none] (57) at (14, 2) {};
%		\node [style=box] (60) at (11, 1.5) {$f$};
%		\node [style=none] (61) at (10, 1.5) {};
%		\node [style=none] (62) at (10.5, 1.75) {A};
%		\node [style=none] (63) at (15, 1.5) {$=$};
%	\end{pgfonlayer}
%	\begin{pgfonlayer}{edgelayer}
%		\draw (31.center) to (35);
%		\draw [bend left, looseness=1.25] (35) to (32.center);
%		\draw [bend right] (35) to (33.center);
%		\draw (33.center) to (34.center);
%		\draw (34.center) to (37.center);
%		\draw (37) to (38);
%		\draw (32.center) to (45);
%		\draw (45) to (39.center);
%		\draw (42.center) to (46);
%		\draw (46) to (47.center);
%		\draw [bend left, looseness=1.25] (53) to (50.center);
%		\draw [bend right] (53) to (51.center);
%		\draw (61.center) to (60);
%		\draw (60) to (53);
%		\draw (50.center) to (57.center);
%		\draw (51.center) to (56);
%	\end{pgfonlayer}
%\end{tikzpicture}
%\end{figure}
%\noindent
%\item Employing the associativity and commutativity axioms of $\nabla$, we have the chain of equalities
%\begin{figure}[H]
%\centering
%\begin{tikzpicture}[scale=0.60, transform shape]
%	\begin{pgfonlayer}{nodelayer}
%		\node [style=none] (31) at (4.75, 1.5) {};
%		\node [style=none] (32) at (6.25, 2) {};
%		\node [style=none] (33) at (6.25, 1) {};
%		\node [style=none] (34) at (7, 1) {};
%		\node [style=nodonero] (35) at (5.75, 1.5) {};
%		\node [style=none] (36) at (5.25, 1.75) {A};
%		\node [style=box] (37) at (7, 1) {$g$};
%		\node [style=nodonero] (38) at (8.25, 1) {};
%		\node [style=none] (39) at (8.25, 2) {};
%		\node [style=none] (40) at (7.25, 2.25) {A};
%		\node [style=none] (41) at (7.75, 0.75) {C};
%		\node [style=none] (44) at (12, 1.75) {$=$};
%		\node [style=none] (50) at (8.75, 2.5) {};
%		\node [style=none] (51) at (8.75, 1.5) {};
%		\node [style=none] (52) at (9.5, 1.5) {};
%		\node [style=nodonero] (53) at (8.25, 2) {};
%		\node [style=box] (55) at (9.5, 1.5) {$f$};
%		\node [style=nodonero] (56) at (10.75, 1.5) {};
%		\node [style=none] (57) at (10.75, 2.5) {};
%		\node [style=none] (58) at (10.75, 2.75) {A};
%		\node [style=none] (59) at (10.25, 1.25) {B};
%		\node [style=none] (60) at (19, 1.5) {};
%		\node [style=none] (61) at (20.5, 2) {};
%		\node [style=none] (62) at (20.5, 1) {};
%		\node [style=none] (63) at (21.25, 1) {};
%		\node [style=nodonero] (64) at (20, 1.5) {};
%		\node [style=none] (65) at (19.5, 1.75) {A};
%		\node [style=box] (66) at (21.25, 1) {$f$};
%		\node [style=nodonero] (67) at (22.5, 1) {};
%		\node [style=none] (68) at (22.5, 2) {};
%		\node [style=none] (69) at (21.5, 2.25) {A};
%		\node [style=none] (70) at (22, 0.75) {B};
%		\node [style=none] (71) at (23, 2.5) {};
%		\node [style=none] (72) at (23, 1.5) {};
%		\node [style=none] (73) at (23.75, 1.5) {};
%		\node [style=nodonero] (74) at (22.5, 2) {};
%		\node [style=box] (75) at (23.75, 1.5) {$g$};
%		\node [style=nodonero] (76) at (25, 1.5) {};
%		\node [style=none] (77) at (25, 2.5) {};
%		\node [style=none] (78) at (25, 2.75) {A};
%		\node [style=none] (79) at (24.5, 1.25) {C};
%		\node [style=none] (80) at (13, 2) {};
%		\node [style=none] (81) at (14.5, 2.5) {};
%		\node [style=none] (82) at (14.5, 1.5) {};
%		\node [style=nodonero] (84) at (14, 2) {};
%		\node [style=none] (85) at (13.5, 2.25) {A};
%		\node [style=none] (89) at (16.5, 2.75) {A};
%		\node [style=none] (100) at (14.5, 1.5) {};
%		\node [style=none] (101) at (15, 2) {};
%		\node [style=none] (102) at (15, 1) {};
%		\node [style=none] (103) at (15.75, 1) {};
%		\node [style=nodonero] (104) at (14.5, 1.5) {};
%		\node [style=box] (105) at (15.75, 1) {$f$};
%		\node [style=nodonero] (106) at (17, 1) {};
%		\node [style=none] (107) at (17, 2) {};
%		\node [style=none] (108) at (16.5, 1.75) {C};
%		\node [style=none] (109) at (16.5, 0.75) {B};
%		\node [style=none] (110) at (17, 2.5) {};
%		\node [style=box] (111) at (15.75, 2) {$g$};
%		\node [style=none] (112) at (18, 1.75) {$=$};
%		\node [style=nodonero] (113) at (17, 2) {};
%	\end{pgfonlayer}
%	\begin{pgfonlayer}{edgelayer}
%		\draw (31.center) to (35);
%		\draw [bend left, looseness=1.25] (35) to (32.center);
%		\draw [bend right] (35) to (33.center);
%		\draw (33.center) to (34.center);
%		\draw (34.center) to (37.center);
%		\draw (37) to (38);
%		\draw (32.center) to (39.center);
%		\draw [bend left, looseness=1.25] (53) to (50.center);
%		\draw [bend right] (53) to (51.center);
%		\draw (51.center) to (52.center);
%		\draw (52.center) to (55.center);
%		\draw (55) to (56);
%		\draw (50.center) to (57.center);
%		\draw (60.center) to (64);
%		\draw [bend left, looseness=1.25] (64) to (61.center);
%		\draw [bend right] (64) to (62.center);
%		\draw (62.center) to (63.center);
%		\draw (63.center) to (66);
%		\draw (66) to (67);
%		\draw (61.center) to (68.center);
%		\draw [bend left, looseness=1.25] (74) to (71.center);
%		\draw [bend right] (74) to (72.center);
%		\draw (72.center) to (73.center);
%		\draw (73.center) to (75);
%		\draw (75) to (76);
%		\draw (71.center) to (77.center);
%		\draw (80.center) to (84);
%		\draw [bend left, looseness=1.25] (84) to (81.center);
%		\draw [bend right] (84) to (82.center);
%		\draw [bend left, looseness=1.25] (104) to (101.center);
%		\draw [bend right] (104) to (102.center);
%		\draw (102.center) to (103.center);
%		\draw (103.center) to (105);
%		\draw (105) to (106);
%		\draw (101.center) to (107.center);
%		\draw (81.center) to (110.center);
%	\end{pgfonlayer}
%\end{tikzpicture}
%\end{figure}
%\noindent
%\item This point follows directly from the associativity of $\nabla$.\\
%\item Since $f$ is assumed to be $\mC$-functional, we have that
%\begin{figure}[H]
%\centering
%\begin{tikzpicture}[scale=0.60, transform shape]
%	\begin{pgfonlayer}{nodelayer}
%		\node [style=none] (0) at (-5.75, 0) {=};
%		\node [style=box] (11) at (-8.75, 0) {$f$};
%		\node [style=none] (13) at (-8, 0.25) {B};
%		\node [style=none] (14) at (-10, 0) {};
%		\node [style=none] (15) at (-9.5, 0.25) {A};
%		\node [style=none] (17) at (-7, 0.5) {};
%		\node [style=none] (18) at (-7, -0.5) {};
%		\node [style=none] (19) at (-6.25, -0.5) {};
%		\node [style=none] (20) at (-6.25, 0.5) {};
%		\node [style=nodonero] (21) at (-7.5, 0) {};
%		\node [style=none] (23) at (-6.25, -0.5) {};
%		\node [style=none] (24) at (-5.25, 0) {};
%		\node [style=none] (25) at (-3.75, 0.5) {};
%		\node [style=none] (26) at (-3.75, -0.5) {};
%		\node [style=nodonero] (29) at (-4.25, 0) {};
%		\node [style=none] (30) at (-4.75, 0.25) {A};
%		\node [style=box] (32) at (-2.5, -0.5) {$f$};
%		\node [style=none] (34) at (-1.75, -0.75) {B};
%		\node [style=none] (35) at (-3.75, -0.5) {};
%		\node [style=none] (36) at (-3.25, -0.75) {A};
%		\node [style=none] (37) at (-1.25, -0.5) {};
%		\node [style=box] (39) at (-2.5, 0.5) {$f$};
%		\node [style=none] (40) at (-1.75, 0.75) {B};
%		\node [style=none] (42) at (-3.25, 0.75) {A};
%		\node [style=none] (43) at (-1.25, 0.5) {};
%	\end{pgfonlayer}
%	\begin{pgfonlayer}{edgelayer}
%		\draw (14.center) to (11);
%		\draw [bend left, looseness=1.25] (21) to (17.center);
%		\draw (17.center) to (20.center);
%		\draw [bend right] (21) to (18.center);
%		\draw (18.center) to (19.center);
%		\draw (11) to (21);
%		\draw (24.center) to (29);
%		\draw [bend left, looseness=1.25] (29) to (25.center);
%		\draw [bend right] (29) to (26.center);
%		\draw (35.center) to (32);
%		\draw (32) to (37.center);
%		\draw (39) to (43.center);
%		\draw (25.center) to (39);
%	\end{pgfonlayer}
%\end{tikzpicture}
%\end{figure}
%and from this we can conclude that
%\begin{figure}[H]
%\centering
%\begin{tikzpicture}[scale=0.60, transform shape]
%	\begin{pgfonlayer}{nodelayer}
%		\node [style=none] (32) at (6.25, 2) {};
%		\node [style=none] (33) at (6.25, 1) {};
%		\node [style=none] (34) at (7, 1) {};
%		\node [style=nodonero] (35) at (5.75, 1.5) {};
%		\node [style=none] (36) at (4, 1.75) {A};
%		\node [style=box] (37) at (7, 1) {$g$};
%		\node [style=nodonero] (38) at (8.25, 1) {};
%		\node [style=none] (39) at (8.25, 2) {};
%		\node [style=none] (40) at (5.25, 1.75) {B};
%		\node [style=none] (41) at (7.75, 0.75) {C};
%		\node [style=none] (44) at (9, 1.5) {$=$};
%		\node [style=none] (49) at (9.75, 1.5) {};
%		\node [style=none] (50) at (11.25, 2) {};
%		\node [style=none] (51) at (11.25, 1) {};
%		\node [style=none] (52) at (12, 1) {};
%		\node [style=nodonero] (53) at (10.75, 1.5) {};
%		\node [style=none] (54) at (10.25, 1.75) {A};
%		\node [style=box] (55) at (12, 1) {$f$};
%		\node [style=nodonero] (56) at (14.75, 1) {};
%		\node [style=none] (57) at (14.75, 2) {};
%		\node [style=none] (58) at (13.25, 2.25) {B};
%		\node [style=none] (59) at (12.75, 0.75) {B};
%		\node [style=box] (60) at (12, 2) {$f$};
%		\node [style=box] (61) at (13.5, 1) {$g$};
%		\node [style=none] (62) at (14.25, 0.75) {C};
%		\node [style=box] (63) at (4.5, 1.5) {$f$};
%		\node [style=none] (64) at (3.25, 1.5) {};
%		\node [style=none] (65) at (7.75, 2.25) {B};
%	\end{pgfonlayer}
%	\begin{pgfonlayer}{edgelayer}
%		\draw [bend left, looseness=1.25] (35) to (32.center);
%		\draw [bend right] (35) to (33.center);
%		\draw (33.center) to (34.center);
%		\draw (34.center) to (37.center);
%		\draw (37) to (38);
%		\draw (49.center) to (53);
%		\draw [bend left, looseness=1.25] (53) to (50.center);
%		\draw [bend right] (53) to (51.center);
%		\draw (51.center) to (52.center);
%		\draw (52.center) to (55);
%		\draw (55) to (56);
%		\draw (50.center) to (60);
%		\draw (60) to (57.center);
%		\draw (64.center) to (63);
%		\draw (63) to (35);
%		\draw (32.center) to (39.center);
%	\end{pgfonlayer}
%\end{tikzpicture}
%\end{figure}
%\end{enumerate}
%
%\subsection{Proof of Theorem \ref{thm C-functionals form p-cat}}\label{proof: thm C-functionals form p-cat}
%Since every arrow of $\pfn{\mC}$ is $\mC$-functional, $\nabla$ induces a natural transformation in this sub-category. Finally, it is immediate to observe that the axioms of p-categories follow from those of gs-monoidal ones.
%}
%
%\subsection{Proof of Proposition \ref{thm C gs-cat implies C-total maps have weak cart prod}}\label{proof: thm C gs-cat implies C-total maps have weak cart prod}
%
%\begin{enumerate}
%\item Let us consider two objects $A$ and $B$ of $\mC$ and show that $A\ox B$ is a weak product in $\total{\mC}$, with projections $\freccia{A\ox B}{\id_A\ox \,!_B}{A}$ and $\freccia{A\ox B}{!_A\ox \id_B}{B}$. First notice that by Proposition \ref{prop: ! and nabla are total and functional} these projections are $\mC$-total arrows, since they are monoidal products of $\mC$-total arrows. 
%
%Now let us consider two $\mC$-total arrows $\freccia{C}{f}{A}$ and $\freccia{C}{g}{B}$. We claim that the following diagram commutes
%\[\xymatrix@+1.5pc{
%A & A\ox B \ar[l]_{\id_A\ox \,!_B} \ar[r]^{!_A\ox {}\id_B} & B\\
%& C\ar|{(f\ox g)\nabla_C}[u]\ar[ru]_g\ar[lu]^f
%}\]
%In particular, consider the left triangle. In this case
%\[
%	(\id_A\ox \,!_B) (f\ox g) \nabla_C = (f\ox \,!_Bg) \nabla_C,
%\]
%and since $g$ is total, that is $!_B g = {}!_C$, we get
%\[
%	(\id_A\ox \,!_B)(f\ox g) \nabla_C=(f\ox \,!_C)\nabla_C=f.
%\]
%
%Similarly one can check that $g=(!_A\ox \id_B)(f\ox g)\nabla_C$. 
%The claim now follows since $(f \otimes g) \nabla_C$ is itself $\mC$-total by Proposition \ref{prop: ! and nabla are total and functional}.
%
%\item For any two objects $A$ and $B$ of $\mC$, we show that $A\ox B$ is the categorical product of $A$ and $B$ in $\fn{\mC}$ with projections $\freccia{A\ox B}{\id_A\ox\,!_B}{A}$ and $\freccia{A\ox B}{!_A\ox {}\id_B}{B}$. First, by the previous point, we know that this defines a weak product in $\total{\mC}$, and thus in particular in $\fn{\mC}$, since if $f$ and $g$ are $\mC$-functional and $\mC$-total, then again so is $(f \otimes g)\nabla_C$ by Proposition~\ref{prop: ! and nabla are total and functional}.
%
%We still have to prove the uniqueness part of the universal property. So suppose that there exists another $\mC$-total, $\mC$-functional arrow $\freccia{C}{h}{A\ox B}$ such that $(\id_A\ox \,!_B)h=f$ and $(!_A\ox {}\id_B)h=g$. Then we have
%\begin{align*}
%	(f\ox g)\nabla_C & = \left[(\id_A\ox \,!_B)h\ox (!_A\ox \id_B)h\right]\nabla_C	\\[2pt]
%		& = (\id_A\ox \,!_B\,\ox \,!_A\ox {}\id_B)(h\ox h)\nabla_C.
%\end{align*}
%Since $h$ is $\mC$-functional, we can further evaluate to
% \[(f\ox g)\nabla_C=(\id_A\ox \,!_B\, \ox \,!_A\ox {}\id_B)\nabla_{A\ox B}h=h.\]
%\end{enumerate}

%\section{Omitted proofs, Section~\ref{sec:oplax_cart}}
%
%\subsection{Proof of Proposition \ref{prop: fdom(f)=gs f}}\label{proof: prop: fdom(f)=gs f}
%\begin{enumerate}
%	\item By definition of oplax cartesian category, we have an inequality $\duefreccianoname{!_Bf}{\,!_A}$, and therefore
%		\[
%			\duefreccianoname{\dom (f) = (\id_A \ox \,!_Bf)\nabla_A }{(\id_A \ox \,!_A)\nabla_A=\id_A}.
%		\]
%	\item The inequality $\duefreccianoname{f\dom (f) }{f}$ follows by the previous point.
%We can prove the reverse $\duefreccianoname{f}{f\dom (f)}$ by considering the following diagram
%
%
%% https://q.uiver.app/?%q=WzAsOCxbMCwxLCJBIl0sWzEsMSwiQVxcb3RpbWVzIEEiXSxbMiwxLCJBXFxvdGltZXMgQiJdLFsyLDAsIkJcXG90aW1lcyBCIl0sWzAsMCwiQiJdLFsxLDAsIkJcXG90aW1lcyBCIl0sWzMsMSwiQS4iXSxbMywwLCJCIl0sWzAsMSwiXFxuYWJsYV9BIiwyXSxbMSwyLCJpZF9BXFxvdGltZXMgZiIsMl0sWzIsMywiZlxcb3RpbWVzIGlkX0IiLDJdLFswLDQsImYiXSxbNCw1LCJcXG5hYmxhX0IiXSxbNSwzLCJpZF97Qlxcb3RpbWVzIEJ9Il0sWzIsNiwiaWRfQVxcb3RpbWVzICFfQiIsMl0sWzYsNywiZiIsMl0sWzMsNywiaWRfQlxcb3RpbWVzICFfQiJdLFsxLDUsImZcXG90aW1lcyBmIiwyXSxbMTEsMTcsIlxcbGVxIiwxLHsic2hvcnRlbiI6eyJzb3VyY2UiOjIwLCJ0YXJnZXQiOjIwfSwic3R5bGUiOnsiYm9keSI6eyJuYW1lIjoibm9uZSJ9LCJoZWFkIjp7Im5hbWUiOiJub25lIn19fV1d
%\[\begin{tikzcd}
%	B & {B\otimes B} & {B\otimes B} & B \\
%	A & {A\otimes A} & {A\otimes B} & {A}
%	\arrow["{\nabla_A}"', from=2-1, to=2-2]
%	\arrow["{\id_A\otimes f}"', from=2-2, to=2-3]
%	\arrow["{f\otimes \id_B}"', from=2-3, to=1-3]
%	\arrow[""{name=0, anchor=center, inner sep=0}, "f", from=2-1, to=1-1]
%	\arrow["{\nabla_B}", from=1-1, to=1-2]
%	\arrow["{\id_{B\otimes B}}", from=1-2, to=1-3]
%	\arrow["{\id_A\otimes !_B}"', from=2-3, to=2-4]
%	\arrow["f"', from=2-4, to=1-4]
%	\arrow["{\id_B\otimes !_B}", from=1-3, to=1-4]
%	\arrow[""{name=1, anchor=center, inner sep=0}, "{f\otimes f}"', from=2-2, to=1-2]
%	\arrow["\leq"{description}, draw=none, from=0, to=1]
%\end{tikzcd}\]
%The second and the third squares commute, while in the first one we have the first defining inequality of oplax cartesian category.
%Since $(\id_B\ox \,!_B) \nabla_B=\id_B$ by the definition of gs-monoidal category, the upper composite is just $f$ and we obtain the required $\duefreccianoname{f}{f\dom (f)}$. 
%\end{enumerate}

%\subsection{Proof of Proposition~\ref{prop:unicity oplax cartesian cat}}\label{proof:thm:unicity oplax cartesian cat}
%Let us consider the two operators $!$ and $!'$.  By definition of oplax cartesian category and, in particular, employing the second axiom combined with the gs-monoidal axiom $!_I=\id_I$, we have that since $\mC$ is oplax cartesian with respect to $\nabla$ and $!$ we can conclude $!'_A = {} !_I !'_A \leq {}!_A$ for every object $A$.
%We then obtain $!'_A \approx  {}!_A$ by symmetry.
%With a similar argument we can prove that $\nabla_A\approx \nabla'_A$.

%\section{Omitted proofs, Section~\ref{sec:Kleisli and span}}
%
%\subsection{Proof of Proposition~\ref{prop: Kleisli su cartesiane sono gs}}\label{proof: thm: Kelisli su cartesiane sono gs}
%
%It is know that, under the current assumptions, the Kleisli  category $\mA_T$ is a symmetric monoidal category. Now let us consider an object
% $A$ of the  Kleisli category $\mA_T$.
%As mentioned in the statement, we define the arrow $\freccia{A}{\nabla_A^{\sharp}}{A\otimes A}$ of $\mA_T$ as represented by the arrow
%\[\xymatrix@+1.5pc{
%A\ar[r]^{\nabla_A} & A\otimes A \ar[r]^{\eta_{A\otimes A}} & T(A\otimes A)
%}\]
%of $\mA$. Similarly, we define the arrow $\freccia{A}{!_A^{\sharp}}{I}$ of $\mA_T$ as represented by the arrow
%\[\xymatrix@+1.5pc{
%A\ar[r]^{!_A}& I\ar[r]^{\eta_I} & TI.
%}\]
%Although it is possible now to verify directly that the axioms required of a gs-monoidal category are satisfied, there is a more concise and more insightful argument that works as follows.\footnote{See~\cite[Corollary~3.2]{Fritz_2020}: It was previously used for Markov categories.}
%By definition, duplicators and dischargers in $\mA_T$ are defined as the images of those in $\mA$ under the inclusion functor $\mA \to \mA_T$, which is strict symmetric monoidal.
%It now suffices to note that a strict symmetric monoidal functor maps commutative comonoids to commutative comonoids, and the monoidal multiplicativity conditions~\eqref{monoidal_mult} transfer from $\mA$ to $\mA_T$ for the same reason.
%Therefore $\mA_T$ is a gs-monoidal category.

%\subsection{Proof of Proposition~\ref{prop:I_A is a gs functor}}\label{proof:prop:I_A is a gs functor}
%We need to check that $I_{\mA}(\nabla_A)=\nabla_A$ and $I_{\mA}(!_A)= {}!_A$. Both hold by definition of the action $I_{\mA}$, which gives $I_{\mA}(\nabla_A)^\sharp = \eta_{A\otimes A}\nabla_A=\nabla_A^\sharp$ and $I_{\mA}(!_A)^\sharp=\eta_I!_A={}!_A^\sharp$.
%\subsection{Proof of Lemma~\ref{lem: Kleisli are cartesian}}\label{proof:lem: Kleisli are cartesian}
%Let us consider an arrow $\freccia{A}{f}{B}$ of $\mA_T$. We first show that $f$ is functional, meaning $\nabla_B\circ f= (f\otimes f) \circ\nabla_A$.
%Reasoning in terms of representing morphisms in $\mA$, observe that 
%\[(\nabla_B\circ f)^{\sharp}=\mu_{B\otimes B}T(\eta_{B\otimes B}\nabla_B)f^{\sharp}=T(\nabla_B)f^{\sharp}.\]
%By definition of the monoidal structure on $\mA_T$, it is direct to check that
%\[((f\otimes f) \circ\nabla_A)^{\sharp}=c_{B,B}( f^{\sharp}\otimes f^{\sharp})\nabla_A.\]
%Employing first the assumption that $T$ is a gs-monoidal monad and that $\mA$ is cartesian, we have that
%\[ T(\nabla_B)f^{\sharp}= c_{B,B}\nabla_{T(B)}f^{\sharp}=  c_{B,B}( f^{\sharp}\otimes f^{\sharp})\nabla_A.\]
%Therefore we can conclude that $\nabla_B\circ f=(f\otimes f) \circ\nabla_A$.
%Similarly, one can check that $!_Bf={}!_A$ in $\mA_T$.

%\subsection{Proof of Proposition~\ref{prop: Kleisli is oplax cartesian}}\label{proof:prop: Kleisli is oplax cartesian}
%
%\begin{enumerate}
%	\item It is direct to check that $\mA_T$ is poset-enriched with the poset given by $f\leq_{\mA_T} g \iff f^{\sharp}\leq_{\mA} g^{\sharp}$ (one has to use the fact that $\mA$ is poset-enriched and that $T$ is a poset-enriched functor).
%	
%	Checking the oplax cartesianity of $\mA_T$ is analogous to Section~\ref{proof:lem: Kleisli are cartesian}.
%
%	Let us consider an arrow $\freccia{A}{f}{B}$ of $\mA_T$. We first show that $\nabla_B\circ f\leq_{\mA_T} (f\otimes f) \circ\nabla_A$. By definition of $\leq_{\mA_T}$, this holds if and only if $(\nabla_B\circ f)^{\sharp}\leq_{\mA} ((f\otimes f) \circ\nabla_A)^{\sharp}$. Observe that 
%\[(\nabla_B\circ f)^{\sharp}=\mu_{B\otimes B}T(\eta_{B\otimes B}\nabla_B)f^{\sharp}=T(\nabla_B)f^{\sharp}.\]
%By definition of the monoidal structure on $\mA_T$, it is direct to check that
%\[((f\otimes f) \circ\nabla_A)^{\sharp}=c_{B,B}( f^{\sharp}\otimes f^{\sharp})\nabla_A.\]
%Employing first the assumption that $T$ is a colax cartesian functor and then that $\mA$ is oplax cartesian, we have that
%\[ T(\nabla_B)f^{\sharp}\leq c_{B,B}\nabla_{T(B)}f^{\sharp}\leq c_{B,B}( f^{\sharp}\otimes f^{\sharp})\nabla_A.\]
%Therefore we can conclude that $\nabla_B\circ f\leq_{\mA_T} (f\otimes f) \circ\nabla_A$.
%Similarly, one can check that $!_Bf\leq {}!_A$ in $\mA_T$.
%	
%	\item $F_T$ is trivially oplax cartesian since $F_T(\nabla_A) = \nabla_{F_T(A)}$ and $F_T(!_A) = {}!_{F(A)}$ by definition of the gs-monoidal structure on $\mA_T$.
%	\item $G_T$ is oplax cartesian by $G_T(\nabla_A) = \mu_{A \otimes A} \circ T(\nabla_A^\sharp) = T(\nabla_A)$ and the assumption that $T$ is a colax cartesian functor.
%\end{enumerate}

%\section{Omitted proofs, Section~\ref{sec:span}}

%\subsection{Proof of Proposition~\ref{prop: Pspan(A) is oplax cartesian}}\label{proof: prop: Pspan(A) is oplax cartesian}
%It is well-known~\cite[Section~3.1]{Bruni2003}\footnote{Although that reference only considers spans in $\Set$, the proofs go through for any category with finite limits.} and also easy to check in the same way as in Proposition~\ref{prop: Kleisli su cartesiane sono gs} that $\mathbf{PSpan}(\mA)$ is gs-monoidal with respect to the given duplicators and dischargers.
%	So, we just check that the axioms for oplax cartesianity hold in addition.
%	Let us consider an arrow from $X$ to $Y$ in $\mathbf{PSpan}(\mA)$, i.e. $(X \xleftarrow{f} A \xrightarrow{g} Y)$. We have to show the inequality
%	\begin{align*}
%		\nabla_Y^s & \circ(X \xleftarrow{f} A \xrightarrow{g} Y) \\
%			& \le ((X \xleftarrow{f} A \xrightarrow{g} Y)\otimes (X \xleftarrow{f} A \xrightarrow{g} Y) )\circ\nabla_X^s.
%	\end{align*}
%First, note that by definition of composition in $\mathbf{PSpan}(\mA)$, we have that
%$$\nabla_Y^s\circ(X \xleftarrow{f} A \xrightarrow{g} Y)=(X \xleftarrow{f} A \xrightarrow{\nabla_Y g} Y\times Y),
%$$
%and since $\mA$ is cartesian, and hence $\nabla_Yg=(g\times g)\nabla_A$, this evaluates further to
%$$\nabla_Y^s\circ(X \xleftarrow{f} A \xrightarrow{g} Y)=(X \xleftarrow{f} A \xrightarrow{(g\times g)\nabla_A} Y\times Y).
%$$
%Now, employing the universal property of pullbacks and the naturality of $\nabla$ in $\mA$, it is direct to check that 
%\begin{align*}
%	(X \xleftarrow{f} & A \xrightarrow{(g\times g)\nabla_A} Y\times Y)	\\
%		& \leq ((X \xleftarrow{f} A \xrightarrow{g} Y)\otimes(X \xleftarrow{f} A \xrightarrow{g} Y))\circ\nabla_X^s,
%\end{align*}
%as was to be shown.
%Similarly we have the inequality
%$$!_Y^s\circ(X \xleftarrow{f} A \xrightarrow{g} Y)\leq {}!_X^s$$
%via the 2-cell obtained via $f$, since the left-hand side is equal to $(X \xleftarrow{f} A \xrightarrow{!_A} 1)$.
%\dav {new I am checking again point 1) of Prop 4.16}
%\subsection{Proof of Proposition~\ref{prop_functional_and_total_spans}}\label{proof:prop_functional_and_total_spans}
%\begin{enumerate}
%	\item By the first axiom of oplax cartesian categories, it is enough to show that the inequality
%		\begin{align*}
%			\nabla_Y^s & \circ(X \xleftarrow{f} A \xrightarrow{g}Y )	\\
%				& \ge (X\times X \xleftarrow{f\times f} A \times A\xrightarrow{g\times g}Y\times Y )\circ \nabla_X^s
%		\end{align*}
%		holds if and only if $fh_1=fh_2$ implies $gh_1=gh_2$. As in the previous proof, we have
%	$$\nabla_Y^s\circ(X \xleftarrow{f} A \xrightarrow{g}Y )=(X \xleftarrow{f} A \xrightarrow{(g \times g)\nabla_A}Y\times Y ), $$
%	while $(X\times X \xleftarrow{f\times f} A \times A\xrightarrow{g\times g}Y\times Y )\circ \nabla_X^s$ is given by the composite span in
%	
%	
%\[\begin{tikzcd}[column sep=small]
%	&& \bullet \\
%	& X && {A\times A} \\
%	X && {X\times X} && {Y\times Y}
%	\arrow["\id"', from=2-2, to=3-1]
%	\arrow["{\nabla_X}"', from=2-2, to=3-3]
%	\arrow[from=1-3, to=2-2]
%	\arrow[from=1-3, to=2-4]
%	\arrow["{f\times f}", from=2-4, to=3-3]
%	\arrow["{g\times g}", from=2-4, to=3-5]
%	\arrow["\lrcorner"{anchor=center, pos=0.125, rotate=-45}, draw=none, from=1-3, to=3-3]
%\end{tikzcd}\]
%Therefore, employing the universal property of pullbacks and the definition of the poset $\leq$ in $\mathbf{PSpan}(\mA)$, it is direct to check that $fh_1=fh_2$ implies $gh_1=gh_2$ if and only
%\begin{align*}
%			\nabla_Y^s & \circ(X \xleftarrow{f} A \xrightarrow{g}Y )	\\
%				& \ge (X\times X \xleftarrow{f\times f} A \times A\xrightarrow{g\times g}Y\times Y )\circ \nabla_X^s
%		\end{align*}
%i.e. if and only if $(X \xleftarrow{f} Z \xrightarrow{g}Y )$ is weakly $\mathbf{PSpan}(\mA)$-functional. 
%
%%if and only if the commutative square
%%% https://q.uiver.app/?q=WzAsNCxbMCwxLCIgWCJdLFsxLDEsIlhcXHRpbWVzIFgiXSxbMCwwLCJaIl0sWzEsMCwiWlxcdGltZXMgWiJdLFswLDEsIlxcbmFibGFfWCIsMl0sWzIsMCwiZiIsMl0sWzMsMSwiZlxcdGltZXMgZiJdLFsyLDMsIlxcbmFibGFfWiJdXQ==
%%\begin{equation}
%%	\label{eq_pullback mono}
%%	\begin{tikzcd}
%%	A & {A\times A} \\
%%	{ X} & {X\times X}
%%	\arrow["{\nabla_X}"', from=2-1, to=2-2]
%%	\arrow["f"', from=1-1, to=2-1]
%%	\arrow["{f\times f}", from=1-2, to=2-2]
%%	\arrow["{\nabla_A}", from=1-1, to=1-2]
%%\end{tikzcd}
%%\end{equation}
%%is a pullback in $\mA$. It is direct to check that \eqref{eq_pullback mono} is a pullback if and only if $f$ is a monomorphism, hence we can conclude that $(X \xleftarrow{f} A \xrightarrow{g}Y )$ is weakly $\mathbf{PSpan}(\mA)$-functional if and only if $f$ is a monomorphism.
%\item Notice that $!_Y^s\circ(X \xleftarrow{f} A \xrightarrow{g}Y ) = (X \xleftarrow{f} A \xrightarrow{!_A}1 )$. Hence we have the relevant inequality
%	\[
%		!_Y^s\circ(X \xleftarrow{f} A \xrightarrow{g}Y ) \ge {} !_X^s
%	\]
%	if and only if there exists an arrow $\freccia{X}{h}{A}$ such that $fh=\id_X$, i.e.~if and only if $f$ is a split epimorphism. 
%\end{enumerate}
%\iffalse
%\subsection{Proof of Proposition~\ref{prop:every affine commutative monad is weakly affine} }\label{proof:prop:every affine commutative monad is weakly affine}
%Let $m_{X,Y} : T(X \times Y) \longrightarrow TX \times TY$
%	be the arrow defined as the pairing of $T(\pi_1)$ and $T(\pi_2)$.
%	Then it is known that $T$ is affine if and only if $m_{X,Y} c_{X,Y} = \id_{TX \times TY}$~\cite[Lemma~4.2(i)]{Jacobs1994}.\footnote{For probability monads, this equation can be interpreted as stating that the marginals of a product distribution are the original factors~\cite{Fritz2018}.}
%	In particular, $c_{X,Y}$ is a split mono and therefore mono.
%
%	To show that~\eqref{c_assoc_pullback} is a pullback, we prove the universal property starting with a diagram
%	\begin{equation}
%		\label{candidate_pullback}
%		\begin{tikzcd}[column sep=3pc]
%			A \ar[bend left]{drr}{(f_1, f_2)} \ar[bend right,swap]{ddr}{(g_1, g_2)} \ar[dashed]{dr}{\exists!}							\\
%				& TX \times TY \times TZ \ar{r}{\id \times c_{Y,Z}} \ar[swap]{d}{c_{X,Y} \times \id}	& TX \times T(Y \times Z) \ar{d}{c_{X,Y \times Z}}	\\
%				& T(X \times Y) \times TZ \ar{r}{c_{X\times Y,Z}}						& T(X \times Y \times Z)
%		\end{tikzcd}
%	\end{equation}
%	where the dashed arrow will be constructed; its uniqueness is clear since $\id \times c_{Y,Z}$ and $c_{X,Y} \times \id$ are mono, so it remains to prove existence.
%	Taking the unlabelled arrows to be (induced by) product projections, we have the commutative diagram
%	\iffalse
%	\[
%		\begin{tikzcd}[column sep=small]
%			A \ar{r}{(g_1, g_2)} \ar[swap]{d}{(f_1, f_2)}					& T(X \times Y) \times TZ \ar{r}{c_{X \times Y, Z}} 	& T(X \times Y \times Z) \ar{d}	\\
%			TX \times T(Y \times Z) \ar{rr}	\ar["c_{X,Y \times Z}" description]{urr} 	&							& T(Y \times Z)
%		\end{tikzcd}
%	\]
%	\fi
%% https://q.uiver.app/?q=WzAsNSxbMCwwLCJBIl0sWzEsMCwiVChYXFx0aW1lcyBZKVxcdGltZXMgVFoiXSxbMCwxLCJUWFxcdGltZXMgVChZXFx0aW1lcyBaKSJdLFszLDAsIlQoWFxcdGltZXMgWVxcdGltZXMgWikiXSxbMywxLCJUKFlcXHRpbWVzIFopIl0sWzAsMSwiKGdfMSxnXzIpIl0sWzAsMiwiKGZfMSxmXzIpIiwyXSxbMyw0XSxbMiw0XSxbMiwzLCJjX3tYLFlcXHRpbWVzIFp9IiwxXSxbMSwzXV0=
%\[\begin{tikzcd}[column sep=tiny]
%	A & {T(X\times Y)\times TZ} && {T(X\times Y\times Z)} \\
%	{TX\times T(Y\times Z)} &&& {T(Y\times Z)}
%	\arrow["{(g_1,g_2)}", from=1-1, to=1-2]
%	\arrow["{(f_1,f_2)}"', from=1-1, to=2-1]
%	\arrow[from=1-4, to=2-4]
%	\arrow[from=2-1, to=2-4]
%	\arrow["{c_{X,Y\times Z}}"{description}, from=2-1, to=1-4]
%	\arrow[from=1-2, to=1-4]
%\end{tikzcd}\]
%	where the upper left triangle commutes by assumption, and the lower right triangle commutes by naturality of $c$ with respect to the unique arrow $X \to 1$ together with $T1 \cong 1$ and the fact that $c_{1,Y \times Z}$ is a coherence isomorphism.
%	By the naturality of $c$, $f_2$ can be written as the composite
%	\[
%		\begin{tikzcd}[column sep=scriptsize]
%			A \ar{r}{(g_1, g_2)}	& T(X \times Y) \times TZ \ar{r}	& TY \times TZ \ar{r}{c_{Y,Z}}	& T(Y \times Z).			
%		\end{tikzcd}
%	\]
%	By analogous reasoning, we identify $g_1$ with the composite
%	\[
%		\begin{tikzcd}[column sep=scriptsize]
%			A \ar{r}{(f_1, f_2)}	& TX \times T(Y \times Z) \ar{r}	& TX \times TY \ar{r}{c_{X,Y}}	& T(X \times Y).			
%		\end{tikzcd}
%	\]
%	Getting back to~\eqref{candidate_pullback}, we take the dashed arrow to be the arrow whose component on $TX$ is given by $f_1$, on $TZ$ by $g_2$, and on $TY$ by the diagonal in the diagram
%	\[
%		\begin{tikzcd}
%			A \ar{r}{f_2} \ar[swap]{d}{g_1}	& T(Y \times Z)	\ar{d}	\\
%			T(X \times Y) \ar{r}		& TY
%		\end{tikzcd}
%	\]
%	which commutes for similar reasons as above.
%	The fact that this arrow recovers the $f_2$ component after composition with $\id \times c_{Y,Z}$ and the $g_1$ component after composition with $c_{X,Y} \times \id$ follows by the expressions for $f_2$ and $g_1$ derived above.
%	The fact that it recovers $f_1$ and $g_2$ is by construction.
%\subsection{Proof of Lemma~\ref{lem:weakly affine monad useful pullback}}\label{proof:lem:weakly affine monad useful pullback}
%Commutativity of the diagram can be seen by unfolding the definition of $c_{X,Y}$ in terms of the strength and multiplication of $T$ and applying the compatibility of these with the unit.
%	%\tob{Should we elaborate or is it clear enough?}
%	In the diagram
%	\[
%		\begin{tikzcd}
%			X \times Y \ar{r}{\id} \ar[swap]{d}{\eta_X \times \eta_Y}	& X \times Y \ar{d}{\eta_X \times \eta_Y}	\\
%			TX \times TY \ar{r}{\id} \ar[swap]{d}{\id}			& TX \times TY \ar{d}{c_{X,Y}}			\\
%			TX \times TY \ar{r}{c_{X,Y}}					& T(X \times Y)
%		\end{tikzcd}
%	\]
%	the upper square is a trivial pullback and the lower square is a pullback since $c_{X,Y}$ is mono.
%	Therefore also the composite square is a pullback.
%	This composite square is exactly~\eqref{meekly_affine}.
%\subsection{Proof of Theorem~\ref{thm: S is colax Frobenius opcartesian lax-functor}}\label{proof:thm: S is colax Frobenious opcartesian lax functor}
%We show that $S_T$ is a \emph{lax functor}: let us consider two arrows $\freccia{X}{f}{Y}$ and $\freccia{Y}{g}{Z}$ of $\mA_T$, represented by $\freccia{X}{f^\sharp}{TY}$ and $\freccia{Y}{g^\sharp}{TZ}$ in $\mA$. The span $S_T(g)\circ S_T(f)$ is represented by the composite in
%
%\begin{tikzpicture}[baseline= (a).base]
%	\node[scale=.82] (a) at (0,0){
%		\begin{tikzcd}
%			& & X \times_{f^\sharp,\eta_Y} Y \ar[swap]{dl}{\ell} \ar{dr}{r}		\\
%			& X \ar[swap]{dl}{\eta_X} \ar{dr}{f^\sharp}	& & Y \ar[swap]{dl}{\eta_Y} \ar{dr}{g^\sharp} \\
%			TX	& & TY	& & T Z
%		\end{tikzcd}};
%	\end{tikzpicture}
%
%\noindent
%	where the upper square is a pullback.
%	On the other hand, the span $S_T(g \circ f)$ is  $(TX\xleftarrow{\eta_X}X\xrightarrow{\mu_ZT(g^\sharp)f^\sharp}TZ)$. Now notice that the diagram 
%	\[
%		\begin{tikzcd}[column sep=small,row sep=small]
%			& X\times_{f^\sharp,\eta_Y}Y\ar[dl, bend right, swap,"\eta_X\ell"] \ar[rd, bend left, "g^\sharp r"]\ar[dd, "\ell"] & \\
%			TX  & & TZ  \\
%			& X \ar[ul, bend left,"\eta_X"] \ar[ur,  bend right, swap,"\mu_ZT(g^\sharp) f^\sharp"]
%		\end{tikzcd}
%	\]
%	commutes, since $\mu_ZT(g^\sharp)f^\sharp\ell = \mu_ZT(g^\sharp)\eta_Y r$ and by the naturality of $\eta$, we can conclude the desired $\mu_Z T(g^\sharp)f^\sharp\ell = (\mu_Z\eta_{TZ})g^\sharp r = g^\sharp r$.
%	Therefore we obtain that $S_T(g)\circ S_T(f)\leq S_T(g\circ f)$.
%
%	Similarly, it is easy to check that $\id_{S_T(X)}\leq S_T(\id_X)$ in $\mathbf{PSpan}(\mA)$. Thus  $\freccia{\mA_T}{S_T}{\mathbf{PSpan}(\mA)}$ is a lax functor.\\
%
%Now we show that $S_T$ is \emph{lax symmetric monoidal}, with the structure arrows given by the spans
%\begin{align*}
%	\psi_0 & := ( 1 \xleftarrow{\id }1 \xrightarrow{\eta_{1} }T(1)),	\\
%	\psi_{X,Y} & :=( TX\times TY \xleftarrow{\id}TX\times TY \xrightarrow{c_{X,Y}}T(X\times Y)).
%\end{align*}
%
%
%
%	We first prove that $\psi$ is a natural transformation. Let us consider two arrows $\freccia{X}{f}{W}$ and $\freccia{Y}{g}{Z}$ of $\mA_T$. We have to check that
%	\[
%		\psi_{W,Z}\circ (S_T(f)\otimes S_T(g))=S_T(f\otimes g)\circ \psi_{X,Y}.
%	\]
%	First notice that $\psi_{W,Z}\circ (S_T(f)\otimes S_T(g))$ is 
%	\[(TX\times TY\xleftarrow{\eta_X\times \eta_Y}X\times Y \xrightarrow{c_{Z,W}(f^\sharp \times g^\sharp)} T(Z\times W)). \]
%	Now, since the monad $T$ is weakly affine by hypothesis, we have that $S_T(f\otimes g)\circ \psi_{X,Y}$ is given precisely by the composite span in
%	
%	\begin{tikzpicture}[baseline= (a).base]
%		\node[scale=.82] (a) at (0,0){
%	\begin{tikzcd}[column sep =small]
%	&& {X\times Y} \\
%	& {TX\times TY} && {X\times Y} \\
%	{TX\times TY} && {T(X\times Y)} && {T(Z\times W)}
%	\arrow["\id"', from=2-2, to=3-1]
%	\arrow["{\eta_X\times \eta_Y}"', from=1-3, to=2-2]
%	\arrow["{c_{X,Y}}"', from=2-2, to=3-3]
%	\arrow["\id", from=1-3, to=2-4]
%	\arrow["{\eta_{X\times Y}}", from=2-4, to=3-3]
%	\arrow["{c_{Z,W}(f^{\sharp}\times g^{\sharp})}", from=2-4, to=3-5]
%	\arrow["\lrcorner"{anchor=center, pos=0.125, rotate=-45}, draw=none, from=1-3, to=3-3]
%\end{tikzcd}};
%\end{tikzpicture}
%%\tob{This seems to be using that $\psi_{X,Y}$ is equivalently given by $TX \times TY \leftarrow TX \times TY \stackrel{c_{X,Y}}{\rightarrow} T(X \times Y)$, so perhaps we need to prove that first or redefine $\psi$ accordingly} 
%and then we can conclude that  $\psi_{W,Z}\circ (S_T(f)\otimes S_T(g))=S_T(f\otimes g)\circ \psi_{X,Y}$. So $\psi$ is a natural transformation.
%	
%	The associativity and unitality diagrams of a lax monoidal structure (Definition \ref{def:lax monoidal functor}) follow from the fact that $\psi$ is given by the composition of strength arrows and multiplications.
%	Therefore $\psi$ equips $S_T$ with the structure of a lax symmetric monoidal functor.\\
%
%Now we show that $S_T$ is \emph{oplax symmetric monoidal}, with the structure arrows given by the spans
%\begin{align*}
%	\phi_0 & := ( T(1) \xleftarrow{\eta_1 }1 \xrightarrow{\id }1)	\\
%	\phi_{X,Y} & :=( T(X\times Y)  \xleftarrow{c_{X,Y}} TX\times TY \xrightarrow{\id}T(X)\times T( Y)).
%\end{align*}
%	We first prove that $\phi$ is a natural transformation. Let us consider two arrows $\freccia{X}{f}{W}$ and $\freccia{Y}{g}{Z}$ of $\mA_T$. We have to check that
%	\[
%		\phi_{W,Z}\circ S_T(f \otimes g)=(S_T(f)\otimes S_T(g))\circ \phi_{X,Y}.
%	\]
%	First notice that, since $c_{X,Y}(\eta_X\times \eta_Y) = \eta_{X\times Y}$, we have that $(S_T(f)\otimes S_T(g))\circ \phi_{X,Y}$ is represented by the span
%	\[T(X\times Y)\xleftarrow{\eta_{X\times Y}}X\times Y\xrightarrow{f^\sharp\times g^\sharp}TW\times TZ.\]
%	Now since the monad $T$ is weakly affine, the span  $\phi_{W,Z}\circ S_T(f \otimes g)$ is given precisely by the composite span in
%	\hspace{-.1cm}
%	\begin{tikzpicture}[baseline= (a).base]
%		\node[scale=.82] (a) at (0,0){
%	\begin{tikzcd}[column sep=small]
%		&& {X\times Y} \\
%		& {X\times Y} && {TW\times TZ} \\
%		{T(X\times Y)} && {T(W\times Z)} && {TW\times TZ}
%		\arrow["{\eta_{X\times Y}}"', from=2-2, to=3-1]
%		\arrow["{c_{W,Z}(f^\sharp \times g^\sharp)}"', from=2-2, to=3-3]
%		\arrow["{c_{W,Z}}", from=2-4, to=3-3]
%		\arrow["\id", from=2-4, to=3-5]
%		\arrow["\id"', from=1-3, to=2-2]
%		\arrow["{f^\sharp \times g^\sharp}", from=1-3, to=2-4]
%		\arrow["\lrcorner"{anchor=center, pos=0.125, rotate=-45}, draw=none, from=1-3, to=3-3]
%	\end{tikzcd}};
%\end{tikzpicture}
%	and then $\phi_{W,Z}\circ S_T(f \otimes g)=(S_T(f)\otimes S_T(g))\circ \phi_{X,Y}$.
%	As for the lax structure, the associativity and unitality diagrams of an oplax monoidal structure follow from the fact that $\phi$ is given by the composition of strength arrows and multiplications.
%	Therefore $\phi$ equips $S_T$ with the structure of an oplax symmetric monoidal functor.\\
%
%Now we show that $S_T$ is a \emph{Frobenius monoidal lax functor}.
%By the definition of Frobenius monoidality (Definition \ref{def:frobenius monoidalfunctor}), we have to prove that the two diagrams
%
%\begin{tikzpicture}[baseline= (a).base]
%	\node[scale=.88] (a) at (0,0){
%\begin{tikzcd}[column sep=scriptsize]
%		{S_T(X\times Y)\otimes S_T(Z)} && {S_T(X\times Y\times Z)} \\
%		{S_T(X)\otimes S_T(Y)\otimes S_T(Z)} && {S_T(X)\otimes S_T(Y\times Z)} \\
%		{S_T(X)\otimes S_T(Y\times Z)} && {S_T(X\times Y\times Z)} \\
%		{S_T(X)\otimes S_T(Y)\otimes S_T(Z)} && {S_T(X\times Y)\otimes S_T(Z)}
%		\arrow["{\phi_{X,Y}\otimes {}\id}"', from=1-1, to=2-1]
%		\arrow["{\psi_{X\otimes Y,Z}}", from=1-1, to=1-3]
%		\arrow["{\phi_{X,Y\otimes Z}}", from=1-3, to=2-3]
%		\arrow["{\id\otimes \psi_{Y,Z}}"', from=2-1, to=2-3]
%		\arrow["{\psi_{X,Y\otimes Z}}", from=3-1, to=3-3]
%		\arrow["{\phi_{X\otimes Y,Z}}", from=3-3, to=4-3]
%		\arrow["{\id\otimes \phi_{Y,Z}}"', from=3-1, to=4-1]
%		\arrow["{\psi_{X\otimes Y,Z}\otimes {}\id}"', from=4-1, to=4-3]
%	\end{tikzcd}};
%\end{tikzpicture}
%
%\noindent
%	commute in $\mathbf{PSpan}(\mA)$. Let us consider the first square. By definition of $\psi$ and $\phi$, the arrow $(\id\otimes \psi_{Y,Z})\circ (\phi_{X,Y}\otimes \id)$ is given by the span
%	\[ T(X\times Y)\times TZ \xleftarrow{c_{X,Y}\times {}\id}TX\times TY\times TZ\xrightarrow{\id\times c_{Y,Z}}TX\times T(Y\times Z).\]
%Employing the assumption that $T$ is weakly affine, and in particular that the diagram \eqref{c_assoc_pullback} is a pullback, we can conclude that the composite $\phi_{X,Y\otimes Z}\circ \psi_{X\otimes Y,Z}$ is given by the same span and hence that the first square commutes.
%By analogous reasoning, we can conclude that also the second square commutes, and hence that $S_T$ is Frobenius monoidal.\\
%
%Now we show that $S_T$ is a \emph{colax cartesian lax functor}: employing the fact that $\mA$ is cartesian, and hence that $\nabla$ is natural, it is straightforward to check that in $\mathbf{PSpan}(\mA)$ we have the inequality
%		\[ S_T(\nabla_{X})\leq\psi_{X,X} \circ \nabla_{S_T(X)}\]
%		Similarly, it is straightforward to check that
%		\[  S_T(!_A) \leq \psi_0\circ{} !_{S_TA}.\]
%	%	
%	Consider now the two spans
%		\begin{align*}
%			\nabla_{S_T(X)} &:= (TX\xleftarrow{\id_X}TX\xrightarrow{\nabla_{TX}}TX\times TX),			\\
%			S_T(\nabla_{X}) &:= (TX\xleftarrow{\eta_X}X\xrightarrow{\eta_{X\times X}\nabla_X}T(X\times X))
%		\end{align*}
%		In $\mathbf{PSpan}(\mA)$, we will prove the inequality
%		\[ S_T(\nabla_{X})\leq\psi_{X,X} \circ \nabla_{S_T(X)}\]
%		as witnessed by the unit $\eta_X$. Indeed, we have that $$\psi_{X,X} \circ \nabla_{S_T(X)}=(TX\xleftarrow{\id_X}TX\xrightarrow{c_{X,X}\nabla_{TX}}TX\times TX)$$
%		%\tob{see above}
%	and, employing the naturality of $\nabla$ we have that $$c_{X,X}\nabla_{TX}\eta_X=c_{X,X}(\eta_X\times \eta_X)\nabla_X$$
%	And since $c_{X,X}(\eta_X\times \eta_X)=\eta_{X\times X}$, we have that the following diagram commute
%	\[\begin{tikzcd}[column sep=small,row sep=small]
%		& X \\
%		TX && {T(X\times X)} \\
%		& TX
%		\arrow["\id", curve={height=-12pt}, from=3-2, to=2-1]
%		\arrow["{\eta_X}"', curve={height=12pt}, from=1-2, to=2-1]
%		\arrow["{\eta_{X\times X}\nabla_X}", curve={height=-12pt}, from=1-2, to=2-3]
%		\arrow["{\eta_X}", from=1-2, to=3-2]
%		\arrow["{c_{X,X}\nabla_{TX}}"', curve={height=12pt}, from=3-2, to=2-3]
%	\end{tikzcd}\]
%	
%	It is immediate to check that
%	\[  S_T(!_X) \leq \psi_0\circ {}!_{S_TX}\]
%	Thus we can conclude that $S_T$ is a colax cartesian functor.\\
%
%	
%	
%	
%	
%	
%	Now we show that $S_T$ is a \emph{colax opcartesian lax functor}: we must check that 
%			\[ \phi_{X,X} \circ S_T(\nabla_{X})\leq \nabla_{S_T(X)},\]
%			and that 
%			\[  \phi_0\circ S_T(!_A) \leq {}!_{S_TA}.\]
%			First notice that $\phi_{X,X} \circ S_T(\nabla_{X})$ is given the span
%			\[\phi_{X,X} \circ S_T(\nabla_{X}):=(TX \xleftarrow{\eta_X} X \xrightarrow{(\eta_X\times \eta_X) \nabla_X} TX\times TX)\]
%	obtained by the composition
%	
%	\begin{tikzpicture}[baseline= (a).base]
%		\node[scale=.82] (a) at (0,0){
%	\begin{tikzcd}[column sep=scriptsize]
%		&& X \\
%		& X && {TX\times TX} \\
%		TX && {T(X\times X)} && {TX\times TX}
%		\arrow["{\eta_X}"', from=2-2, to=3-1]
%		\arrow["{\eta_{X\times X}\nabla_X}"', from=2-2, to=3-3]
%		\arrow["{c_{X,X}}", from=2-4, to=3-3]
%		\arrow["\id", from=2-4, to=3-5]
%		\arrow["{(\eta_X\times \eta_X)\nabla_X}", from=1-3, to=2-4]
%		\arrow["\id"', from=1-3, to=2-2]
%		\arrow["\lrcorner"{anchor=center, pos=0.125, rotate=-45}, draw=none, from=1-3, to=3-3]
%	\end{tikzcd}};
%\end{tikzpicture}
%	
%\noindent
%	while $\nabla_{S_T(X)} := (TX\xleftarrow{\id_X}TX\xrightarrow{\nabla_{TX}}TX\times TX)$. Employing the fact that $\mA$ is cartesian, and hence every arrow is $\mA$-functional, it is direct to check that $(\eta_X\times \eta_X)\nabla_X=\nabla_{TX}\eta_X$, and then that $\phi_{X,X} \circ S_T(\nabla_{X})\leq \nabla_{S_T(X)}$ via $\eta_X$.
%	Similarly, employing the fact that every arrow in a cartesian category is also $\mA$-total, it is direct to check that $ \phi_0\circ S_T(!_A) \leq {}!_{S_TA}$.
%		
%	\fi
%\section{Omitted proofs, Section~\ref{sec:completeness}}

%\subsection{Proof of Lemma~\ref{lax_is_cogs}}\label{proof:lax_is_cogs}

%For any $F : \mC \to \Set$, there is an equivalence between lax symmetric monoidal structures on $F$ and commutative monoid structures with respect to Day convolution, in such a way that the monoidal natural transformations are in natural bijection with the monoid homomorphisms\footnote{See e.g.~\cite[Example~3.2.2]{Day70}, and~\cite[Proposition~22.1]{Mandell01} for a version of the statement for presheaves with values in topological spaces.}
%	Thus it suffices to show that the category of commutative monoids in the symmetric monoidal category of functors under Day convolution is 
%	co-gs-monoidal in a canonical way.
%	But this latter statement is an instance of the fact that the category of commutative monoids in \emph{any} symmetric monoidal category is a co-gs-monoidal category in a canonical way (Example~\ref{comon_is_gs}).
%
%	In any such category, the co-total and co-functional arrows are exactly the monoid homomorphisms.
%	This implies the claim that the total and functional arrows in $\mathbf{LaxSymMon}(\mC,\Set)^{\op}$ are exactly the formal opposites of monoidal natural transformations.

%	  \subsection{Proof of Proposition~\ref{prop:Yoneda}}\label{proof:prop:Yoneda}
%	  On objects, we define $\yo(A) \coloneqq \mC(A,-)$, which is a lax monoidal functor $\mC \to \Set$; for the lax monoidal structure,
%	  we refer forward to the proof of Theorem~\ref{thm:simple_complete_bi_lax}.
%	The action of $\yo$ on an arrow $f : B \to A$ is given by precomposition, and this defines a natural transformation 
%	$\mC(B,-) \to \mC(A,-)$.
%	The full faithfulness of $\yo$ holds by the standard Yoneda embedding.
%
%	It remains to equip $\yo$ with a oplax gs-monoidal structure, recalling that $\mathbf{LaxSymMon}(\mC,\Set)^{\op}$ carries the gs-monoidal structure introduced in Lemma~\ref{lax_is_cogs}.
%	For the oplaxator, note that we have a transformation
%	\[
%		\begin{tikzcd}[row sep=1pt,column sep=1.4pc]
%			\mC(A,X) \otimes \mC(B,Y) \ar{r}	& \mC(A \otimes B, X \otimes Y)	\\
%			(f,g)			\ar{r}		& f \otimes g
%		\end{tikzcd}
%	\]
%	that is natural in all four arguments $A,B,X,Y \in \mC$. By the universal property of Day convolution, this can be regarded as an arrow
%	\[
%		\begin{tikzcd}[column sep=1.4pc]
%			\mC(A,-) \boxtimes \mC(B,-) \ar{r}	& \mC(A \otimes B,-)
%		\end{tikzcd}
%	\]
%	in $\mathbf{LaxSymMon}(\mC,\Set)$.
%	Its naturality in $A$ and $B$ also follows by the universal property.
%	Let us show that considering these transformations in the opposite category defines the comultiplication of the claimed oplax monoidal structure on $\yo$, while by $\yo(I) = \mC(I,-)$ we have strict counitality.
%	The coassociativity equation for the comultiplication amounts exactly to the associativity of the monoidal structure of $\mC$.
%	The left unitality equation holds by the commutativity of the diagram
%	\[
%		\begin{tikzcd}[column sep=small]
%			&	\mC(I,-) \boxtimes \mC(A,-) \ar{dr}{\text{\scriptsize{laxator}}}	\\
%			\mC(A,-) \ar{ur}{\parbox{2.2cm}{\centering \scriptsize{unitor of\\ Day convolution}}} \ar[swap]{rr}{\text{\scriptsize{induced by unitor in }} \mC}	&& \mC(I \otimes A,-)
%		\end{tikzcd}
%	\]
%	and similarly for right unitality.
%
%	The preservation of the duplicators amounts to the diagram
%	\[
%		\begin{tikzcd}[column sep=small]
%			& \mC(A,-) \boxtimes \mC(A,-) \ar{dr}{\text{\scriptsize{laxator}}} \ar[swap]{dl}{\text{\scriptsize{lax structure on }} \mC(A,-)}	\\
%			\mC(A,-) && \mC(A \otimes A, -) \ar[swap]{ll}{\mC(\nabla_A,-)}	
%		\end{tikzcd}
%	\]
%	which holds by definition of the lax structure on $\mC(A,-)$, and similarly for the dischargers.

%	\subsection{Proof of Lemma ~\ref{lem:compositin_lax_mon_and_lax_on_ident_mon}}\label{proof:lem:compositin_lax_mon_and_lax_on_ident_mon}

%	Let us denote by $\psi^F$ and $\psi_0^F$ the lax monoidal structure of $F$ and by $\psi^G$ and $\psi_0^G$ the one of $G$. We claim that $F\circ G$ is a lax monoidal lax-on identities functor with $\psi_{X,Y}^{FG}$ given by the composite
%	% https://q.uiver.app/?q=WzAsMyxbMCwwLCJGRyhYKVxcb3RpbWVzIEZHKFkpIl0sWzIsMCwiRihHWFxcb3RpbWVzIEdZKSJdLFs0LDAsIkZHKFhcXG90aW1lcyBZKSJdLFswLDEsIlxccHNpX3tHWCxHWX1eRiJdLFsxLDIsIkYoXFxwc2leR197WCxZfSkiXV0=
%\[\begin{tikzcd}[column sep=small]
%	{FG(X)\otimes FG(Y)} && {F(G(X)\otimes G(Y))} && {FG(X\otimes Y)}
%	\arrow["{\psi_{GX,GY}^F}", from=1-1, to=1-3]
%	\arrow["{F(\psi^G_{X,Y})}", from=1-3, to=1-5]
%\end{tikzcd}\]
%and $\psi_0^{FG}:=F(\psi_0^G) \psi_0^F$.
%First, it is easy to check that $FG$ is a lax-on-identities functor, in particular since
%\[
%	\id_{FG(A)}\leq F(\id_{G(A)})=FG(\id_A),
%\]
%and that $\psi^{FG}$ is a natural transformation. Now we show that $\psi^{FG}$ satisfies the associativity axiom of a lax monoidal lax-on-identity functor, that is for all objects $A,B,C$ in $\mC$,
%\begin{equation}
%	\label{eq:FG_assoc}
%	\psi_{A,B\otimes C}^{FG} (FG(\id_A)\otimes \psi_{B,C}^{FG})= 
%	\psi_{A\otimes B,C}^{FG}(\psi_{A,B}^{FG}\otimes FG(\id_C)).
%\end{equation}
%Now, by definition of $\psi^{FG}$, and since $G$ is strict on identities by hypothesis, we have
%\begin{align*}
%	\psi_{A,B\otimes C}^{FG} & (FG(\id_A)\otimes \psi_{B,C}^{FG}) \\
%		& = F(\psi_{A,B\otimes C}^G)\psi_{G(A),G(B\otimes C)}^F \\
%		& \quad\, (F(\id_{G(A)})\otimes F(\psi^G_{B,C})) \\
%		& \quad\, (F(\id_{G(A)})\otimes \psi_{G(B),G(C)}^F)
%\end{align*}
%By naturality of $\psi^F$, and since $F$ preserves the composition strictly, we can further evaluate this to
%\begin{align*}\label{eq:second_associativity_ax}
%	F(\psi_{A,B\otimes C}^G & (\id_{G(A)}\otimes \psi^G_{B,C})) \\
%		& \psi_{G(A),G(B)\otimes G(C)}^F(F(\id_{G(A)})\otimes \psi_{G(B),G(C)}^F)
%\end{align*}
%If we apply analogous steps to the right-hand side of~\eqref{eq:FG_assoc}, then we obtain 
%\begin{align*}
%	F(\psi_{A\otimes B, C}^G & (\psi^G_{A,B}\otimes \id_{G(C)})) \\
%		& \psi_{G(A)\otimes G(B), G(C)}^F(\psi_{G(A),G(B)}^F\otimes F(\id_{G(C)})).
%\end{align*}
%This differs from the previous expression by an application of the associativity axioms of $\psi^G$ and $\psi^F$ (with $F(\id)$ instead of $\id$ in the second case).
%Therefore $\psi^{FG}$ satisfies the associativity as well.
%
%Similarly, one can check that also the unitality axioms are satisfied.


\end{appendices}




\end{document}