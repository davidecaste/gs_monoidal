
\theoremstyle{definition}
\newtheorem*{notation*}{Notation}

%%%%%%%%%%%%%%%%frecce%%%%%%%%%%%%%%
\newcommand{\cobang}{\mathrel{\rotatebox[origin=c]{180}{!}}}

%\newcommand{\relazione}[3]{\xymatrix@-0.8pc{#2 \colon #1  \ar[rr]\ar@{-|}[r]& &  #3}}
\newcommand{\relazione}[3]{#2 \colon #1  \to #3}
\newcommand{\freccia}[3]{#2 \colon #1  \to #3}
\newcommand{\frecciainj}[3]{#2 \colon #1  \to #3}
\newcommand{\frecciasopra}[3]{\xymatrix{ #1  \ar[r]^{#2} &  #3}}
\newcommand{\frecciasopralunga}[3]{\xymatrix{ #1  \ar[rr]^{#2} &&  #3}}
\newcommand{\pbmorph}[2]{#1^{\ast}#2} 
\newcommand{\duefreccia}[3]{\xymatrix@C=0.5cm{#2 \colon #1  \ar@{=>}[r] &  #3}}
%\newcommand{\duefreccianoname}[2]{\xymatrix@C=0.5cm{#1  \ar@{=>}[r] &  #2}}
\newcommand{\duefreccianoname}[2]{#1\leq #2}
\newcommand{\equivalence}[3]{#2 \colon #1 \equiv #3}

\newcommand{\comsquare}[8]{ \xymatrix@+1pc{ 
#1 \ar[r]^{#5} \ar[d]_{#6} & #2 \ar[d]^{#7} \\
#3 \ar[r]_{#8} & #4 
}}
\newcommand{\pullback}[8]{ \xymatrix@+1pc{ 
#1 \pullbackcorner \ar[r]^{#5} \ar[d]_{#6} & #2 \ar[d]^{#7} \\
#3 \ar[r]_{#8} & #4 
}}
\newcommand{\quadratocomm}[8]{ \xymatrix@+1pc{ 
#1 \ar[r]^{#5} \ar[d]_{#6} & #2 \ar[d]^{#7} \\
#3 \ar[r]_{#8} & #4 
}}
\newcommand{\comsquarelargo}[8]{ \xymatrix@+1pc{ 
#1 \ar[rr]^{#5} \ar[d]_{#6} && #2 \ar[d]^{#7} \\
#3 \ar[rr]_{#8} && #4 
}}
\newcommand{\parallelmorphisms}[4]{\xymatrix@+1pc{
#1 \ar @<+4pt>[r]^{#2} \ar @<-4pt>[r]_{#3} & #4
}}
\newcommand{\relation}[4]{\xymatrix@+1pc{
\angbr{#2}{#3}\colon #1 \ar @<+4pt>[r] \ar @<-4pt>[r] & #4
}}
\newcommand{\frecceparalleleopposte}[4]{\xymatrix@+1pc{
#1 \ar@<+4pt>[r]^{#2} \ar@<-4pt>@{<-}[r]_{#3} & #4
}}
\newcommand{\equalizer}[6]{\xymatrix@+1pc{
#1 \ar[r]^{#2} & #3 \ar @<+4pt>[r]^{#4} \ar @<-4pt>[r]_{#5} & #6
}}
\newcommand{\coequalizer}[6]{\xymatrix@+1pc{
 #1 \ar @<+4pt>[r]^{#2} \ar @<-4pt>[r]_{#3} & #4 \ar[r]^{#5} & #6
}}

\newcommand{\subobject}[3]{\xymatrix{
#1 \ar@{>->}[r]^{#2} & #3
}}

\newcommand{\pullbackcorner}[1][ul]{\save*!/#1+1.2pc/#1:(1,-1)@^{|-}\restore}



%%%%%%%%%%%%%%%%%%%%%%%% Categorie %%%%%%%%%
\def\mA{\mathcal{A}}
\def\mB{\mathcal{B}}
\def\mC{\mathcal{C}}
\def\mX{\mathcal{X}}
\def\mD{\mathcal{D}}
\def\mE{\mathcal{E}}
\def\mF{\mathcal{F}}
\def\mM{\mathcal{M}}
\def\mN{\mathcal{N}}
\def\mH{\mathcal{H}}
\def\mK{\mathcal{K}}
\def\mO{\mathcal{O}}
\def\mQ{\mathcal{Q}}
\def\mV{\mathcal{V}}

%%%%%%%%%%%%%%%%%%%%%%%%%%%%%%%%
\def\FinSet{\mathbf{FinSet}}
\def\FinRel{\mathbf{FinRel}}
\def\Set{\mathbf{Set}}
\def\Rel{\mathbf{Rel}}
\def\Preord{\mathbf{Preord}}
\newcommand{\pfn}[1]{#1\mbox{-}\mathbf{Fun}}
\newcommand{\fn}[1]{#1\mbox{-}\mathbf{TFun}}
\newcommand{\total}[1]{#1\mbox{-}\mathbf{Total}}
\newcommand{\alg}[1]{\mathbf{Alg}(#1)}
\newcommand{\Tadj}[2]{#1^{#2}}
%%%%%%%%%%%%%%%%%operatori monoidali%%%%%%%%%%%%%%%%%%%
\def\ox{\otimes}
%%%%%%%%%%%%%%%%%%%%%%%%identità, domini, cod ecc%%%%%%%%5
\def\pr{\operatorname{ pr}}         %projection
\def\id{\operatorname{ id}}         %identity
\def\op{\operatorname{ op}}         %opposite
\def\cod{\operatorname{ cod}}       %codomain
\def\dom{\operatorname{ dom}}      %domain
\def\im{\operatorname{ im}}        %image
\def\ob{\operatorname{ ob}}        %objects
\def\I{I}                          %identity monoidale


%%%%%%%%%%%%%%%%%%%%%%%%%%%%%%%%%%%%%%%%%%%%%%%%%%%%


\newcommand{\angbr}[2]{\langle #1,#2 \rangle} 

\def\gsmcat{\mathbf{GSM\mbox{-}Cat}}

\newcommand{\syntcat}[1]{\mathbf{Th}(#1)}
\newcommand{\context}[3]{[#1_1:#2_1,\dots,#1_{#3}:#2_{#3}]}
\newcommand{\termincontext}[3]{#1:#2 \; [#3]}
\newcommand{\MSalgebra}[1]{\mathrm{#1}}

