\documentclass{article}
\usepackage[english]{babel}
\usepackage{natbib}
% Useful packages
\usepackage[all,2cell]{xy}

%\usepackage{natbib}
\usepackage{graphicx}
\usepackage{amsmath}
\usepackage{amsthm}
\usepackage{amsfonts}
\usepackage{amssymb}
\usepackage{hyperref}
 \usepackage{tikzit}				% String diagrams
 \input{markov.tikzstyles}



%%%%%%%%%%%%%%%%%%%%%%%%%%%%%%%%%%%%%

\title{Conditionals}
\author{XXXXX}
\date{}


%%%%%%%%%%%%%%%%%%%%%%%%%%%%%%%%

\theoremstyle{plain} %italico
\newtheorem{theorem}{Theorem}[section]
\newtheorem{cor}[theorem]{Corollary}
\newtheorem{lemma}[theorem]{Lemma}
\newtheorem{proposition}[theorem]{Proposition}
\newtheorem{fact}[theorem]{Fact}

\theoremstyle{definition} %stampatello
\newtheorem{definition}[theorem]{Definition}
\newtheorem{remark}[theorem]{Remark}
\newtheorem{example}[theorem]{Example}
\newtheorem{exercise}[theorem]{Exercise}
\newtheorem{conjecture}[theorem]{Conjecture}

\begin{document}
\maketitle

\begin{abstract}

\end{abstract}    

\section{Introduction}
The key diagrams for the decomposition of an arrow into a marginal and a conditional can be given for share categories, i.e., where only duplicators exist. 
%
The aim of this note is to propose an axiomatisation of marginals and conditional in such a category, pointing out the interaction with the monoidal product.

We fix a share category $ \langle \mathcal{C}, \otimes, \nabla\rangle$, namely a monoidal category with a duplicator.

\begin{definition}
A share category admits a decomposition if for every objects $A$, $B$ and arrow $f: X \to A \otimes B$ there exist arrows $m(f, A, B): X \to A$ and $c(f, A, B): A \otimes X \to B$ such that
\[
f = \nabla_X ; (m(f, A, B) \otimes id_X) ; \nabla_A \otimes id_x) ; (id_A \otimes c(f, A, B) \otimes id_X)
\]
and subject to suitable axioms.
\end{definition}
In terms of string diagrams
\ctikzfig{mc}


We need to properly axiomatise these two operators, in such a way that their laws hold at least for the canonical interpretation in a Markov category 
of the marginal $m(f, A, B): X \to A$ as $f ; (id_A \otimes !_B)$. This implies that

\begin{itemize}
\item $m(\nabla_X, X, X) = id_X$
\item $m(f, A, B_1 \otimes B_2) = m(m(f, A \otimes B_1, B_2), A, B_1)$ 
\end{itemize}

Note that the first item forces the identity  $id_X = \nabla_X ; c(\nabla_X, X, X)$ and the second suggests

\begin{itemize}
\item $id_A \otimes c(f, A, B_1 \otimes B_2) = (id_{A \otimes A} \otimes \nabla_X) ; (c(m(f, A \otimes B_1, B_2), A, B_1) \otimes id_X) ; (\nabla_{A \otimes B_1} \otimes id_X) ; (id_{A \otimes B_1} \otimes c(f, A, \otimes B_1, B_2))$
\end{itemize}


We also have immediately arrows that are akin to projections, and precisely $m(id_{A \otimes B}, A, B): A \otimes B \to A$. Note however that
$c(id_{A \otimes B}, A, B):  A \otimes A \otimes B \to B$, and the meaning of the latter as well as of decomposition of $id_{A \otimes B}$ has to be better investigated.
In general, what happens for composite arrows $f \otimes g$?

As long as we assume to have the identity of the monoid $1$ and strictly $A \otimes 1 = A$ we obtain dischargers $m(id_A, 1, A): A \to 1$ as well as a law reminiscent of quasi-totality, 
since for $f: X \to A$ we have
$f = \nabla_X ; (m(f, 1, A) \otimes c(f, 1, A))$. Nevertheless, while requiring $c(id_A, 1, A) = id_A$ seems reasonable, since it basically means that $m(id_A, 1, A)$ is in fact a discharger, I am not sure about 
$c(f, 1, A) = f$, which is not true in some examples. In particular, it is not true in the canonical interpretation of marginals and conditionals in a gs-monoidal category, since quasi-totality holds only 
up-to pre-order equivalence.

We also have $m(f, A, 1): X \to A$, and it seems reasonable to have $m(f, A, 1) = f$, while $c(f, A, 1): A \otimes X \to A$ should be further investigated. 

\section{Share categories and garbage-share categories}

\begin{definition}
	A \emph{share category} or \emph{copy category} is a symmetric monoidal category $(C,\otimes,I)$ together with, for every object $X$, a distinguished map $\nabla_X:X\to X\otimes X$
	\ctikzfig{share}
	which is coassociative and cocommutative, i.e.\ such that the following equations hold
	\ctikzfig{coass}
	\ctikzfig{cocomm}
	
	Moreover, we require that these maps are compatible with the tensor product as follows
	\ctikzfig{share_I}
	\ctikzfig{share_tens}
\end{definition}

Note that we do not require the maps $\nabla$ to form a natural transformation.

\begin{definition}
	A \emph{category with garbage} (``garbage category'' sounds weird!) or \emph{discard category} is a symmetric monoidal category $(C,\otimes,I)$ together with, for every object $X$, a distinguished map $!_X:X\to I$ 
	\ctikzfig{bang}
	compatible with the tensor product, meaning that the following equations hold:
	\ctikzfig{bang_I}
	\ctikzfig{bang_tens}
\end{definition}

Note that we do not require the maps $!$ to form a natural transformation. If we did, we would have exactly a semicartesian monoidal category.

\begin{definition}
	A \emph{garbage-share} (\emph{gs}) or \emph{copy-discard} (\emph{CD}) category is a symmetric monoidal category $(C,\otimes,I)$ with a share and a garbage structure
	\[
	\tikzfig{share}
	\qquad\qquad
	\tikzfig{bang}
	\]
	which are compatible in the following (\emph{counital}) way
	\ctikzfig{counital}
\end{definition}

\section{Marginals and garbage}

We would like now to axiomatize the formation of marginals.
First of all, if we have a garbage structure, marginals can be defined in a canonical way.

\begin{definition}
	Let $(C,\otimes,I,!)$ be a garbage category. 
	Given a morphism $f:X\to A\otimes B$, its \emph{first marginal} $f_A:X\to A$ is defined as follows.
	\ctikzfig{marg_A}
	Its second marginal is defined analogously, and so are other marginals for morphisms with more than two outputs.
\end{definition}

Note that if we have a garbage \emph{and} a share structure, we have a garbage-share structure (i.e.\ counitality holds) if and only if the identity is a marginal of the share map.

Let's now define marginals abstractly, without assuming a garbage structure.

\begin{definition}
	Let $\mathcal{C}:=(C,\otimes,I)$ be a symmetric monoidal category. A \emph{ class of abstract marginals} for $\mathcal{C}$  consists of a pair of morphisms $f_A^1:X\to A$ and $f_A^2:X\to B$  for every arrow $f:X\to A\otimes B$ of $\mathcal{C}$,  satisfying the following conditions
	\begin{itemize}
		\item $(f\otimes g)_{X\otimes Y }^1=f \otimes (id_{C\otimes I}^2 g_Y^1)$ with $g:Y\to C\otimes D$;
		\item $(f\otimes g)_{X\otimes Y }^2=(id_{I\otimes B }^1 f_X^2) \otimes g$ with $g:Y\to C\otimes D$.
		\item $id_{I\otimes I}^1=r_I$ and $id_{I\otimes I}^2=l_I$.
	\end{itemize}
\end{definition}
\begin{lemma}
	A symmetric monoidal category with a class of abstract marginals is a garbage category.
\end{lemma}
(...)


\bibliographystyle{plain}
\bibliography{references}
\end{document}